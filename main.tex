\documentclass{article}

\usepackage[spanish,es-noindentfirst]{babel}

\usepackage[margin=1in]{geometry}

\usepackage[utf8]{inputenc}
\usepackage[T1]{fontenc}

\title{Idolatria e Iconoclastia}
\author{Gustavo Serrano \\ \tt{gustavo.serrano.diez@gmail.com}}
%\date{March 2015}

\usepackage{natbib}
\usepackage{graphicx}

\setcounter{tocdepth}{1}

\begin{document}

\maketitle

\begin{abstract}
\noindent
Esta es mi posici\'on respecto a los comentarios que me hiciste llegar la \'ultima vez.
\end{abstract}

\tableofcontents

\section{Introducci\'on}

Llamo \emph{argumento de excepci\'on} a aquel que enuncia que la \'unica raz\'on para apartarse de lo que dice alg\'un mandamiento de la Ley de Dios seg\'un Ex 20,3-17 y Dt 5,7-21 es precisamente cuando Dios ordena expl\'{i}citamente lo contrario; en tal caso, ser\'{i}a pecado no obedecerle.

Las citas b\'{i}blicas textuales est\'an acompa\~nadas de la versi\'on de que fueron tomadas de la siguiente forma:

\begin{description}
\item[LBJ] Biblia de Jerusal\'en 1967
\item[NC] Biblia N\'acar Colunga 1944
\item[RVR] Biblia Reyna-Valera 1960
\item[LH] Liturgia de las horas
\end{description}

\noindent
En caso de ser meras referencias omitir\'e tal abreviaci\'on. Si son citas no textuales antepondr\'e \emph{cf}.

En el caso de que una cita b\'{i}blica forme parte de la cita textual de otra fuente, la dejar\'e tal cual aparece en la fuente citada.

\section{Las dos iglesias cat\'olicas}

As\'{i} que decidiste dejar la iglesia cat\'olica. Ha sido la decisi\'on m\'as importante de tu vida. Seguramente no lo hiciste a la carrera. Te tomaste tu tiempo y te aseguraste de que fuera la decisi\'on correcta. 

Cuando el Se\~nor te habl\'o, te hizo ver tu pecado. Fuste obediente a su Palabra y saliste de las tinieblas de la idolatr\'{i}a para entrar en su luz admirable (\emph{cf.} 1 Pe 2,9).

Sin embargo creo que no conociste la Iglesia que abandonaste. Debo contarte una historia para que me entiendas mejor.

\subsection{Una an\'ectdota m\'as bien personal}

Cuando era adolescente, yo estaba convencido de que en la Iglesia hab\'{i}a dos clases de cat\'olicos: los de \emph{nombre} y los practicantes.

Para mi era tan clara la distinci\'on entre ambos grupos que f\'acilmente pod\'{i}a se\~nalar qui\'en estaba en un grupo y qui\'en en otro.

Entonces yo estaba convencido de que los cat\'olicos de \emph{nombre} deb\'{i}an dejar de ser cat\'olicos y admitir su ate\'{i}smo o su paganismo en lugar de estar desprestigiando con su mala conducta a los cat\'olicos de \emph{a deveras}.

Debo admitir que yo estaba francamente equivocado y que la cosa no es tan simple. ?`Qui\'en decide en cu\'al grupo est\'a cada qui\'en? Por algo Dios se reserva el juicio para s\'{i} mismo y nos advierte que quien juzga al hermano \emph{no es cumplidor de la ley, sino su juez} (\emph{cf.} Sant 4,11-12).

Sin embargo, creo que la distinci\'on nos puede ayudar para aclarar algunas ideas.

\subsection{La iglesia cat\'olica que dejaste}

\begin{flushright}
\emph{{\ldots}y en su frente un nombre escrito, un misterio:\\
<<Babilonia la Grande, la madre de las rameras\\
y de las abominaciones de la tierra>>}.\\
Ap 17,5 RVR
\end{flushright}

\noindent
Me contaste que \emph{te diste cuenta de tu pecado y abandonaste la Iglesia Cat\'olica para buscar al Se\~nor y servirle en su Palabra escrita}. Seguramente fue la decisi\'on m\'as importante y dif\'{i}cil de tu vida. Con certeza no fue lo que deseabas inicialmente y tuviste que vencer muchas resistencias tanto internas como externas\ldots hasta que, citando tus palabras, \emph{te diste cuenta de tu pecado} y decidiste salir.

%todo: todas las abreviaturas deben ser iguales
Aunque no me has contado mayores detalles puedo imaginarlo: una comunidad enferma (si es que a eso se le puede llamar comunidad), en la que el mandato del amor al pr\'ojimo (Lv 19,18; Mc 12,29-31; Jn 13,34) fue reemplazado por el odio al hermano (Ez 34); el conocimiento de Dios (Jn 17,3), por la posesi\'on de im\'agenes como si fueran amuletos (Is 45,20), abandonando a Dios, \emph{fuente de agua viva} y construyendo \emph{cisternas rotas} (Jer 2,13); el culto en esp\'{i}ritu y en verdad (Jn 4,23-24), por procesiones que lucen en la calle para que la gente los admire (Mt 6,1.2.5); la oraci\'on en lo \'{i}ntimo del coraz\'on (Mt 6,6), por la vana palabrer\'{i}a (Mt 6,7); el mandato de Dios, por las tradiciones de hombres (Mt 15,3.6; Mc 7,8-13)\ldots

Esperabas reconocer a los disc\'{i}pulos del Se\~nor por el amor que se tienen (Jn 13,34) y s\'olo encontraste una turba, como dice la Escritura:

\begin{quote}
<<\ldots atestados de toda injusticia, fornicación, perversidad, avaricia, maldad; llenos de envidia, homicidios, contiendas, engaños y malignidades; murmuradores, detractores, aborrecedores de Dios, injuriosos, soberbios, altivos, inventores de males, desobedientes a los padres, necios, desleales, sin afecto natural, implacables, sin misericordia; \emph{quienes habiendo entendido el juicio de Dios, que los que practican tales cosas son dignos de muerte, no sólo las hacen, sino que también se complacen con los que las practican}>>\\
Rm 1,29-32 RVR
\end{quote}

\noindent
Y pensaste <<\emph{esa es la gente ignorante, pero en la comunidad parroquial, donde la gente es conocedora, seguramente las cosas son de distinta manera}>> ?`Cu\'al fue tu sorpresa cuando participaste en el consejo parroquial, de que no s\'olo practicaban tales cosas, sino que a\~nad\'{i}an m\'as delitos por su cuenta? por decir algunos: los dos \'unicos temas en las juntas de consejo parroquial eran: a) ?`c\'omo sacarle dinero a la gente? y b) murmuraciones de unos contra otros

A esto tal vez agregar\'{i}as que nadie se interesaba por conocer realmente la Palabra de Dios: ten\'{i}an biblias muy bonitas -- y costosas -- de la primera comuni\'on, de la boda, etc. todas ellas empolvadas porque nunca las abrieron. Si les hubieras hecho un examen -- a libro abierto! -- diciendo que te copiaran: Jn 3,16 estabas seguro de que la mitad no sab\'{i}a lo que era "Jn", de la mitad que sab\'{i}a eso, la mitad ignoraba si era el Antiguo o el Nuevo Testamento, los que s\'{i} sab\'{i}an, se estar\'{i}an preguntando: <<?`qu\'e es eso de 3,16?>>.

Y al adentrarte por tu cuenta en la Palabra divina, \emph{te llovi\'o condenaci\'on, pero tambi\'en, m\'aximas de salvaci\'on}, as\'{i} que decidiste despojarte del hombre viejo y revestirte del hombre nuevo.

D\'ejame decirte que lo que conociste de la Iglesia Cat\'olica es una caricatura superficial, una comunidad enferma cuya fe sin obras est\'a muerta (\emph{cf}. Sant 2,17), pero \textbf{esa no es TODA la Iglesia Cat\'olica, ni siquiera la mitad}.

Te quedaste sin conocer la Iglesia que abandonaste; no conociste al Cuerpo sino \'unicamente un miembro enfermo, moribundo.

A continuaci\'on te voy a mostrar algunas de las cosas que te quedaste sin conocer

\subsection{La Iglesia Cat\'olica que \emph{no} conociste}

\begin{flushright}
\emph{<<Ven, te mostrar\'e a la desposada,\\
la Esposa del Cordero>>}.\\
\emph{cf.} Ap 21,9
\end{flushright}

\noindent
En esta ocasi\'on voy a hablar s\'olamente en t\'erminos humanos. El sentido espiritual y b\'{i}blico es tan extenso y profundo que requerirá un tratado aparte.

Antes de contarte mi experiencia personal, perm\'{i}teme relatarte dos ejemplos actuales en el plano internacional:

\begin{description}
\item[Asia Bibi (Pakistán)]%
        \footnote{http://asiabibi.org/?lang=es}
    Desde 2009 está en el corredor de la muerte por confesar su fe y negarse a abrazar el Islam.%
        \footnote{http://www.libertaddigital.com/mundo/prefiero-morir-como-cristiana-en-la-horca-que-salir-de-prision-siendo-musulmana-1276407149/}
    Se mantiene firme, dispuesta a morir por Cristo como atestigua la carta enviada a su familia.%
        \footnote{http://www.religionenlibertad.com/sale-a-la-luz-la-emotiva-carta-de-asia-bibi-a-20869.htm}
    Ayuna y ofrece sus sufrimientos por sus compañeros de prisión y por su país.%
        \footnote{http://www.larazon.es/historico/3958-asia-bibi-reza-y-ayuna-desde-su-celda-ILLA\_RAZON\_377708}

\item[China]
    «\emph{Abrid las puertas para que entre un pueblo justo,/
    que observa la lealtad;/
    su ánimo está firme y mantiene la paz,/
    porque confía en ti}» (Is 26,2-3 LH) Estas líneas se aplican a la perfección a la Iglesia en China, no necesito darte muchas explicaciones, el testimonio del \emph{hombre celestial} no es diferente del de los católicos en aquel país.%
        \footnote{http://maslibres.org/?p=19779}
    
\item[Mi experiencia personal]
    Puedo nombrar a «una nube de testigos» (\emph{cf.} Heb 12,1), entre laicos, religiosos y sacerdotes; jóvenes y viejos, hombres y mujeres; cuya fe atestiguada por las obras (Sant 2,14-26) ha brillado ante los hombres para que glorifiquen a Dios (\emph{cf.} Mt 5,16), se aman unos a otros, siguen al Señor, oran sin cesar, estudian su Palabra, se perdonan mutuamente y no dan motivo de queja a nadie\ldots son el arbol bueno que se conoce por sus frutos (Mt 7,15-20, Lc 6,43-44)
\end{description}

\subsection{Para reflexionar}

¿Cómo es posible que en la misma Iglesia se encuentren realidades tan contrapuestas? Medita en tu coraz\'on la par\'abola del trigo y la ciza\~na (Mt 13,24-30).

\section{Invalidez del argumento de excepci\'on}

A continuaci\'on voy a mostrar suscintamente por qu\'e el argumento de excepci\'on es inv\'alido:

\begin{itemize}
\item Primero describir\'e a fondo en qu\'e consiste este argumento seg\'un he entendido todo lo que me has escrito.%
    \footnote{Pongo mi mayor empe\~no en no caricaturizar o minimizar tu argumento sino expresarlo con toda la solidez que tiene. Ya has recibido el resumen que hice a tus observaciones y la presente descripci\'on, aunque breve, pretende seguir la misma l\'{i}nea y consideraci\'on.}
\item Posteriormente expondr\'e las dos razones por las que el argumento carece de validez: a) para el mandamiento <<no matar\'as>> y b) para el caso de la idolatr\'{i}a.
\end{itemize}

\subsection{?`En qu\'e consiste el argumento de excepci\'on?}

El argumento central de tu respuesta se basa en la siguiente premisa:

\begin{quote}
\emph{La \'unica raz\'on para contravenir un mandamiento expl\'{i}cito de Dios es que Dios mismo haya ordenado otra cosa. Siendo \'unicamente cuando Dios lo manda y sin hacerlo por cuenta propia.}
\end{quote}

\noindent
Las caracter\'{i}sticas de una excepci\'on son las siguientes:

\begin{enumerate}
\item Dios dispone \emph{expl\'{i}citamente} una excepci\'on a alguno de sus mandamientos
    
    Por \emph{expl\'{i}cito} me estoy refiriendo a que simult\'aneamente: a) est\'a escrito en la Biblia, b) con una claridad que no deja lugar a dudas y c) es Dios el autor de la disposici\'on, no los hombres

\item S\'olo aplica para aquellos casos y personas que Dios dispone
\item En el tiempo y la forma en que \'El lo dispone
\item No es l\'{i}cito al hombre tomar esos casos para hacer \emph{excepciones} por cuenta propia
\item No es l\'{i}cito al hombre tomar esos casos para hacer \emph{ampliaciones} por cuenta propia
\item Siendo mandato expreso de Dios, es pecado no obedecerle
\end{enumerate}

\noindent
La primera caracter\'{i}stica enunciada es la que determina si se trata de una excepci\'on v\'alida o no, debe cumplirse a cabalidad. Las dem\'as caracter\'{i}sticas son las consecuencias e implicaciones. El ejemplo, claro y di\'afano est\'a expresado de la siguiente manera:

\begin{quote}
\emph{Dios dice: <<no matar\'as>>, el \'unico caso en que es permitido matar a un hombre es cuando Dios lo ordena, porque Dios lo manda, s\'olo cuando as\'{i} lo indica. En este caso es pecado no obedecerle. Pero no es l\'{i}cito al hombre tomar este ejemplo para hacer por cuenta propia sus <<excepciones>>.}
\end{quote}

\subsection{Para el mandamiento <<no matar\'as>> hay excepciones no fijadas por Dios}

Sin embargo para el caso <<no matar\'as>> hay un caso en el que es l\'{i}cito a un hombre matar a otro y sin embargo no es mandado por Dios expl\'{i}citaemente; me refiero al caso de la \emph{leg\'{i}tima defensa}.

\subsubsection{?`En qu\'e consiste la leg\'{i}tima defensa?}

Tenemos polic\'{i}as y ej\'ercito; y los tenemos armados; y esperamos que usen sus armas para defendernos incluso si eso significa matar al agresor; adem\'as, podemos vernos en la necesidad de defendernos por nuestra cuenta sin el auxilio de los poderes p\'ublicos.

Bajo el t\'ermino \emph{leg\'{i}tima defensa} estoy abarcando tres categor\'{i}as relacionadas: a) a nivel individual es la propiamente llamada \emph{leg\'{i}tima defensa}; b) la defensa de los ciudadanos realizada en tiempos de paz por la polic\'{i}a y poderes p\'ublicos equivalentes; y c) la as\'{i} llamada \emph{guerra justa}.\footnote{La \emph{pena de muerte} es considerada en los textos cl\'asicos como un caso particular en que una comunidad (ciudad o estado) se defiende de un agresor, siendo su \'ultimo recurso el privarle la vida. Trat\'andose de un tema muy controversial prefiero dejarlo de lado.}

Los tres casos mencionados tienen en com\'un dos caracter\'{i}sticas: a) se trata de acciones donde el que las ejerce pudiera privar de la vida a un agresor; incluso deliberadamente y b) son acciones consideradas \emph{defensivas} cada una seg\'un su \'ambito y nivel.

?`Son estos casos aceptables? ?`bajo qu\'e condiciones? Parece ser que no toda intervenci\'on armada es justa, incluso si es meramente defensiva; determinar la legitimidad de matar en defensa propia exige condiciones m\'as precisas:%
    \footnote{Cito aqu\'{i} el n\'umero 2309 del Catecismo de la Iglesia Cat\'olica, por ser el que lo expresa de la manera m\'as concisa; las causas enunciadas son las de la \emph{guerra justa} pero no te costar\'a trabajo hacer las adecuaciones para los dem\'as casos de la leg\'{i}tima defensa. Puedes consultarlo en la siguiente liga: http://www.vatican.va/archive/catechism\_sp/p3s2c2a5\_sp.html}

\begin{enumerate}
\item Que el daño causado por el agresor a la nación o a la comunidad de las naciones sea duradero, grave y cierto.
\item Que todos los demás medios para poner fin a la agresión hayan resultado impracticables o ineficaces.
\item Que se reúnan las condiciones serias de éxito.
\item Que el empleo de las armas no entrañe males y desórdenes más graves que el mal que se pretende eliminar. El poder de los medios modernos de destrucción obliga a una prudencia extrema en la apreciación de esta condición.
\end{enumerate}

\noindent
Las condiciones arriba citadas para la validez de la leg\'{i}tima defensa no est\'an \emph{expl\'{i}citamente} enunciadas en la Biblia como veremos a continuaci\'on.

\subsubsection{Argumentaciones b\'{i}blicas insuficientes}

Aparentemente es de sentido com\'un que existen condiciones que hacen l\'{i}cito el uso de la fuerza para defenderse, incluso si como consecuencia el defensor da muerte al agresor; sin embargo, no hay ning\'un vers\'{i}culo que lo diga con claridad y que permita saber sin ambig\"uedad bajo qu\'e condiciones aplica este principio.

Me tom\'e la libertad de buscar fundamentaciones b\'{i}blicas de tipo cristiano-evang\'elicas acerca de la leg\'{i}tima defensa; estos son los sitios que me parecen m\'as pertinentes y completos:%
    \footnote{En espan\~nol \'unicamente est\'a el estudio de Tito Mart\'{i}nez (citado y re-citado en m\'ultiples sitios web), no lo utilizo por dos razones: a) insuficiente, b) a juzgar por otros escritos suyos est\'a muy desviada su doctrina. Este es su estudio: http://www.las21tesisdetito.com/autodefensa.htm}

\begin{enumerate}
\item http://www.biblicalselfdefense.com/
\item http://www.kingjamesbibleonline.org/Bible-Verses-About-Self-Defense/
\item http://armedcitizensnetwork.org/the-bible-and-self-defense
\item http://www.nationalreview.com/corner/338845/biblical-and-natural-right-self-defense-david-french
\end{enumerate}

\noindent
Estas son las citas m\'as relevantes:%
    \footnote{Debo agradecerte este diálogo, aunque con una muy larga pausa, gracias a él terminé de leer la Biblia en su totalidad, siendo una gran bendición para mi; no obstante, en mi búsqueda no encontré nada distinto las citas que están a continuación.}

\begin{enumerate}
\item Neh 4,8-23
\item Est 8-9
\item Lc 22,35-38
\item Ap 11
\item Ex 22,2
\item Rom 13,1-4
\end{enumerate}

\noindent
Revisemos cada una de estas citas bajo los siguientes criterios:

\begin{description}
\item[Universalidad] se trata de un caso general no limitado a personas, grupos o tiempos particulares.
\item[Explicitez] se trata de una indicaci\'{o}n expl\'{i}cita y no de meras inferencias; est\'a hablando en sentido literal y no en sentido figurado.
\item[Autoridad] es Dios mismo el que lo indica as\'{i} y no los hombres por su cuenta.
\end{description}

\noindent
Revisemos cada uno de los pasajes y apliquemos a cada uno los criterios arriba citados.

\paragraph{Nehem\'{i}as 4,8-23}

El pasaje no deja ver con claridad ninguno de los tres criterios: a) \emph{Universalidad:} No es claro si aplica \'unicamente para ellos o para todo el que se encuentre en situaci\'on similar; b) \emph{Explicitez:} No es una indicaci\'on expl\'{i}cita; M\'as a\'un, de hecho \emph{no combatieron} y el v.20 al decir <<Dios pelear\'a por nosotros>> no permite saber si literalmente los israelitas no tendr\'an necesidad de pelear o bien si en caso de hacerlo, Dios les dar\'{i}a la victoria; y c) \emph{Autoridad:} No es Dios el que les da la indicaci\'on.

\paragraph{Esther 8-9}

Este pasaje carece de \emph{universalidad} y de \emph{autoridad} para ser considerado un caso v\'alido de leg\'{i}tima defensa. Fue un caso particular en un momento particular y Dios no se los orden\'o expl\'{i}cita o impl\'{i}citamente. Es, en cambio, manifiesto que los israleitas fueron autorizados por parte del rey pagano Asuero para quitar la vida a aquellos que los amenazaban y de hecho as\'{i} lo hicieron.

No hay ninguna indicaci\'on acerca de la aprobaci\'on o rechazo por parte de Dios a estas acciones simplemente se anota que de hecho sucedieron.

\paragraph{Ap 11}

Este pasaje habla de dos testigos y muestra claramente como se defienden de cualquiera que atente contra su vida (\emph{explicitez}) y que lo hacen con la aprobaci\'on y poder de Dios (\emph{autoridad}).

En el texto es muy claro que \'unicamente se refiere a los dos testigos mientras profetizan durante 1260 d\'{i}as. No puede aplicarse fuera de este caso. Carece de \emph{universalidad}.

\paragraph{Lucas 22,35-38}

Aqu\'{i} existe toda la \emph{autoridad} pues es Jes\'us el que habla.

Sin embargo, este pasaje carece de \emph{univeralidad} para ser aplicado en la leg\'{i}tima defensa, pues no permite ver si su aplicaci\'on es \'unicamente en ese momento o bien en cualquier situaci\'on de peligro.

Por otro lado, no es suficientemente \emph{expl\'{i}cito} si, en el caso de \emph{comprar una espada}, se refiere literalmente a comprar una o est\'a hablando en sentido figurado. Tampoco menciona el tema de privar la vida a un agresor. Y es evidente que Jes\'us NO se estaba refiriendo a resistir a los soldados de los sumos sacerdotes, pues \emph{ten\'{i}a que cumplirse la Escritura}.

Cuando en el v.38 los ap\'ostoles le dicen a Jes\'us que tienen dos espadas, \'El les responde: <<Basta>> (\emph{c.f.} Lc 22,38); pero no queda claro si <<Basta>> quiere decir \emph{con eso es suficiente para defendernos} o bien \emph{no me entendieron as\'{i} que dejemos de hablar de <<espadas>>}. De hecho la expresi\'on griega $\iota\kappa\alpha\nu o \nu$ (suficiente) $\varepsilon\sigma\tau\iota\nu$ (est\'a siendo) tiene ambas acepciones.%
    \footnote{Puedes consultar cualquier Nuevo Testamento griego-espa\~nol interlineal y alg\'un diccionario de l\'exico griego del Nuevo Testamento.}
    
Por mi parte, aunque no estoy cerrado a la interpretaci\'on literal, me inclino m\'as a que Jes\'us estaba hablando en sentido figurado y los ap\'ostoles lo entendieron en sentido literal; pues aquel es el que va m\'as acorde con todo el pasaje, desde la \'ultima cena hasta que Jes\'us es llevado ante el sumo sacerdote.%
    \footnote{Puedes consultar un interesante estudio en la siguiente liga: http://www.biblestudytools.com/commentaries/utley/lucas/lucas22.html}

\paragraph{Ex 22,2}

En este caso s\'{i} se dan los tres elementos como indicamos anteriormente: \emph{universalidad, explicitez y autoridad}. Sin embargo, \'unicamente constituye un caso que es el del robo y no habla de mayores circunstancias que la de que \emph{es de noche}.

En los estudios consultados interpretan esta circunstancia diciendo que de noche significa que no es posible saber si el invasor va a robar o va a matar. Por esta raz\'on su muerte no es imputable, mientras que el d\'{i}a representa la claridad de intenciones y en tal caso, la muerte es imputable, pues la vida vale m\'as que la propiedad.

Por tanto, a partir de este texto, \'unicamente de manera parcial se puede inferir el principio de leg\'{i}tima defensa.

\paragraph{Rom 13,1-4}

Al igual que en Ex 22,2 se cumplen los tres criterios a cabalidad. Con todo, se trata de un \'unico caso, a saber: el derecho que tiene la autoridad para usar un poder letal (la espada) como instrumento para mantener el orden.

En este caso, la Biblia indica con claridad el \emph{sujeto} autorizado a usar la espada: el poder civil. A\'un as\'{i}, no hay informaci\'on suficiente acerca del \emph{modo} adecuado de usar la espada y los \emph{casos} en que es l\'{i}cito privar la vida a otro en el ejercicio de su deber.

Despu\'es de revisar los pasajes m\'as relevantes podemos formular algunas conclusiones y analizar las consecuencias.

\subsubsection{Conclusiones y Consecuencias}

Los pasajes anteriormente citados ponen de manifiensto que \textbf{el \emph{argumento de excepci\'on} carece de \emph{solidez} as\'{i} como lo has formulado}; esto por dos razones:

\begin{enumerate}
\item Las citas b\'{i}blicas analizadas son insuficientes para establecer de manera congruente y \emph{expl\'{i}cita} la existencia el principio de la leg\'{i}tima defensa siguiendo tu l\'ogica sobre las excepciones.
\item A\'un suponiendo que en los pasajes citados pudiera deducirse de manera impl\'{i}cita la leg\'{i}tima defensa, no hay informaci\'on suficiente en la Biblia para establecer con precisi\'on las circunstancias y modos de efectuarla.
\end{enumerate}

\noindent
\emph{De aqu\'{i} que sean los hombres quienes mediante el uso de la raz\'on que Dios les dio, provean estos casos y sus l\'{i}mites} con lo cual se invalida el \emph{argumento de excepci\'on}.

Siendo inv\'alido el argumento de excepci\'on para el mandamiento <<no matar\'as>>, podemos concluir que, adem\'as de las excepciones expl\'{i}citamente se\~naladas por Dios, existan exepciones no definidas o explicitadas por Dios en que es l\'{i}cito a un hombre privar de la vida a otro hombre.

\subsection{El argumento de excepci\'on no es aplicable al caso de la idolatr\'{i}a}

A\'un suponiendo que el argumento de excepci\'on fuera v\'alido para Ex 20,13 <<no matar\'as>> todav\'ia tropieza con un obst\'aculo: no es aplicable al caso de la idolatr\'{i}a.

La siguiente parte del estudio comienza por definir el t\'ermino \emph{idolatr\'{i}a} y posteriormente analiza si Dios hace excepciones para este caso. Inmediatamente despu\'es hacemos un an\'alisis a profundidad de Ex 20,4-6 revisando enunciado por enunciado si Dios hace excepciones a su mandamiento.

\subsubsection{Definicion de idolatr\'{i}a}

Del griego $\varepsilon\iota\delta\omega\lambda o$ y $\lambda\alpha\tau\rho\varepsilon\iota\alpha$; el vocablo $\varepsilon\iota\delta\omega\varsigma$ significa lo aparente, un reflejo sin realidad, un fantasma, una \emph{imagen} en sentido mental; la palabra \emph{idea} se deriva de este vocablo; por otra parte, $\lambda\alpha\tau\rho\varepsilon\iota\alpha$ significa adoraci\'on.

Bas\'andonos \'unica y exclusivamente en la etimolog\'{i}a, la definici\'on de \emph{idolatr\'{i}a} ser\'{i}a: \emph{adoraci\'on de las apariencias}.

Pero no basta con analizar el vocablo y su ra\'{i}z griega, es preciso conocer el uso que se le da en la Biblia; ahora bien, la versi\'on de los LXX emple\'o la palabra $\varepsilon\iota\delta\omega\lambda o$ para designar varios t\'erminos hebreos, la Gran Enciclopedia Rialp los enuncia de la siguiente manera:

\begin{quote}
Idolo (en griego eídólon) es la traducción más común de unos nombres hebreos, diversos entre sí. La palabra eídólon significa propiamente la imagen, el fantasma forjado por la fantasía. En la traducción del A. T. al griego por los Setenta se emplea para designar unas realidades más concretas, expresadas en el original hebreo por voces diversas: Selem: que significa «talla», «escultura» (Num 33,52); 'Asabbim: usado siempre en plural, significa «imagen tallada» (1 Sam 31,9); Semel: nombre de origen fenicio; significa «estatua de piedra» o «de madera» (Ez 8,3.5); Massékáh: «imagen fundida», en molde de arcilla (Ex 32,4.8); 'Eben maskith: «piedra con alguna imagen tallada» (Lev 26,1). La palabra maskith no significa necesariamente una imagen idolátrica; puede designar las imaginaciones de la fantasía. etc.\\
(Gran Enciclopedia Rialp)\footnote{No he podido dar con la versi\'on completa \emph{online}, en la siguiente liga hay algunos extractos: http://www.mercaba.org/Rialp/I/idolatria\_escritura.htm}
\end{quote}

\noindent
Teniendo esto en cuenta, podemos ampliar la definici\'on inicial para que quede como sigue:

\begin{quote}
El idólatra es el que <<\emph{aplica a cualquier cosa, en lugar de a Dios, la indestructible noción de Dios}>>\footnote{La cita es de un autor cristiano del siglo III: Orígenes, \emph{Contra Celsum}, 2, 40} y act\'ua en consecuencia: d\'andole \emph{adoraci\'on}.
\end{quote}

\noindent
Esta definici\'on abarca una serie de actitudes y acciones como las que sintetiza el Catecismo de la Iglesia Cat\'olica en su n\'umero 2113:

\begin{quote}
La idolatría no se refiere sólo a los cultos falsos del paganismo. Es una tentación constante de la fe. \emph{Consiste en divinizar lo que no es Dios. Hay idolatría desde el momento en que el hombre honra y reverencia a una criatura en lugar de Dios.} Trátese de dioses o de demonios (por ejemplo, el satanismo), de poder, de placer, de la raza, de los antepasados, del Estado, del dinero, etc. <<No podéis servir a Dios y al dinero>>, dice Jesús (Mt 6, 24). Numerosos mártires han muerto por no adorar a <<la Bestia>> (cf Ap 13-14), negándose incluso a simular su culto. La idolatría rechaza el único Señorío de Dios; es, por tanto, incompatible con la comunión divina (cf Gál 5, 20; Ef 5, 5).\footnote{http://www.vatican.va/archive/catechism\_sp/p3s2c1a1\_sp.html El subrayado es m\'{i}o.}
\end{quote}

\subsubsection{Dios no ordena la idolatr\'{i}a en ning\'un caso}

No hay NINGUN caso en toda la Escritura donde Dios ordene o permita que se de adoraci\'on a ning\'un otro que no sea \'El mismo.

Las citas sobre el Arca, la decoraci\'on del Templo y la serpiente de bronce no indican adoraci\'on a tales objetos; no pueden ser consideradas \emph{casos en los que Dios permite la idolatr\'{i}a} pues no hay \emph{adorac\'on} a tales objetos.

Puedes revisar 2 Re 18,3-4 para saber lo que le sucedi\'o a la serpiente de bronce cuando el pueblo de Israel le rindi\'o culto de adoraci\'on.

Por lo tanto \textbf{el \emph{argumento de excepci\'on} es \emph{inv\'alido} para el caso de idolatr\'{i}a}.

\subsubsection{Ci\~n\'endonos a lo que dice literalmente Ex 20,4-6}

Cuando dijiste en el comentario que me enviaste que \emph{lo mismo aplica para el caso de la idolatr\'{i}a} lo m\'as probable es que no te estuvieras refiriendo a la \emph{adoraci\'on}; pero ?`qu\'e hay entonces de atenerse a lo que dice literalmente Ex 20,4-6?

\begin{quote}
<<$^4$No te har\'as esculturas ni imagen alguna de lo que hay en lo alto de los cielos, ni de lo que hay abajo sobre la tierra, ni de lo que hay en las aguas debajo de la tierra. $^5$No te postrar\'as ante ellas, y no las servir\'as, porque yo soy Yahveh, tu Dios, un Dios celoso, que castiga en los hijos las iniquidades de los padres hasta la tercera y cuarta generaci\'on de los que me odian, $^6$y hago misericordia hasta mil generaciones de los que me aman y guardan mis mandamientos.>>\\ 
Ex 20,4-6 NC
\end{quote}

\noindent
En el pasaje arriba citado podemos identificar claramente cuatro partes:

\begin{enumerate}
\item La acci\'on material: <<No te har\'as esculturas\ldots>> (v4)
\item La acci\'on corporal: <<No te postrar\'as ante ellas>> (v5a)
\item La acci\'on espiritual: <<y no las servir\'as>> (v5b)
\item El motivo y las consecuencias: <<porque yo soy Yahveh, un Dios celoso\ldots>> (vv5c-6)
\end{enumerate}

\noindent
Revisemos con detenimiento las tres primeras partes (vv4-5b) pues constituyen el mandato de Dios.

\subsubsection{Dios, en algunos casos manda hacer im\'agenes y objetos semejantes}

Son los casos ya mencionados del Arca, la decoraci\'on del Templo y la serpiente de bronce. Y constituyen el centro del argumento de excepci\'on aplicado a Ex 20,4-6.

La \emph{premisa mayor}\footnote{Busca un manual de l\'ogica si tienes dudas sobre este t\'ermino, por ejemplo: http://es.wikipedia.org/wiki/Premisa} parece ser que Dios, en Ex 20,4-6; prohibe \emph{tres} cosas: a) Hacer im\'agenes (v4), b) inclinarse ante ellas (v5a) y c) servirlas (v5b). Vuelve a leer este p\'arrafo, es esencial para entender lo que sigue.

Suponiendo que esta tesis fuera cierta, revisemos en este apartado la primera prohibici\'on, dejando las otras dos para los apartados siguientes.

El argumento de \emph{excepci\'on} aplicado a la primera prohibici\'on quedar\'{i}a enunciado como sigue:

\begin{quote}
\emph{Ex 20,4 NC dice: <<No te har\'as esculturas ni imagen alguna de lo que hay en lo alto de los cielos, ni de lo que hay abajo sobre la tierra, ni de lo que hay en las aguas debajo de la tierra>>. El \'unico caso en que le es l\'{i}cito al hombre hacer im\'agenes es cuando Dios se lo indica, a quienes se lo indica, como se lo indica; tales son los casos del Arca, el Templo y la serpiente de bronce. No le es l\'{i}cito al hombre tomar estos ejemplos para hacer sus propias r\'eplicas o sus propias im\'agenes o dise\~nos por muy buenas y sinceras que sean sus intenciones.}
\end{quote}

\noindent
Est\'a muy claro: el Arca, el Templo y la serpiente de bronce son las \'UNICAS excepciones dispuestas por Dios. NO HAY M\'AS EXCEPCIONES. !`NINGUNA! No es l\'{i}cito a los hombres inventarse sus propios casos, no le est\'a permitido hacer sus propias r\'eplicas\ldots tampoco \emph{interpretar} la Escritura para <<anular el mandamiento de Dios con la tradici\'on>> (\emph{c.f.} Mc 7,9).

Por ejemplo: hacer un becerro de oro (Ex 32), una r\'eplica de Baal o Astart\'e (Num 25,3; Jue 2,13) o una estatuilla de Budha, Vishn\'u o Shiva.

Veamos algunos ejemplos un poco m\'as dif\'{i}ciles:

\begin{enumerate}
\item Tu credencial de elector con fotograf\'{i}a
\item Los animales de juguete de tus hijos
\item Hacer figuras de animales con plastilina
\item Las ilustraciones de la enciclopedia de tu familia, donde se ve la selva, las estrellas, el oc\'eano, \emph{lo que hay arriba en el cielo, abajo sobre la tierra y en las aguas debajo de la tierra} (\emph{c.f.} Ex 20,4).
\item Las fotos de tu madre, esposa e hijos
\end{enumerate}

\noindent
Creo deber\'{i}as deshacerte de esas abominaciones antes de acusarme de tener im\'agenes. Lo digo de verdad, si Ex 20,4-5b prohibe tres cosas entonces son tres cosas las que proh\'{i}be. No pretendo torcer tus argumentos a mi conveniencia, contin\'ua leyendo y ver\'as.

En estos momentos te viene a la mente la aclaraci\'on que me hiciste acerca de que en Ex 20,4 \emph{<<!`el punto malo es hacerlas con motivos religiosos!>>}, pues \emph{<<el hombre por motivos personales condenables en TODA la escritura es que se hace imágenes y las venera. ?`Y por qu\'e es condenable? precisamente porque las adoran y les sirven>>}; respondo con tus mismas palabras: \emph{<<NO FIJATE BIEN Y LEELO OTRA VEZ, ES HACERLAS>>}; no le a\~nadas a la Escritura tu interpretaci\'on diciendo: \emph{con motivos religiosos}. ?`En qu\'e libro de la Biblia, cap\'{i}tulo o vers\'{i}culo dice \emph{<<con motivos religiosos>>}?

Pero hay otra posibilidad: la posibilidad de que lo mencionado en Ex 20,4-6 se trate de \emph{una} sola prohibici\'on en lugar de tres; prohibici\'on en la cual el \emph{hacer im\'agenes} (v4) constituye el signo \emph{material} del acto \emph{interior} de servirlas (v5b).\footnote{M\'as adelante analizamos conjuntamente todo Ex 20,4-5b con m\'as precisi\'on.}

Si esto es as\'{i}, entonces \textbf{el \emph{argumento de excepci\'on} para la cita textual de Ex 20,4 es \emph{inv\'alido}, pues el v4 \emph{no puede separarse} del v5a-b}.

Antes de analizar todo el mandamiento en su conjunto, revisemos si es posible aplicar el \emph{argumento de excepci\'on} al texto: \emph{<<no te postrar\'as ante ellas>> (Ex 20,5a)}.

\subsubsection{Dios en ning\'un caso manda o permite postrarse ante las im\'agenes}

Ni siquiera ante las im\'agenes que \'El mismo manda realizar ordena o da permiso expl\'{i}cito para que los hombres se inclinen o se postren delante de ellas.

Sin embargo, tenemos el ejemplo \emph{expl\'{i}cito} de Josu\'e postr\'andose delante del Arca junto con todos los ancianos de Israel (Jos 7,6); implora el favor de Dios y \'El le responde, le aclara que el motivo de su disgusto es la prevaricaci\'on de los hijos de Israel, pues se apropiaron de objetos dados al anatema (Jos 7,1); sin embargo Yahveh no le reclama a Josu\'e ni a los ancianos el haberse postrado rostro en tierra \emph{delante} del Arca.

Podr\'as decirme que el Arca era especial y \'unica; que \emph{siendo dise\~nada por Dios no nos lleva a la corrupci\'on}\ldots pero el tema es otro, el tema es que Dios no da su consentimiento ni su mandato \emph{expl\'{i}cito} para lo que Josu\'e y los ancianos de Israel hicieron.

Por lo tanto \textbf{el \emph{argumento de excepci\'on} para la cita textual de Ex 20,5a es \emph{inv\'alido}}.

Revisemos ahora si el \emph{argumento de excepci\'on} tiene validez considerando Ex 20,5b antes de considerar todo el mandamiento en su conjunto.

\subsubsection{Dios en ning\'un caso manda servir a una imagen / estatua / pintura / etc}

Servir significa \emph{obedecer}, estar sujeto a la voluntad de otro, estar atento a sus deseos para cumplirlos.

Dios no ordena ni autoriza en ning\'un caso que se le de obediencia a las im\'agenes, ni siquiera a las que \'El mismo mand\'o hacer (el Arca, el Templo, la serpiente de bronce).

Por lo tanto \textbf{el \emph{argumento de excepci\'on} para la cita textual de Ex 20,5b es \emph{inv\'alido}}.

Estamos en condiciones para emitir una conclusi\'on general acerca del \emph{argumento de excepci\'on} para el caso de la idolatr\'{i}a. En la secci\'on siguiente analizaremos Ex 20,4-6 conjuntamente y con mayor profundidad pera tratar de aprehender cual es su significado.

\subsection{Conclusi\'on general sobre el \emph{argumento de excepci\'on} para el caso de la idolatr\'{i}a}

Mi conclusi\'on acerca del \emph{argumento de excepci\'on} es la siguiente: aunque no carece de l\'ogica; as\'{i} como lo has planteado, es decir, como la \emph{total exclusi\'on} de cualquier excepci\'on que no sea estrictamente \emph{se\~nalada por Dios} y totalmente \emph{particular} \textbf{\emph{NO TIENE VALIDEZ}}.

% En los casos analizados Ex 20,13 y Ex 20,4-6 hemos visto que no es aplicable. Si no es aplicable, entonces \emph{est\'a abierto a otras opciones}.

% Antes de analizar Ex 20,4-6 con mayor profundidad hagamos aqu\'{i} una anotaci\'on:

\noindent
Todav\'{i}a quedan muchas cuestiones por resolver para determinar si la pr\'actica cat\'olica de la veneraci\'on de las im\'agenes es conforme con las Santas Escrituras o por el contrario \emph{anula el mandato de Dios con la tradici\'on} (\emph{cf}. Mc 7,6-13).

En la siguientes dos secciones abordaremos el tema, retomando el hilo de nuestro an\'alisis sobre Ex 20,4-6 donde se qued\'o.

\section{El pecado de idolatr\'{i}a es contra el primer mandamiento}

Una vez visto que no es posible separar Ex 20,4 de Ex 20,5a-b debemos analizar el texto conjuntamente para tener alguna inteligencia de lo que Dios est\'a mandando.

Comenzaremos analizando Ex 20,4-5b consider\'andolo como una unidad; del an\'alisis brotar\'a que se trata de una ampliaci\'on de Ex 20,3, lo cual queda confirmado por la unidad arm\'onica que abre en el Ex 20,3 y cierra en Ex 20,6 pues de otra forma se rompe la sem\'antica del texto.

Despu\'es de esto, entraremos de lleno a la cuesti\'on de si la Iglesia Cat\'olica \emph{escondi\'o} el segundo mandamiento o si los protestantes \emph{dividieron} el primero.

En la secci\'on siguiente analizaremos la legitimidad de la pr\'actica cat\'olica acerca de las im\'agenes.

\subsection{Unidad arm\'onica de Ex 20,3-6/Dt 5,5-10}

Dec\'{i}amos m\'as arriba que en Ex 20,4-6 pod\'{i}amos identificar cuatro partes, esta divisi\'on se centr\'o en distinguir aquello que Dios manda de otras informaciones contenidas en el mandamiento. Ahora nos conviene profundizar m\'as y hacer la distinci\'on completa:\footnote{Los escol\'asticos dec\'{i}an: \emph{<<unir sin confundir y distinguir sin separar>>} para hacer este tipo de an\'alisis en el que se identifican diferentes partes que forman un todo.}

\begin{enumerate}
\item La acci\'on material: <<No te har\'as esculturas ni imagen alguna de lo que hay en lo alto de los cielos, ni de lo que hay abajo sobre la tierra, ni de lo que hay en las aguas debajo de la tierra>> (v4 NC)
\item La acci\'on corporal: <<No te postrar\'as ante ellas>> (v5a NC)
\item La acci\'on espiritual: <<y no las servir\'as>> (v5b NC)
\item El motivo: <<porque yo soy Yahveh, un Dios celoso>> (v5c NC)
\item Las consecuencias de la desobediencia: <<que castiga en los hijos las iniquidades de los padres hasta la tercera y cuarta generaci\'on de los que me odian>> (v5d NC)
\item Las consecuencias de la obediencia: <<y hago misericordia hasta mil generaciones de los que me aman y guardan mis mandamientos>> (v6 NC)
\end{enumerate}

\noindent
Los primeros tres puntos (vv4-5b) constituyen la parte \emph{imperativa} del mandamiento. Revis\'emosla tomada como un \emph{todo} para comprender \emph{qu\'e} es lo que Dios est\'a mandando.

\subsubsection{Parte \emph{imperativa}}

Para analizar la parte \emph{imperativa}, nos ser\'a de utilidad el siguiente ejemplo tomado de la vida cotidiana: \emph{?`es malo caminar?} Respuesta: tomado en \emph{s\'i mismo} no es malo, es indiferente e incluso puede ser bueno, ya que proporciona salud al ejercitarnos. \emph{?`es malo visitar a una anciana y a su hermana?} Respuesta: visitar a los ancianos es una cosa muy buena, pues alivias su soledad. \emph{?`es malo asesinar a sangre fr\'{i}a a una anciana y a su hermana indefensas?}\footnote{El ejemplo lo tomo de la novela de Fiódor Dostoyevski, \emph{Crimen y Castigo}; en el contexto de la novela el homicida \emph{no} se est\'a defendiendo, asesina a sangre fr\'{i}a a dos ancianas inocentes, por lo que queda exclu\'{i}da toda posibilidad de \emph{leg\'{i}tima defensa}.} Respuesta es muy malo y la sangre de esas dos ancianas \emph{clama al cielo} (\emph{c.f.} Gen 4,10).

Y la pregunta es: \emph{?`cu\'al es el juicio moral de las acciones <<caminar>> y <<visitar a una aiciana y a su hermana>> cuando estas acciones van encaminadas a <<asesinar a la anciana y a su hermana>>?} Creo que estar\'as de acuerdo conmigo en que en tal caso, estas acciones son tan malas como el fin que se proponen pues son sus instrumentos.

El ejemplo que te acabo de dar, aunque no es id\'entico, tiene algunas similitudes con Ex 20,4-5b si consideramos los vv4-5b como una unidad, es decir: cuando el acto \emph{material} (hacer im\'agenes) y el acto \emph{corporal} (postrarse ante ellas) van encaminados al acto \emph{espiritual}: (servir a las im\'agenes) es cuando se da el \emph{pecado}. Pues para este caso, el acto \emph{espiritual} (v5b) es el que \emph{califica moralmente a los otros actos}.

Vuelve a leer el p\'arrafo anterior porque tiene consecuencias muy importantes.

\'Esta es la raz\'on por la que puedes con toda tranquilidad tener fotos de tu familia o una enciclopedia ilustrada. En estos casos, las im\'agenes no son utilizadas con el prop\'osito de servirlas.

Por otro lado, \'esta es la raz\'on por la que Josu\'e y los ancianos de Israel no cometieron pecado alguno al postrarse \emph{delante} del Arca (Jos 7,6).

\'Esta es la raz\'on por la cual Ezequ\'{i}as hizo lo que es recto a los ojos de Yahveh al destruir la serpiente de bronce que por indicaci\'on de Dios hab\'{i}a hecho Mois\'es (2 Re 18,3-4).

\'Esta es la raz\'on que t\'u intu\'{i}as cuando dec\'{i}as que el mandamiento s\'olo aplica cuando se hacen im\'agenes \emph{con prop\'osito religioso}. Sin embargo te falta precisi\'on, necesitas profundizar m\'as para identificar cu\'ales prop\'ositos religiosos est\'an prohibidos por el mandamiento y cu\'ales no.

Hasta aqu\'{i} las similitudes con el ejemplo de la vida cotidiana. En el caso del v4 es claro que se trata de una acci\'on cuya calificaci\'on moral proviene del objeto que se propone. Pero es importante revisar con m\'as detenimiento el v5a-b para aprehender su significado y extraer las consecuencias.

\paragraph{Significado de \emph{postrarse}}

En Ex 20,5a NC dice: <<no te postrar\'as ante ellas>>. ?`qu\'e es postrarse?

Postrarse tiene muchos significados y existen muchos vocablos tanto en hebreo como en griego, pero en los casos en que se habla de adoracion podemos entenderlo gr\'aficamente de la siguiente manera: el que se postra asume una postura corporal que significa abajamiento, reconocimiento de la superioridad del otro, sumisi\'on.

Sumisi\'on que puede ser \emph{absoluta} (adoraci\'on) como en: Ex 34,8 o Is 2,20; 44,15.17 o \emph{relativa} (sin adoraci\'on), como en: Gen 18,2; 37,7-10; 1 Sam 24,8 o Rut 2,10.%
    \footnote{El vocablo hebreo \emph{shajah/shahah} (inclinarse, doblegarse, hacer reverencia, postrarse) sirve tanto para indicar el homenaje debido a personas importantes (e.g. los reyes: Samuel, David, Salom\'on) como para expresar la adoraci\'on a Dios. El hecho de que un mismo vocablo se utilice para dos cosas diferentes nos recuerda la importancia de atender al esp\'{i}ritu de la Escritura y no quedarnos \'unicamente con la letra (\emph{cf}. 2 Cor 3,6). Adem\'as de los ejemplos arriba expuestos puedes revisar los siguientes:\\ Sin \emph{adoraci\'on}: Gen 43,28; 2 Sam 1,2; 9,6; 14,4. Con \emph{adoraci\'on}: Gen 22,5; 1 Sam 15,25; Jer 7,2.\\ Revisa, entre otros, el siguiente enlace: http://archive.org/stream/DiccionarioBiblicoVine/DiccionarioBiblicoVine\_djvu.txt}
%http://www.abarc.org/Courses/Spanish/Christian%20Doctrine/Volume%20II/Christian%20Doctrine%20Vol%202%20Spanish%20Chapter%2012.pdf
% \begin{enumerate}
% \item Gen 19,1
% \item 1 Cro 21,16
% \item Gen 17,17
% \item Gen 37,10; cf. 35;16-19
% \item Gen 42,6
% \item 1 Sam 2,36
% \item 2 Sam 14,4
% %shajah - postracion sin adoracion
% \item Gen 18,2
% \item 1 Sam 24,8 RVR (9 NC)
% \item Rut 2,10
% \item Gen 37,7-10
% %shahah - con adoracion
% \item 1 Sam 15,25 y Jer 7,2
% \item Ex 34,8
% \item Is 2,20; 44,15.17
% %gonupeteo
% \item Mt 17,14; Mc 1,40
% \item Mc 10,17
% \item Mt 27,29
% %proskuneo
% \item Mc 5,6 <-- no se si adorando o no
% \item Mt 8,2; 9,18; 15,25; 20,20; Ap 3,9 ?? <-- adorar
% %prospipto
% \item Mc 3,11; 5,33; 7,25; Lc 8,28.47; Hch 16,29
% %titemigonata
% \item Lc 22,41; Hch 7,60; 9,40; 20,36; 21,5
% %balo
% \item Mt 8,6
% %katastronnumi
% \item 1 Cor 10,5
% %katafero
% \item Hch 20,9
% \end{enumerate}

%http://archive.org/stream/DiccionarioBiblicoVine/DiccionarioBiblicoVine_djvu.txt

\paragraph{Servir a las im\'agenes}

Con el \emph{servicio} pasa algo similar a lo que pasa con la \emph{postraci\'on} tenemos dos casos: de forma \emph{absoluta}, como obediencia y sumisi\'on total (adoraci\'on) como en  Dt 6,13; Mt 4,10; o bien, de forma \emph{relativa}, como en: Mc 1,31; 9,35; Lc 22,27 o como lo que hizo Jes\'us con sus acciones en el lavatorio de los pies (Jn 13).

\paragraph{Reuniendo las piezas de la \emph{parte imperativa} (Ex 20,4-5b)}

Reuniendo todo lo analizado hasta aqu\'{i} tenemos que la \emph{parte imperativa} significa lo siguiente:

\begin{quote}
Dios pide que s\'olo a \'El se de la obediencia \emph{absoluta} y que s\'olo a \'El se le otorgue el reconocimiento de la superioridad \emph{absoluta}; atribu\'{i}rselo a cualquier otro ser es abominaci\'on, por ejemplo, a las figuras que se fabrican con las manos. 
\end{quote}

\noindent
La composici\'on interna del texto nos dice claramente que la prohibici\'on de Dios es precisamente la \emph{adoraci\'on} a otros dioses, es decir, \textbf{Ex 20,4-5b prohibe tener otros dioses fuera de Yahveh, igual que Ex 20,3}.

Por eso el que se fabrica im\'agenes para inclinarse ante ellas (atribuy\'endoles absoluta superioridad) y las sirve (con sumisi\'on absoluta) comete \emph{idolatr\'{i}a}, es decir, \emph{<<aplica a cualquier cosa en lugar de a Dios, la indestructible noci\'on de Dios>>}.

Esta afirmaci\'on se ve corroborada por la \emph{parte no imperativa} del texto (Ex 20,5c-6) y por Ex 20,3. Examinemos ambas partes por separado.

\subsubsection{Incorporando Ex 20,5c-6 al an\'alisis}

La parte \emph{no imperativa} del texto viene a confirmar lo que venimos analizando: que lo mandado en Ex 20,4-5b es \emph{la prohibici\'on de adorar a otros dioses}.

En el apartado anterior lo dedujimos de la \emph{estructura interna} de Ex 20,4-5b; ahora lo deduciremos del \emph{contexto inmediato} (externo), es decir, de Ex 20,5c-6 y Ex 20,3.

\paragraph{Los celos de Dios}

El Ex 20,5c NC dice: <<porque yo soy Yahveh, un Dios \emph{celoso}>>, la parte \emph{no imperativa} comienza explicando la raz\'on de la \emph{parte imperativa}. Y el motivo es que Yahveh es un Dios \emph{celoso}.

Los \emph{celos} en la Escritura tienen un significado muy profundo. Dios se \emph{ha desposado} con su pueblo (Is 54; 62,1-5; Ez 16,1-14; Ef 5,25-27) y tener otros dioses es como ser infiel al esposo (Jer 3; Ez 16,15-58). El tema central del libro de Oseas es este drama esponsal.

As\'{i} que ir en pos de otros dioses es para el creyente, lo mismo que para la esposa ir en pos de otros hombres.%
% Es como si Dios dijera: \emph{no tengas fotograf\'{i}as de otros pretendientes}. Esta analog\'{i}a de las fotograf\'{i}as nos servir\'a m\'as adelante, no la pierdas de vista.

Por tanto, Ex 20,5c \emph{refuerza} la afirmaci\'on de que \textbf{Ex 20,4-5b forma parte de un \'unico mandamiento con Ex 20,3}.

Antes de analizar Ex 20,3 digamos alguna palabra sobre Ex 20,5d-6.

\paragraph{Las consecuencias de la desobediencia y la obediencia}

En Ex 20,5d-6 NC leemos: \emph{<<que castiga en los hijos las iniquidades de los padres hasta la tercera y cuarta generaci\'on de los que me odian y hago misericordia hasta mil generaciones de los que me aman y guardan mis mandamientos>>}.

Rescato de aqu\'{i} dos palabras, que reflejan dos actitudes contrapuestas: \emph{odiar} y \emph{amar}. Abandonar a Dios e ir en pos de otros dioses equivale a \emph{odiar} a Dios. Serle fiel como una esposa, \emph{amarlo}, se ve reflejado en \emph{guardar sus mandamientos}.

Este texto tiene la estructura de un \emph{pacto} entre Dios y su pueblo: Dios pide a Israel la \emph{fidelidad} que un esposo pide a la esposa. Si el hombre es fiel a los mandamientos de Dios, recibir\'a \emph{misericordia} por mil generaciones. Si el hombre prostituye su fe y se va en pos de otros dioses, ser\'a \emph{castigado} hasta la tercera y cuarta generaci\'on.

Esta idea del \emph{pacto} ser\'a importante para lo que viene a continuaci\'on.

\subsubsection{Incorporando Ex 20,3 al an\'alisis}

No \'unicamente Ex 20,5c-6 refuerza la idea de que Ex 20,4-5b se refiere a la adoraci\'on de otros dioses. Tambi\'en Ex 20,3 lo sugiere. El texto sigue una estructura circular en la que:

\begin{enumerate}
\item Dios recuerda al pueblo qui\'en los a rescatado de la esclavitud (Ex 20,2)
\item Pide que no tengan otros dioses (Ex 20,3)
\item lo ilustra de manera gr\'afica (Ex 20,4-5b)
\item indica el \emph{motivo} (Ex 20,5c), el cual confirma lo dicho en el v3
\item cierra con las consecuencias (Ex 20,5d-6)
\end{enumerate}

\noindent
Si t\'u lo separas, rompes su estructura interna: la de un \emph{pacto}\footnote{Una estructura similar aunque m\'as sencilla se da en Ex 15,23-26 y en Ex 19,1-6.} en la que Dios escoge a Israel como propiedad suya y le pide a cambio fidelidad esponsal advirti\'endole las consecuencias de apartarse de su camino. Dentro del mismo pacto aparece el resto de los mandamientos. Se sigue una estructura similar en el v7, otra en los vv8-11 y otra en el v12. Aunque \'estas no re\'unen todos los elementos. El v2 sirve simult\'aneamente como apertura para Ex 20,2-6 y para Ex 20,2-17.

Por tanto, la estructura gramatical y sem\'antica de Ex 20,2-17 tambi\'en refuerza la idea de que \textbf{el primer mandamiento va desde Ex 20,2 hasta Ex 20,6} y que lo que proh\'{i}be es tener \emph{otros dioses}; aqu\'{i} el papel de Ex 20,4-5b es \emph{mostrar el tipo de pr\'acticas que hacen aquellos que adoran otros dioses} m\'as que prohibir hacer im\'agenes, incluso con \emph{prop\'osito religoso}.

\subsection{?`La Iglesia Cat\'olica \emph{escondi\'o} el segundo mandamiento?}

Llegado es el tiempo de resolver la cuesti\'on de si la Iglesia Cat\'olica, en su af\'an de promover la idolatr\'{i}a \emph{escondi\'o} el segundo mandamiento que proh\'{i}be las im\'agenes o bien si los protestantes \emph{dividieron} el primer mandamiento para fundamentar su separaci\'on de Roma.

El Catecismo de la Iglesia Cat\'olica lo resume en el n\'umero 2066 de la siguiente manera:

\begin{quote}
\emph{La división y numeración de los mandamientos ha variado en el curso de la historia}. El presente catecismo sigue la división de los mandamientos establecida por san Agustín y que ha llegado a ser tradicional en la Iglesia católica. Es también la de las confesiones luteranas. Los Padres griegos hicieron una división algo distinta que se usa en las Iglesias ortodoxas y las comunidades reformadas.\footnote{http://www.vatican.va/archive/catechism\_sp/p3s2\_sp.html}
\end{quote}

\noindent
El an\'alisis comienza con la pregunta \emph{<<Seg\'un la Biblia ?`Cu\'antos son los mandamientos del dec\'alogo?>>}; posteriormente revisaremos tres formas de numeraci\'on que han existido y sus principales caracter\'{i}sticas; entonces podremos responder a la pregunta acerca de si la Iglesia Cat\'olica \emph{escondi\'o} el segundo mandamiento o bien los protestantes \emph{dividieron} el primero.

\subsubsection{?`Cu\'antos son los mandamientos?}

La palabra \emph{dec\'alogo} proviene del griego \emph{$o \iota$ $ \delta\varepsilon\kappa\alpha$ $\lambda o \gamma o \iota$} el cual es utilizado en la versi\'on de los LXX para traducir \emph{{\lq}aseret haddebarim} en Dt 10,4 y significa <<diez palabras>>.\footnote{http://www.mercaba.org/DicTB/D/decalogo.htm}

Seg\'un Dt 5,22 fueron escritos en dos tablas de piedra. Pero hay dos cosas que la Biblia no precisa: a) de qu\'e forma se \emph{distribuyeron} las palabras en las tablas y b) la numeraci\'on exacta de las palabras.

Algunos discuten si las diez palabras se distribuyeron entre las dos tablas o bien si cada una de las tablas conten\'{i}a todo el dec\'alogo repetido como si fuera un contrato con dos copias: una para Dios y otra para el pueblo.\footnote{Estos temas son bastante extensos y muchos aspectos son controvertibles; as\'{i} que me centrar\'e en lo esencial para nuestra discusi\'on.}

\subsubsection{Distintas maneras de hacer la numeraci\'on de los mandamientos}

Sin adentrarme en esas cuestiones, voy a se\~nalar cuatro tradiciones interpretativas antiguas que corresponden con tres maneras de hacer la numeraci\'on:\footnote{En esta liga puedes encontrar un cuadro comparativo: http://es.wikipedia.org/wiki/Diez\_Mandamientos; sin embargo, el cuadro requiere algunas precisiones de las tradiciones que lista: a) Donde dice \emph{septuaginta} deber\'{i}a decir \emph{Or\'{i}genes de Alejandr\'{i}a}; y b) Donde dice \emph{Catecismo de la Iglesia Cat\'olica} es redundante, pues el mismo catecismo indica que est\'a siguiendo a San Agust\'{i}n.}

\begin{enumerate}
\item El Talmud\footnote{http://www.tora.org.ar/contenido.asp?idcontenido=856}
\item Fil\'on de Alejandr\'{i}a\footnote{http://dadun.unav.edu/bitstream/10171/13235/1/ST\_XXIX-2\_03.pdf}
\item Or\'{i}genes de Alejandr\'{i}a\footnote{http://dadun.unav.edu/bitstream/10171/13256/1/ST\_XXX-1\_03.pdf}
\item San Agust\'{i}n de Hipona\footnote{http://www.augustinus.it/spagnolo/ y http://dialnet.unirioja.es/descarga/articulo/233589.pdf}
\end{enumerate}

\noindent
Te recomiendo ampliamente que revises las fuentes que estoy dejando en los pies de p\'agina, en especial las que se refieren a Fil\'on, Or\'{i}genes y Agust\'{i}n. Valen mucho la pena m\'as all\'a del debate que estamos sosteniendo nosotros.

\paragraph{El Talmud}

afirma que el primer mandamiento es Ex 20,2 y el segundo mandamiento es Ex 20,3-6, es decir que para los judíos \textbf{es un único mandamiento no tener otros dioses y no hacerse figuras y adorarlas}, justo lo que venimos diciendo; mientras que el d\'ecimo mandamiento es Ex 20,17 colocando a la mujer entre las \emph{posesiones} del pr\'ojimo. Esta tradici\'on es importante, porque muy probablemente fue la que estaba vigente en tiempo de Jes\'us en Jerusal\'en.

\paragraph{Fil\'on de Alejandr\'{i}a  (13 a.C.-50 d.C.)}

se\~nala que el primer mandamiento es Ex 20,3; el segundo es Ex 20,4-6 y el d\'ecimo es Ex 20,17. La obra de Fil\'on intenta armonizar su fe jud\'{i}a con el pensamiento griego. Armonizar la revelaci\'on de Dios con el conocimiento filos\'ofico. Sin embargo se mantiene dentro de los horizontes del Antiguo Testamento.

Fil\'on de Alejandr\'{i}a es importante por dos razones: a) conform\'o la tradici\'on de los jud\'{i}os de habla griega, distanci\'andose del juda\'{i}smo tradicional y b) Tendr\'a una influencia \emph{decisiva} en Or\'{i}genes.

\paragraph{Or\'{i}genes de Alejandr\'{i}a (185-254 d.C.)}

recibi\'o una influencia considerable de Fil\'{o}n de Alejandr\'{i}a tanto en el esfuerzo de armonizar el pensamiento filos\'{o}fico con el religioso; como en la \emph{numeraci\'on} de los mandamientos.

Sin embargo, entre Fil\'on y Or\'{i}genes hay \emph{dos siglos} de cristianismo. El \emph{contenido} de los comentarios de Or\'{i}genes al dec\'alogo tiene toda la impronta cristiana.

Or\'{i}genes sigue al pie de la letra a Fil\'on en la manera de dividir los mandamientos; sin embargo, como dato importante destaca la \emph{justificaci\'on} que da Or\'{i}genes a dividir en dos mandamientos lo escrito en Ex 20,3 y Ex 20,4-6:

\begin{quote}
<<La razón que da Orígenes a este desdoblamiento es que aunque
algunos piensan que todas estas cosas forman un sólo mandamiento, pero
entonces no se llegaría al número diez y entonces ¿dónde estaría la verdad
del término decálogo? Por tanto, hay que desglosarlo en dos mandamientos
diferentes y explicó cada uno de ellos>>.\\
(LLUCH Baixauli, Miguel; \emph{La Interpretación de Orígenes al Decálogo})\footnote{http://dadun.unav.edu/bitstream/10171/13256/1/ST\_XXX-1\_03.pdf}
\end{quote}

\noindent
Llama la atenci\'on esta \emph{justificaci\'on} porque \emph{no alude ning\'un motivo teol\'ogico} sino m\'as bien el que sean \emph{diez} los mandamientos. Este dato nos revela dos cosas: a) que Or\'{i}genes no ve objeci\'on \emph{teol\'ogica} a que Ex 20,3 forme \emph{un} mandamiento con Ex 20,4-6 y b) que en el tema de Ex 20,17 se mantiene en la misma l\'{i}nea que el juda\'{i}smo.

La influencia de Or\'{i}genes en el pensamiento cristiano es enorme. Pero para el tema que nos ocupa diremos que es importante porque \emph{las iglesias de Oriente} siguen a Or\'{i}genes en la numeraci\'on de los mandamientos.

\emph{A\'un as\'{i}, las iglesias de Oriente no interpretan Ex 20,4-6 en el sentido que lo hacen los protestantes; oriente es la cuna de los \'{i}conos cristianos y ellos no tuvieron ning\'un conflicto con eso}. 

\emph{Siglos m\'as tarde, fue en la ciudad griega (oriental) de Nicea donde en el a\~no 787 en concilio ecum\'enico se proclam\'o la legitimidad de las im\'agenes religiosas as\'{i} como la manera correcta de usarlas; todo esto frente a la herej\'{i}a iconoclasta, la cual surgi\'o, entre otras razones, por influencia del Islam}.

\subsubsection{San Agust\'{i}n de Hipona (354-430 d.C.)}

Es el m\'as grande pensador cristiano de la antig\"uedad. Para el tema que nos ocupa cito a Miguel Lluch que dice:

\begin{quote}
Será San Agustín el que unirá los dos primeros origenianos
y distinguirá los dos últimos. Para San Agustín es distinto desear
la mujer del prójimo que el resto de los bienes del prójimo y así, entenderá
dos prohibiciones distintas en el último mandamiento y reunirá en uno
sólo el primero. Esta enumeración agustiniana se convertirá en la propia
de la tradición católica. En el siglo XVI Lutero la mantuvo también, \emph{pero
Calvino se separó de ella y volvió a la tradición flloniana}.\footnote{\emph{\'{I}dem.}, el subrayado es m\'{i}o.}
\end{quote}

\noindent
Antes de responder a la pregunta \emph{<<?`es leg\'{i}tima la divisi\'on que hace San Agust\'{i}n de Ex 20,17?>>} quiero resaltar un hecho: los \emph{motivos} que da el santo de Hipona para hacerlo, \emph{no tienen nada que ver} con \emph{esconder} un mandamiento para practicar la \emph{idolatr\'{i}a}.

\paragraph{?`Es leg\'{i}tima la divisi\'on que hace San Agust\'{i}n de Ex 20,17?}

Para comenzar, San Agust\'{i}n est\'a siguiendo a Dt 5,21 donde \emph{s\'{i}} est\'an separados, la \emph{mujer} y los \emph{bienes} del pr\'ojimo.

Los motivos que da el santo de Hipona para hacer as\'{i} la divisi\'on de Dt 5,21 en dos (y unir Ex 20,3-6 en uno solo) son de \'{i}ndole teol\'ogica, entre ellos, podemos se\~nalar dos:\footnote{Revisa http://www.augustinus.it/spagnolo/discorsi/discorso\_009\_testo.htm y \\ http://www.augustinus.it/spagnolo/discorsi/discorso\_010\_testo.htm}

\begin{description}
\item[Para unir Ex 20,3-6] \emph{La confesi\'on de la fe trinitaria} manifestada en tres mandamientos: Ex 20,3-6; Ex 20,7 y Ex 20,8-11. Pues cada uno de \'estos mandamientos nos revela algo sobre cada una de las tres divinas personas.
\item[Para dividir Dt 5,21] El deseo de la mujer es de naturaleza \emph{enteramente distinta} al deseo de los bienes; y as\'i como al mandamiento que proh\'{i}be el robo (Dt 5,19), le corresponde un mandamiento que proh\'{i}be desear lo ageno (Dt 5, 21b), de la misma forma al mandamiento que proh\'{i}be el adulterio (Dt 5,18) le corresonde aquel que proh\'{i}be desear la mujer del pr\'ojimo (Dt 5,18).
\end{description}

\noindent
Como puedes ver, la Iglesia Cat\'olica \emph{no escondi\'o} ning\'un mandamiento para justificar la idolatr\'{i}a; las cosas siguieron otro curso. Preg\'untate ahora si no fue Juan Calvino quien, para justificar sus doctrinas, busc\'o apoyo en tradiciones no cristianas, como la de Fil\'on de Alejandr\'{i}a.

En cualquier caso, m\'as que seguir una numeraci\'on u otra (la Iglesia Cat\'olica no lo considera un dogma), lo importante es atender al \emph{contenido y significado}\ldots despu\'es de un breve par\'entesis profundizaremos un poco m\'as al respecto.

\subsubsection{La Iglesia Cat\'olica NO \emph{escondi\'o} el segundo mandamiento}
La controversia sobre la numeraci\'on de los mandamientos nos lleva a la conclusi\'on de que lo importante no es saber con certeza \emph{a qu\'e n\'umero de mandamiento} pertenece lo mandado en Ex 20,4-6; sino al hecho de que lo prohibido por el mandamiento es la adoraci\'on de \emph{otros dioses}.

Ex 20,4-6 es, en realidad, una \emph{ampliaci\'on} de Ex 20,3 y no se le puede entender separadamente, aunque se le numere por separado.

Por otra parte, puede ser que a estas alturas todav\'{i}a sientas demasiado forzado el argumento, pues claramente Dt 5,18 es \emph{un} solo vers\'{i}culo.

Sin embargo, como buen conocedor de la Biblia que eres, no te ser\'a desconocido, que \emph{la divisi\'on en cap\'{i}tulos y vers\'{i}culos no forma parte de la revelaci\'on}.

Esta inserci\'on fue hecha durante la Edad Media por los \emph{te\'ologos cat\'olicos} para facilitar el estudio de la Sagrada Escritura. As\'{i} que apoyarse en la numeraci\'on de cap\'{i}tulos y vers\'{i}culos para hacer fundamentaciones teol\'ogicas es err\'oneo.

\section{Legitimidad de la pr\'actica cat\'olica de tener im\'agenes}

Una vez visto que el argumento de excepci\'on no es v\'alido y que Ex 20,4-6 forma una unidad con Ex 20,3 donde el \emph{acento} est\'a en no \emph{adorar} otros dioses (de lo cual, fabricar/tener im\'agenes e inclinarse ante ellas son expresiones externas) analizaremos ahora si la pr\'actica cat\'olica en s\'{i} misma es idol\'atrica o no.

Primero resumiremos lo discutido hasta ahora. Posteriormente plantearemos el problema en toda su crudeza. Despu\'es resolveremos el problema apoy\'andonos tanto en la profundizaci\'on de los textos b\'{i}blicos como en comparaciones que nos ayuden a comprender el problema. Por \'ultimo analizaremos el argumento central del segundo concilio de Nicea (s.VIII) que fue el que defini\'o dogm\'aticamente la legitimidad del uso de im\'agenes, as\'{i} como sus limitaciones.

\subsection{La prohibici\'on de Ex 20,4-6 no es absoluta, sino relativa}

La prohibici\'on de las im\'agenes en Ex 20,4-6 no es absoluta, esta ligada al \emph{uso} de dichas im\'agenes el cual est\'a prohibido cuando se trata de adorar a otros dioses, como se indica en Ex 20,3.

Que \'esta sea la interpretaci\'on correcta se deduce de lo que hemos discutido anteriormente. As\'{i} lo han cre\'{i}do los cristianos de todos los tiempos. \'Unicamente los herejes iconoclastas de los siglos VII-VIII y los protestantes en el siglo XVI en adelante, basados en interpretaciones privadas de la escritura (2 Pe 1, 20) han considerado las im\'agenes de Jes\'us y los bienaventurados como idol\'atricas.

Dado que a) el \emph{argumento de excepci\'on} carece de validez b\'{i}blica y b) el mandamiento de Ex 20,3-6 prohibe hacer im\'agenes \emph{\'unicamente} cuando es para la adoraci\'on de otros dioses y no una prohibici\'on \emph{absoluta} de hacer \emph{cualquier} tipo de im\'agenes como pretende el protestantismo iconoclasta, podemos hacer la siguiente afirmaci\'on:

\begin{quote}
\emph{Existen usos leg\'itimos de im\'agenes con prop\'osito religioso. Los ejemplos del Arca, el Templo y la serpiente de bronce constituyen gu\'{i}as de c\'omo hacerlo.}
\end{quote}

\noindent
Ya has dicho \emph{los \'unicos casos v\'alidos que encontramos son cuando Dios expl\'{i}citamente ordena la elaboraci\'on de las im\'agenes}, sin embargo tal afirmaci\'on es falsa, pues encontramos en la escritura acciones no ordenadas por Dios y no reprobadas y por otro lado encontramos acciones reprobadas sobre objetos que \'El mand\'o construir.

\begin{itemize}
\item \emph{Acciones no solicitadas sobre objetos solicitados y que son aprobadas} as\'{i} por ejemplo cuando se postran delante del Arca (Jos 7,6)

\item \emph{Acciones no solicitadas sobre objetos no solicitados y que son aprobadas} por ejemplo cuando Eliseo recibi\'o el manto de El\'{i}as y con \'el separ\'o las aguas del r\'{i}o (2 Re 2,9-14) o cuando Naam\'an pide llevarse tierra de Israel en se\~nal de que s\'olo adorar\'a a Yahveh (2 Re 5,17); pero tambi\'en en el N.T. encontramos algunos ejemplos: una mujer es curada al tocar el manto de Jes\'us (Mc 5,27-29); pon\'{i}an a los enfermos a que la sombra de Pedro pasara sobre ellos y as\'{i} se curaran (Hch 5,15-16)\footnote{Respecto a lo que dec\'{i}as de que la Biblia dice que los colocaban mas no que se curaran, estas considerando que la segunda parte del v16 unicamente aplica a la primera parte y no al v15, es una asociaci\'on arbitraria pues rompes el sentido del pasaje; por otra parte, como ocurre en otras ocasiones, Pedro pudo muy bien aclararles que lo que hac\'{i}an era incorrecto pero no lo hizo; revisa: Hch 14,8-18.}; o los pa\~nos y delantales de Pablo que curaban enfermos (Hch 19,11-12).\footnote{Estos ejemplos son figuras y anticipos de la pr\'actica cat\'olica de la veneraci\'on de las reliquias, que en algunos casos Dios obra milagros por medio de ellas y en otros no, revisa por ejemplo: 2 Re 13,21.}

\item \emph{Acciones no solicitadas sobre objetos s\'{i} solicitados pero que son condenadas} Como cuando el pueblo ador\'o la serpiente de bronce de Mois\'es y por eso, Ezequ\'{i}as (sin mandato expl\'{i}cito de Dios) la destruy\'o%
    \footnote{T\'u puedes decirme: \emph{ah\'{i} no dice que la adoraran sino que le <<quemaron incienso>>}. A lo cual yo te respondo: 1) el pasaje muestra que la serpiente de bronce era el objeto destinatario del incienso, y el incienso en el A.T. es ofrecido \'unicamente a Dios como signo de adoraci\'on (revisa Lev 1-7); y 2) suponiendo que "1)" no sea cierto, \emph{ni} en Ex 20,3-6 \emph{ni} en Dt 5,5-10 se menciona la palabra \emph{incienso}, as\'{i} que seg\'un tu criterio, <<quemar incienso>> a una imagen no est\'a prohibido, entonces ?`con qu\'e autoridad Ezequ\'{i}as destruyo un objeto mandado hacer por Dios?}
\end{itemize}

\noindent
El criterio no es que \emph{Dios haya ordenado hacer esos objetos} (argumento de excepci\'on) sino el \emph{uso} que se les da: si son instrumentos para la adoraci\'on a Dios o por el contrario son adorados en lugar del Creador.

Pudi\'eramos pensar, a partir de lo visto, que hipot\'eticamente existiera alguna forma de tener im\'agenes con prop\'osito religioso y que sea conforme al mandato de Dios. Pero que \emph{definitivamente no es la pr\'actica cat\'olica, la cual se asemeja m\'as al paganismo que ninguna otra y a\'un es peor, al pretender justificarse torciendo la Escritura}; por ejemplo: el hecho de que se le ofrezca incienso a las im\'agenes cat\'olicas, lo cual es una se\~nal inequ\'ivoca de adoraci\'on como yo mismo acabo de decir.
\subsection{?`No es evidente con solo mirarlos que los cat\'olicos son id\'olatras digan lo que digan?}

Pudieras hacerme algunas objeciones como las siguientes:

\begin{itemize}
\item
\emph{En la pr\'actica, adorar y venerar im\'agenes es lo mismo: la gente conf\'{i}a en la imagen que le haga el 'milagrito'}.

\item
\emph{Es obvio que los cat\'olicos son id\'olatras, toma por ejemplo, Daniel 3, y reemplaza la estatua de Nabucodonosor con un crucifijo o una estatua de la Virgen de Zapopan y !`zaz! la escena es igualita, mismo comportamiento, todo igual, deber\'{i}a decir peor, ya que los cat\'olicos ni siquiera necesitan que se les amenace con un horno ardiente\ldots}

\item
\emph{Puedes por otra parte considerar Ezequiel 8 y compararlo con cualquier catedral o iglesia colonial y ver\'as que todo es id\'entico. Im\'agenes por todos lados, la gente postr\'andose, adorando, cosas que Yahv\'e abomina.}

\item
\emph{Mira a la gente c\'omo pone im\'agenes en su casa, en su auto, en su trabajo, esperando que la proteja del mal, mientras son ladrones, ad\'ulteros, borrachos\ldots esperan que esos objetos los libren del juicio de Dios pero s\'olo est\'an colmando el vaso de su ira.}

\item
\emph{Por lo tanto los cat\'olicos tuercen las escrituras para su perdici\'on, mediante argucias tratan de enturbiar lo que Dios revel\'o con total claridad, y t\'u mismo ahora porque no quieres admitir lo que es obvio, s\'olo te enga\~nas a t\'{i} mismo.}
\end{itemize}

\noindent
Antes de entrar en detalles voy a hacer algunas puntualizaciones:

\begin{itemize}
\item Los cat\'olicos no adoramos im\'agenes, cualquiera con mediana frormaci\'on te lo puede decir. Si fuiste catequista, deber\'{i}as saber la enorme diferencia entre una imagen de Jes\'us y la Hostia consagrada. A la imagen se le venera con la intenci\'on de adorar a Jes\'us, a la hostia se le adora porque creemos que ah\'{i} est\'a Jes\'us f\'{i}sicamente presente.\footnote{La discusi\'on sobre la Eucarist\'{i}a no es tema de hoy. Por otro lado, la devoci\'on llamada <<adoraci\'on de la santa cruz>> est\'a mal nombrada; cualquier cat\'olico medianamente formado puede explic\'artelo: se trata de adorar a Jes\'us recordando el dolor que sufri\'o en la cruz, y venerar este instrumento que \'El eligi\'o para nuestra salvaci\'on.} Si no sab\'{i}as esto, pide que te devuelvan tu dinero, te estafaron.

\item Las personas que usan im\'agenes para 'tapar' su mala conducta no est\'an siendo congruentes con su fe. Sea por ignorancia o por necedad, eso no forma parte de la doctrina cat\'olica y los que as\'{i} obran no act\'uan como cat\'olicos en ese punto.

\item Argumentar que lo que hacen los cat\'olicos es igualito a lo que hac\'{i}an los cananeos es falaz. No es lo mismo adorar a Nabucodonosor que a Jes\'us. Considera si en Mateo 2,11; pero en lugar del <<ni\~no>> pon a Herodes ?`Es lo mismo?

\item Si quieres creer a toda costa que somos id\'olatras porque as\'{i} te conviene para justificar lo que de otro modo ser\'{i}a apostas\'{i}a no es mi problema. Pero eso tiene otro nombre: \emph{prejuicio}.

\item No te f\'{i}es de lo que ves. Debes tener fe y no juzgar las apariencias. En la cruz ves a un simple hombre. Necesitas tener fe para reconocer que aquel que cuelga de un madero es Dios-con-nosotros. De igual manera, necesitas ver mas all\'a de la apariencia para notar que no se trata de adoraci\'on a una imagen sino que la intenci\'on remite al original.\footnote{M\'as adelante profundizo en este punto.}
\end{itemize}

\noindent
En lo que sigue voy a mostrarte c\'omo la pr\'actica cat\'olica de la veneraci\'on de las im\'agenes es mucho m\'as de lo que ves y es perfectamente conforme con la voluntad de Dios.

\subsection{Una comparaci\'on \'util}

Una de las figuras recurrentes m\'as hermosas y profundas acerca de la relaci\'on de Dios con su pueblo, es la de la esposa. Yahv\'e ha tomado a Israel por esposa (Is 54; 61,10-62,5; Ez 16,1-14).

Pero como bien sabes, la Antigua Alianza es figura de lo que habr\'{i}a de venir (Heb 10,1) sabemos que la verdadera uni\'on esponsal es la boda del Cordero (Ap 21,9), la uni\'on de Cristo y su Iglesia (Ef 5,25-33).

A partir de esta figura, voy a utilizar una comparaci\'on que nos va a ayudar a entender lo que estamos analizando:

\begin{quote}
\emph{Una mujer estaba comprometida con un hombre al que no conoc\'{i}a f\'{i}sicamente, lo amaba y quer\'{i}a serle fiel; el novio le enviaba de vez en cuando cartas y mensajes a trav\'es de amigos y parientes; incluso en un par de ocasiones le encarg\'o que hiciera algunos objetos para recordarlo.}

\emph{Las chicas rom\'anticas gustan tener fotos del joven que ha conquistado su coraz\'on. Guardan en un lugar especial la foto del amado; la contemplan, la presumen, la besan, la llevan consigo a todas partes\ldots ?`Y esta joven? ?`Que pasar\'{i}a si guardara fotos de diferentes hombres (quiz\'a sacadas de alguna revista de adolescentes) imaginando que alguno de ellos es su prometido?}

\emph{Si as\'{i} hiciera, correr\'{i}a el peligro de que cuando llegue el prometido ya no quiera estar con \'el. Tambi\'en si insiste en el tema, podr\'{i}a relajar su moral y abandonar su compromiso para ir a buscar satisfacciones inmediatas. El peligro se vuelve mayor si el prometido tarda en llegar.}

\emph{En atenci\'on a este peligro, el novio le escribe una carta: <<{\ldots}puesto que no has visto mi rostro, no te hagas de im\'agenes ni rostros de var\'on alguno, sean fotograf\'{i}as, retratos o dibujos, conf\'ormate con escuchar mis palabras hasta que est\'e contigo, conf\'{i}a en mi, pronto estar\'e contigo\ldots>>}
\end{quote}

\noindent
Con esta historia como pre\'ambulo, vamos a analizar ahora el cap\'{i}tulo cuarto del libro del Deuteronomio.

\subsection{Analizando el cap\'{i}tulo cuarto del Deuteronomio}

El cap\'{i}tulo cuarto del Deuteronomio cierra el primer discurso de Mois\'es al pueblo de Israel (vv1-40),%
    \footnote{El libro del Deuteronomio tiene cuatro discursos: \emph{primero}: Dt 1,5-4,40; \emph{segundo}: Dt 4,44-26,18(19); \emph{tercero}: Dt 27,1-28,68; \emph{cuarto}: Dt 29. Adem\'as est\'an las disposiciones y c\'anticos finales: Dt 31-34.}
algunas disposiciones de Mois\'es (vv41-43) y el comienzo del segundo discurso a los israelitas (vv44-49).

La parte que a nosotros nos interesa son los vers\'{i}culos 1-40, los cuales pueden clasificarse de la siguiente manera de acuerdo a su contenido:

%TODO: decidir si itemize/enumerate/etc terminan con punto o sin él
\begin{itemize}
\item Exhortaciones generales a cumplir la ley: vv 1-2, 5-9 y 39-40.
\item Recuerdo de castigos pasados: vv 3 y 21-22.
\item Recuerdo de los prodigios de Dios en favor de su pueblo: vv 4, 10-14 y 32-38.
\item Prohibiciones contra la idolatr\'{i}a: vv 15-20 y 23-24.
\item Amenazas si Israel se vuelve a otros dioses y promesa de perd\'on si hay arrepentimiento: vv 25-31.
\end{itemize}

\noindent
\textbf{El segundo vers\'{i}culo reviste tal importancia que le dedicaremos una secci\'on aparte m\'as adelante.} Ahora estudiaremos los vers\'iculos 15-20 que son los m\'as importantes para nuestro estudio.

\subsubsection{An\'alisis de Dt 4,15-20}

La estructura de esta prohibici\'on es la siguiente:

\begin{itemize}
\item Comienza con una advertencia: <<Tened mucho cuidado de vosotros mismos>> v15a
\item En v15b recuerda Dt 4,12b y lo se\~nala como \emph{la raz\'on} por la que\ldots
\item Se\~nala el peligro de pervertirse haciendo im\'agenes/esculturas de \emph{cualquier representaci\'on que sea} (vv16-18)
\item En v19a completa diciendo que el \emph{efecto} de hacer tales representaciones es que el hombre se vea tentado a adorar tales cosas
\item En v19b indica que Dios ha dado la creaci\'on a los pueblos de la tierra.
\item El v20 cierra diciendo que Israel es propiedad del Se\~nor.
\end{itemize}

\noindent
Hacer im\'agenes conlleva el riesgo de adorarlas. La tentaci\'on es muy fuerte. M\'as para Israel, rodeado de pueblos numerosos, m\'as ricos y poderosos; todos ellos id\'olatras siendo el \'unico pueblo con un solo Dios y sin imagen alguna de \'El.

\textbf{Pero en v15b hay un \emph{algo} que \emph{delimita} todo el pasaje con una relaci\'on de \emph{causa-efecto}:}

\begin{quote}
<<Pues ninguna figura visteis el día que Jehová habló con vosotros de en medio del fuego>>\\
Dt 4,15b RVR
\end{quote}

\noindent
Por lo tanto, \emph{\textbf{LA PROHIBICI\'ON DE IM\'AGENES EN DT 4,15-20 EST\'A SUJETA AL HECHO QUE EL PUEBLO DE ISRAEL NO HA VISTO A DIOS}}.

Si hubiera visto alguna figura en el Horeb, la advertencia no tendr\'{i}a lugar. Como en el caso de la novia que no ha visto el rostro del novio y que por tanto no deber\'{i}a tener fotograf\'{i}as de hombre alguno para ser fiel a su prometido, igual Israel no debe tener im\'agenes para no corromperse.

\subsection{?`Qu\'e ha pasado de la Antigua a la Nueva Alianza?}
?`Qu\'e ha cambiado entre el monte Sina\'{i} y el monte Calvario?

%TODO: todos los cf deben ser \emph{cf}.
El misterio escodido desde los siglos en Dios, que no se dio a conocer en otras generaciones, ha sido revelado a los ap\'ostles y profetas, a los santos, a los principados y potestades, mediante la Iglesia y el Esp\'{i}ritu por la predicaci\'on del Evangelio (\emph{cf}. Ef 3,5.9.10; Col. 1,26).

?`Y cu\'al es este misterio? ?`Cu\'al es \emph{la sabidur\'{i}a de Dios misteriosa, escondida, destinada por Dios desde antes de los siglos para gloria nuestra, desconocida de todos los pr\'{i}ncipes de este mundo --pues de haberla conocido no habr\'{i}an crucificado al Se\~nor de la gloria} (\emph{cf}. 1 Cor 2,7-8)?

?`Qu\'e es <<lo que ni el ojo vio, ni el oído oyó, ni al corazón del hombre llegó, lo que Dios preparó para los que le aman>> (1 Cor 2,9 LBJ)?

\begin{quote}
El $\Lambda o \gamma o \varsigma$ eterno del Padre el cual tambi\'en es Dios (\emph{cf}. Jn 1,1) se ha hecho \emph{carne} por nosotros (\emph{cf}. Jn 1,14). El Hijo de Dios se ha hecho Hijo del Hombre. Las bodas del Cordero (Ap 19,6-9)%
    \footnote{El pasaje se refiere a un tiempo futuro, pero no excluye el tiempo presente (ver Mt 9,14-15 y Ef 5,22-32).}
en que Dios se ha desposado con la humanidad, en que Dios-Hijo se ha hecho con los hombres una sola carne (\emph{cf}. Gen 2,24; Mt 19,5; Ef 5,22-32) ha tenido lugar para nuestra salvaci\'on.

El Dios escondido, el Dios de Israel, el Salvador (\emph{cf}. Is 45,15); al que nadie ha visto jam\'as (\emph{cf}. Jn 1,18) ahora es visible, el Hijo lo ha dado a conocer (\emph{Ibid}), quien ve a Jes\'us, ve al Padre (\emph{cf}. Jn 14,9), el es imagen de Dios invisible (\emph{cf}. Col 1,15); lo vimos, o\'{i}mos y lo tocaron nuestras manos (\emph{cf}. 1 Jn 1,1); si confesamos que Jesucristo ha venido en la carne somos de Dios (\emph{cf}. 1 Jn 4,2).
\end{quote}

\noindent
\textbf{Por lo tanto queda sin efecto la cl\'ausila de Dt 4,15b.} Ya es posible tener im\'agenes con prop\'osito religioso y usarlas para la adoraci\'on a Jesucristo; cuando son rectamente utilizadas no hay temor a caer en idolatr\'{i}a.\footnote{M\'as adelante detallo eso de \emph{rectamente utilizadas}} %TODO: imagenes rectamente utilizadas

\subsection{Objeciones a la transitoriedad de Dt 4,15b}

Frente a este argumento, pueden esgrimirse en contra los siguientes argumentos:

\begin{itemize}
\item Ni una jota ni una tilde pasará de la ley, hasta que todo se haya cumplido (\emph{cf}. Mt 5,18; Lc 16,17)
\item Los ap\'ostoles no usaron im\'agenes religiosas, s\'{i}, en cambio, predicaron la Palabra de Dios
\item No hay ninguna imagen o fotograf\'{i}a de Jes\'us, por lo tanto, para nosotros los que creemos sin haber visto (\emph{cf}. Jn 20,29), sigue teniendo validez Dt 4,15b.
\item Los cat\'olicos no \'unicamente tienen im\'agenes, sino que tambi\'en las adoran, les queman incienso, las besan, se inclinan ante ellas, les cuelgan milagritos\ldots
\end{itemize}

\subsubsection{Respuesta a las objeciones}

A lo que yo respondo:

\begin{itemize}
\item La transitoriedad (o condicional o como lo quieras llamar) est\'a contenida en la misma ley. Si la quitas eres t\'u el que est\'as modificando la ley; te conviertes en juez de la ley y no en su cumplidor (\emph{cf}. Sant 4,11).
\item El argumento <<X no aparece en la Biblia, por tanto, es falso>> no tiene sustento: en la Biblia tampoco viene que los ap\'ostoles se ba\~naran o que fueran al retrete ?`entonces no lo hac\'{i}an? Revisa Juan 21,25.%
    \footnote{Para profundizar en el \emph{por qu\'e} puedes leer: John Henri Newman; \emph{An Essay of Development of Christian Doctrine}; http://www.newmanreader.org/works/development/}
\item Respecto a <<no hay ninguna imagen de Jes\'us>> responderemos en seguida.
\item Respecto al trato que le dan los cat\'olicos a las im\'agenes (besarlas, inclinarse ante ellas, etc.) responderemos inmediatamente despu\'es.
\end{itemize}

\subsection{El esposo no dej\'o im\'agen alguna}

Como indicamos arriba, pudieras objetar: <<\emph{Dado que Jes\'us no dej\'o imagen alguna, luego entonces sigue vigente Dt 4,15b y por tanto los cat\'olicos son id\'olatras, etc}>>.

A lo que yo respondo: ?`Es que nadie lo vio? ?`Por qu\'e todas las im\'agenes que hay de \'El se parecen variando algunos matices? ?`Si te pongo una imagen de Moloc y una de Jes\'us, podr\'{i}as distinguir cu\'al es cu\'al?

El tema es muy profundo y digno de estudiarse,%
    \footnote{Revisa, por ejemplo, el siguiente documento:\\
    http://estetica.uc.cl/images/stories/Aisthesis1/Aisthesis32/imgenes\%20de\%20cristo
    \%20en\%20el\%20arte\%20paleocristiano\_\\
    lis\%20hernan\%20errazuriz\%20l-elda\%20r\%20balbontin\%20b.pdf}
yo lo voy a exponer desde otro enfoque, mediante la continuaci\'on de la analog\'{i}a que ven\'{i}amos tratando acerca de la mujer desposada con un hombre cuyo rostro ignoraba:

%TODO:objecion en Ga 4,8-9
\begin{quote}
\emph{Finalmente lleg\'o el novio y tuvo lugar la boda. Poco tiempo despu\'es el esposo parti\'o a un lugar lejano con la promesa de volver. La esposa estaba embarazada.}%
    \footnote{Gal 4,26}
    
\emph{Naci\'o un hijo que nunca vio a su padre, cuando aprendi\'o a hablar le preguntaba a su madre: <<?`C\'omo es pap\'a?>> y su madre continuamente se lo explicaba.}

\emph{En cierta ocasi\'on, como suelen hacer los ni\~nos peque\~nos, lleg\'o con su madre y le mostr\'o un dibujo diciendo: <<mira, he dibujado a pap\'a, aqu\'{i} estoy yo y por ac\'a est\'as tu tambi\'en>>.}

\emph{Como todas las madres, guard\'o con mucho cari\~no el dibujo de su peque\~no artista; francamente no guardaba ning\'un parecido f\'{i}sico con su padre: en realidad era una pelota algo abollada colocada sobre una l\'{i}nea vertical que m\'as bien parec\'{i}a serpentear y le brotaban otras l\'{i}neas que pretend\'{i}an ser los brazos y piernas\ldots as\'{i} dibujan los ni\~nos peque\~nos.}

\emph{Fueron multiplic\'andose los dibujos teniendo gran variedad unos respecto de otros: en algunos aparec\'{i}a delgado y en otros gordo, formando un \'ovalo la cabeza y otro m\'as grande el cuerpo\ldots La Madre iba acrecentando la <<galer\'{i}a>> pensando: <<se los mostrar\'e a su padre cuando vuelva>>.}

\emph{A medida que iba creciendo, fue mejorando su habilidad para dibujar y fueron siendo m\'as realistas, m\'as acorde con las descripciones que de \'el hac\'{i}a su madre; ella, por su parte, fue insistiendo cada vez m\'as sobre los razgos interiores de su padre mientras la descripci\'on f\'{i}sica era apenas un esbozo.}

\emph{La mujer ya madura por las experiencias y el trato con el esposo fue comprendiendo que la mejor manera de lograr que su hijo reconociera a su padre cuando lo viera era que lo identificara por su alma, sus virtudes, lo que ha hecho por ellos\ldots mientras que el aspecto f\'{i}sico servir\'{i}a de instrumento para reflejar estos razgos.}

\emph{La madre sigui\'o guardando los dibujos con mucho cari\~no, pues expresaban el amor hacia el padre del ni\~no, hacia su esposo, y reflejaban lo que el ni\~no iba aprendiendo acerca de la bondad, la dulzura y la amabilidad de su padre.}
\end{quote}

\noindent
Sabes perfectamente de lo que estoy hablando cuando menciono a un ni\~no de unos dos o tres a\~nos de edad dibujando una bola con patas y que dice: <<es pap\'a>>. ?`es o no es pap\'a? Ambos sabemos que \emph{aunque no se parezca en absoluto} s\'{i} que lo es. Y lo es especialmente porque brota del amor del peque\~no; y no hay peligro porque su madre corrige cuando hay algo que exprese una idea equivocada, por ejemplo, si dibuja a pap\'a como mujer o como desecho sanitario.

\subsubsection{Breve introducci\'on al arte cristiano}

Lo mismo pasa con las im\'agenes cat\'olicas: al principio eran meros s\'imbolos y figuras a las que se le dio nuevo significado; a los cristianos que iban a morir en el circo romano, les interesaba transmitir, mediante el arte, la verdad de la fe. Fueron haciendo m\'as y m\'as im\'agenes resaltando diversos aspectos sobre Jes\'us, sobre los profetas, sobre la historia de salvaci\'on, sobre sus compa\~neros que hab\'{i}an derramado su sangre por fidelidad a la cruz.

A medida que pasaba el tiempo, la inspiraci\'on fue madurando juntamente con la fe y mediante el arte fueron expresando verdades teol\'ogicas m\'as profundas y con una mayor perfecci\'on est\'etica.

Cualquier libro de historia del arte cristiano te lo puede explicar.

Tambi\'en tuvo lugar la superaci\'on de la mentalidad jud\'{i}a: primero fue dejar la circuncisi\'on, despu\'es fue dejar de ir a la sinagoga (pues los primeros cristianos iban a la sinagoga en s\'abado y se reun\'{i}an a celebrar la Eucarist\'{i}a el domingo); para muchos, la destrucci\'on del Templo de Jerusal\'en fue un golpe muy duro, pero tambi\'en lo superaron; entre estas cosas est\'a el uso de las im\'agenes, este proceso es el que venimos describiendo. Otro ejemplo es el no considerar a la mujer del pr\'ojimo como una posesi\'on suya (como en Ex 20,17) sino como era \emph{en el principio} (Mt 19,8): una \emph{ayuda semejante} (Gen 2,18 y Dt 5,21).

\textbf{No se trata de \emph{contaminaci\'on con el paganismo},%
    \footnote{El tema de la <<contaminaci\'on con el paganismo>> es de tal amplitud que merece un tratado aparte; hasta ahora, todos los intentos que he visto de justificar esta postura son insostenibles. Revisa, por ejemplo, de Bruce Sullivan (ex-evang\'elico convertido al catolicismo): 
    http://www.primeraluz.org/index.php?option=com\_content\&view=article\&id=472}
sino de maduraci\'on de la fe en la encarnaci\'on del Hijo de Dios y sus consecuencias. Se trata de desarrollo \emph{desde adentro}.%
    \footnote{John Henri Newman; \emph{An Essay of Development of Christian Doctrine}; http://www.newmanreader.org/works/development/}}
    
Todav\'{i}a podr\'{i}as decir: \emph{!`qu\'e pretexto tan rid\'{i}culo para anular el mandato de Dios con la tradici\'on! (Mt 15,6)}; yo respondo: ?`Qu\'e es lo que muestran las im\'agenes cat\'olicas? ?`Acaso ves alguna imagen de Cristo crucificado con el rostro de Moloc? el arte cristiano expresa lo que quiere expresar, no lo que t\'u quieres que exprese, expresa la verdad teol\'ogica: por ejemplo, el \emph{Pantokr\'ator}%
    \footnote{https://es.wikipedia.org/wiki/Pantocr\%C3\%A1tor}
muestra a Jes\'us como soberano universal que gobierna mediante su Palabra y todo le est\'a sometido (Col 1,12-20).

%TODO:objecion:no podrian los israelitas considerar las acciones de Dios y plasmarlas en imagenes y asi ? por que no lo hicieron? aqui se ve que estaba prohibido y que tu arg. no tiene validez
As\'{i} que no se trata simplemente del \emph{parecido f\'{i}sico}, sino de la verdad expresada en el arte y de la intenci\'on del artista de expresar \emph{esa} verdad y no otra cosa.

\textbf{Preg\'untate, en primer lugar, si crees que Jes\'us, siendo Dios, se ha hecho hombre; y que no ha dejado de serlo. \emph{Admitir las im\'agenes religiosas es confesar el misterio de la encarnaci\'on}}, pues Cristo, \emph{<<siendo de condición divina, no retuvo ávidamente el ser igual a Dios. Sino que se despojó de sí mismo tomando condición de siervo haciéndose semejante a los hombres y apareciendo en su porte como hombre>>} (Flp 2,6-7 LBJ).

Por \'ultimo resta discutir acerca del trato que los cat\'olicos le damos a las im\'agenes, a lo que respondemos en seguida.

\subsection{La intenci\'on remite al original}

Respecto al trato que le damos los cat\'olicos a las im\'agenes, podr\'{i}as objetarme:

\begin{quote}
\emph{Los cat\'olicos dicen que no adoran las im\'agenes, sin embargo se arrodillan ante ellas, las sirven (Ex 20,4-6), les queman incienso, las llevan de un lugar a otro, ponen su confianza en ellas, las besan, les dirigen oraciones y rezos\ldots si eso no es adoraci\'on entonces ?`qu\'e es?}
\end{quote}

\noindent
Los cat\'olicos no hacen de los objetos religiosos (im\'agenes, estatuas, etc.) los \emph{destinatarios} de sus pr\'acticas piadosas, sino que \emph{el honor rendido a la imagen se traspasa al prototipo que representa y el que venera la imagen venera la persona que la imagen representa.}\footnote{II Concilio de Nicea (a\~no 787)}
%http://www.mercaba.org/Herejia/iconoclastas.htm
%https://books.google.com.mx/books?id=78kzIHWKljsC&pg=PA36&lpg=PA36&dq=el+honor+de+la+imagen+prototipo+nicea&source=bl&ots=a0zjf5yjfr&sig=p2LZfitxSu7HCXMrkgAS4T1McI8&hl=es&sa=X&ved=0ahUKEwiX5qCNqdXJAhWDjIMKHZKaCdUQ6AEINTAF#v=onepage&q=el%20honor%20de%20la%20imagen%20prototipo%20nicea&f=false

Para ponerlo mas claro (perdona el ejemplo pero es para darme a entender con claridad): \emph{imag\'{i}nate por un momento que le muestras a un amigo la foto de tu amada esposa ?`si la amas verdad? y \'el toma esa fotograf\'{i}a, la rompe, la tira al piso, la escupe y la pisotea}. ?`C\'omo te sentir\'{i}as? seguramente muy ofendido, ?`pero por qu\'e si la foto claramente \emph{no es} tu esposa? porque la \emph{representa} y adem\'as \emph{es evidente que esas acciones denigrantes van dirigidas a tu esposa representada en la foto.}

Tomemos otro ejemplo: tu hijo menor hizo a los 2 a\~nos un dibujo de su mam\'a y llega el mismo sujeto y realiza las mismas acciones. El dibujo \emph{claramente} no se parece al original. Sin embargo, la sigue representando y \emph{el resultado es el mismo}, incluso mayor, porque en la fotograf\'{i}a se muestran los rasgos f\'{i}sicos, mientras que en el dibujo se muestran los rasgos interiores vistos por un ni\~no: su bondad, su ternura, etc.

%TODO:poner la cita textual (de una buena traduccion) del concilio de Nicea

\section{?`Es b\'{i}blico el \emph{argumento de excepci\'on}?}

En esta secci\'on analizaremos la cita de Dt 4,2:

\begin{quote}
<<No a\~nad\'ais nada a lo que yo os prescribo, ni nada quit\'eis, sino guardad los mandamientos de Yahv\'e, vuestro Dios, que yo os prescribo>>\\
Dt 4,2 NC
\end{quote}

%TODO:eliminar fbox
\fbox{ideas sueltas}
%TODO:analizar Deuteronomio 4,2
dificultades:

1.prohibicion relativa, no absoluta:

1.1. prohibicion de adorar, no de tener/venerar/etc

2. sin autorizacion de Dios, se inclinan ante el arca: Jos 7,6; se lleva tierra 2 Re 5,17; toca el manto Mc 5,27-29; usan su ropa para curar Hch 19,11-12

3. el caso de la legitima defensa indica ademas que hay casos que se establecen por la razon

4. no indica que sea la palabra escrita unicamente, sino TODA su revelacion, lo cual tambien incluye la tradicion oral y su interpretaci\'on

4.1. precisa interpretacion para saber cual es el espiritu de la revelacion

Una idea que ya tra\'ia desde antes:
\begin{quote}
El \emph{argumento de excepci\'on} no es \emph{b\'{i}blico}. No hay \emph{ninguna} parte de la escritura donde est\'e enunciado. No se puede \emph{inferir} que lo haya usado nadie en la Biblia. Y siendo para ti, la Biblia, la \emph{autoridad final}, entonces no puedes utilizar ese argumento \emph{con autoridad}. ?`o has dejado de creer en la \emph{sola scriptura}?
\end{quote}

\fbox{fin ideas sueltas}

%%%%%%%%%%%%%%%%%%%%%%%%%%%%%%%%%%%%%%%%%%%%%%%%%%%%%%%%%%%%%%%%%%%%%%%
% \section{Insostenibilidad del principio de \emph{Sola Scriptura}}

% \begin{flushright}
% \emph{<<{\ldots}pero si tardo, para que sepas cómo hay que portarse\\
% en la casa de Dios, que es la Iglesia del Dios vivo,\\
% columna y fundamento de la verdad>>}.\\
% 1 Tim 3,15 LBJ
% \end{flushright}

% \noindent
% El esfuerzo de descalificar el culto cat\'olico a las im\'agenes parte del presupuesto de que el principio llamado \emph{Sola Scriptura} es v\'alido, m\'as a\'un, que forma parte de la misma revelaci\'on.

% Sin embargo esto no es verdad; tal argumento es una \emph{tradici\'on de hombres} (\emph{cf.} Mt 15,1-9; Mc 7,6-13) muy posterior, aproximadamente 1500 a\~nos despu\'es de que Jesucristo envi\'o a sus ap\'ostoles a predicar.

% Voy a enunciar una serie de razones por las que considero que el principio de \emph{Sola Scriptura} es inv\'alido.

% En primer lugar, est\'a el hecho de que t\'u mismo usaste argumentos que no est\'an en la Biblia para sostener tu posici\'on. En segundo lugar, el principio de \emph{Sola Scriptura} no es b\'{i}blico. En tercer lugar, este mismo principio deja sin explicaci\'on a la misma Biblia.

% En torno a estas tres ideas giran las razones que enuncio a continuaci\'on para mostrar la invalidez de considerar la Biblia como \emph{\'unica regla de fe}.

% \subsection{En qu\'e consiste el principio de \emph{Sola Scriptura}}


% \subsection{Tus argumentos se alejan del principio de \emph{Sola Scriptura}}


% \subsection{Invalidez del argumento: <<Si no hay ejemplos en la Biblia entonces es falso/malo/inv\'alido>>}


% \subsection{La <<autoridad final>> seg\'un la Biblia}


% \subsection{El texto b\'{i}blico requiere interpretaci\'on}


% \subsection{Insuficiencia de informaci\'on en los textos: un ejemplo: la fiesta del Yom Kippur}


% \subsection{Un par\'entesis: Suficiencia Material y Formal en la Biblia (tomado de Dave Armstrong)}

% \begin{verbatim}
%     #VI. Material and Formal Sufficiency of Scripture
%     <http://socrates58.blogspot.mx/2006/11/bible-church-tradition-canon-index.html>
%     <http://socrates58.blogspot.mx/2004/04/material-vs-formal-sufficiency-of.html>
% \end{verbatim}

% \subsection{Si el principio de \emph{Sola Scriptura} fuera v\'alido la Biblia no existir\'{i}a}


% \subsection{?`En qu\'e libro de la Biblia viene qu\'e libros son inspirados? La formaci\'on del canon}


% \subsection{Trasfondo hist\'orico del principio \emph{Sola Scriptura}}


% \subsection{El principio de \emph{Sola Scriptura} termina haciendo del sujeto que lee, <<la autoridad final>>}

%%%%%%%%%%%%%%%%%%%%%%%%%%%%%%%%%%%%%%%%%%%%%%%%%%%%%%%%%

\section{Posibles objeciones y sus respuestas}

\begin{enumerate}
\item Si el argumento de excepci\'on fuera inv\'alido cualquiera podr\'{i}a inventar excusas para violar los preceptos divinos
\item objecion en Ga 4,8-9
\item objecion:no podrian los israelitas considerar las acciones de Dios y plasmarlas en imagenes y asi ? por que no lo hicieron? aqui se ve que estaba prohibido y por lo tanto Dt 4,15b sigue vigente

    o sea: los israelitas no hicieron r\'eplicas del arca, la serpiente, ni el templo, era impensable --> porque esos objetos eran singulares, esos objetos eran mandados por Dios, --> esos objetos ten\'{i}an el poder de Dios ... en cambio, nunca hicieron replicas, por qu\'e ?

\end{enumerate}

% \subsection{Sobre la leg\'{i}tima defensa}

% <<No existe tal cosa llamada leg\'{i}tima defensa>>

% El mandamiento en Ex 20,13 si consideramos su original en hebreo es: <<no asesinar\'as>>, es decir, no privar a alguien de la vida injustamente.

% \subsection{Si el argumento de excepci\'on fuera inv\'alido cualquiera podr\'{i}a inventar excusas para violar los preceptos divinos}

% \subsection{El argumento de excepci\'on aplicado al segundo mandamiento se refiere a hacer im\'agenes, no a la idolatr\'{i}a}

% \subsection{El Arca, el Templo, la serpiente de bronce, siendo dise\~nados por Dios no nos llevan a la corrupci\'on}

% Revisa 2 Re 18,-35

% \subsection{Los cat\'olicos se inclinan, reverencian y sirven a las im\'agenes, ponen su confianza en ellas}


% \subsection{Objeci\'on contra la expresi\'on: <<la intenci\'on remite al original>> usada en el concilio de Efeso}


% \subsection{Sobre <<interpretar>> las Escrituras}


% \subsection{Sobre el culto en la fiesta del \emph{Yom Kippur}}


% \subsection{<<Ustedes han anulado el mandato de Dios con vuestra tradici\'on>> (\emph{c.f.} Mc 7,9)}


% \subsection{La Iglesia Cat\'olica se desacredita por sus obras, no puede ser de Dios, <<por sus frutos los conocer\'as>> (\emph{c.f.} Mt 7,16.20; Lc 6,44)}


\section{Conclusi\'on}

\nocite{*}
\bibliographystyle{plain}
\bibliography{references}
\end{document}

