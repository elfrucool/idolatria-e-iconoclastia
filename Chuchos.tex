\documentclass{article}

\usepackage[spanish,es-noindentfirst]{babel}

\usepackage[margin=1in]{geometry}

\usepackage[utf8]{inputenc}
\usepackage[T1]{fontenc}

\title{Idolatria e Iconoclastia}
\author{Gustavo Serrano \\ \tt{gustavo.serrano.diez@gmail.com}}
\date{March 2015}

\usepackage{natbib}
\usepackage{graphicx}

\setcounter{tocdepth}{2}

\begin{document}

\maketitle

\begin{abstract}
\noindent
Esta es mi posici\'on respecto a los comentarios que me hiciste llegar la \'ultima vez.
\end{abstract}

\tableofcontents

\section{Introducci\'on}

Llamo \emph{argumento de excepci\'on} a aquel que enuncia que la \'unica raz\'on para apartarse de lo que dice alg\'un mandamiento de la Ley de Dios seg\'un Ex 20,3-17 y Dt 5,7-21 es precisamente cuando Dios ordena expl\'{i}citamente lo contrario; en tal caso, ser\'{i}a pecado no obedecerle.

Las citas b\'{i}blicas textuales est\'an acompa\~nadas de la versi\'on de que fueron tomadas de la siguiente forma:

\begin{description}
\item[LBJ] Biblia de Jerusal\'en 1967
\item[NC] Biblia N\'acar Colunga 1944
\item[RVR] Biblia Reyna-Valera 1960
\end{description}

\noindent
En caso de ser meras referencias omitir\'e tal abreviaci\'on. Si son citas no textuales antepondr\'e \emph{cf}.

En el caso de que una cita b\'{i}blica forme parte de la cita textual de otra fuente, la dejar\'e tal cual aparece en la fuente citada.

\section{Las dos iglesias cat\'olicas}

As\'{i} que decidiste dejar la iglesia cat\'olica. Ha sido la decisi\'on m\'as importante de tu vida. Seguramente no lo hiciste a la carrera. Te tomaste tu tiempo y te aseguraste de que fuera la decisi\'on correcta. 

Cuando el Se\~nor te habl\'o, te hizo ver tu pecado. Fuste obediente a su Palabra y saliste de las tinieblas de la idolatr\'{i}a para entrar en su luz admirable (\emph{cf.} 1 Pe 2,9).

Sin embargo creo que no conociste la Iglesia que abandonaste. Debo contarte una historia para que me entiendas mejor.

\subsection{Una an\'ectdota m\'as bien personal}

Cuando era adolescente, yo estaba convencido de que en la Iglesia hab\'{i}a dos clases de cat\'olicos: los de \emph{nombre} y los practicantes.

Para mi era tan clara la distinci\'on entre ambos grupos que f\'acilmente pod\'{i}a se\~nalar qui\'en estaba en un grupo y qui\'en en otro.

Entonces yo estaba convencido de que los cat\'olicos de \emph{nombre} deb\'{i}an dejar de ser cat\'olicos y admitir su ate\'{i}smo o su paganismo en lugar de estar desprestigiando con su mala conducta a los cat\'olicos de \emph{a deveras}.

Debo admitir que yo estaba francamente equivocado y que la cosa no es tan simple. ?`Qui\'en decide en cu\'al grupo est\'a cada qui\'en? Por algo Dios se reserva el juicio para s\'{i} mismo y nos advierte que no somos nadie para juzgar al hermano (\emph{cf.} Sant 4,11-12).

Sin embargo, creo que la distinci\'on nos puede ayudar para dejar en claro algunas ideas.

\subsection{La iglesia cat\'olica que dejaste}

\begin{flushright}
\emph{{\ldots}y en su frente un nombre escrito, un misterio:\\
<<Babilonia la Grande, la madre de las rameras\\
y de las abominaciones de la tierra>>}.\\
Ap 17,5 RVR
\end{flushright}

\subsection{La Iglesia Cat\'olica que no conociste}

\begin{flushright}
\emph{<<Ven, te mostrar\'e a la desposada,\\
la Esposa del Cordero>>}.\\
\emph{cf.} Ap 21,9
\end{flushright}

\subsection{Para reflexionar}

Medita en tu coraz\'on la par\'abola del trigo y la ciza\~na (Mt 13,24-30).

\section{Invalidez del argumento de excepci\'on}

A continuaci\'on voy a mostrar suscintamente por qu\'e el argumento de excepci\'on es inv\'alido. 

Primero describir\'e a fondo en qu\'e consiste este argumento seg\'un he entendido todo lo que me has escrito.%
    \footnote{Pongo mi mayor empe\~no en no caricaturizar o minimizar tu argumento sino expresarlo con toda la solidez que tiene. Ya has recibido el resumen que hice a tus observaciones y la presente descripci\'on, aunque breve, pretende seguir la misma l\'{i}nea y consideraci\'on.}

Posteriormente expondr\'e las dos razones por las que el argumento carece de validez: a) para el mandamiento <<no matar\'as>> y b) para el caso de la idolatr\'{i}a.

\subsection{?`En qu\'e consiste el argumento de excepci\'on?}

El argumento central de tu respuesta se basa en la siguiente premisa:

\begin{quote}
\emph{La \'unica raz\'on para contravenir un mandamiento expl\'{i}cito de Dios es que Dios mismo haya ordenado otra cosa. Siendo \'unicamente cuando Dios lo manda y sin hacerlo por cuenta propia.}
\end{quote}

\noindent
Las caracter\'{i}sticas de una excepci\'on son las siguientes:

\begin{enumerate}
\item Dios dispone \emph{expl\'{i}citamente} una excepci\'on a alguno de sus mandamientos
    
    Por \emph{expl\'{i}cito} me estoy refiriendo a que simult\'aneamente: a) est\'a escrito en la Biblia, b) con una claridad que no deja lugar a dudas y c) es Dios el autor de la disposici\'on, no los hombres

\item S\'olo aplica para aquellos casos y personas que Dios dispone
\item En el tiempo y la forma en que \'El lo dispone
\item No es l\'{i}cito al hombre tomar esos casos para hacer \emph{excepciones} por cuenta propia
\item No es l\'{i}cito al hombre tomar esos casos para hacer \emph{ampliaciones} por cuenta propia
\item Siendo mandato expreso de Dios, es pecado no obedecerle
\end{enumerate}

\noindent
La primera caracter\'{i}stica enunciada es la que determina si se trata de una excepci\'on v\'alida o no, debe cumplirse a cabalidad. Las dem\'as caracter\'{i}sticas son las consecuencias e implicaciones. El ejemplo, claro y di\'afano est\'a expresado de la siguiente manera:

\begin{quote}
\emph{Dios dice: <<no matar\'as>>, el \'unico caso en que es permitido matar a un hombre es cuando Dios lo ordena, porque Dios lo manda, s\'olo cuando as\'{i} lo indica. En este caso es pecado no obedecerle. Pero no es l\'{i}cito al hombre tomar este ejemplo para hacer por cuenta propia sus <<excepciones>>.}
\end{quote}

\subsection{Para el mandamiento <<no matar\'as>> hay excepciones no fijadas por Dios}

Sin embargo para el caso <<no matar\'as>> hay un caso en el que es l\'{i}cito a un hombre matar a otro y sin embargo no es mandado por Dios expl\'{i}citaemente; me refiero al caso de la \emph{leg\'{i}tima defensa}.

\subsubsection{?`En qu\'e consiste la leg\'{i}tima defensa?}

Tenemos polic\'{i}as y ej\'ercito; y los tenemos armados; y esperamos que usen sus armas para defendernos incluso si eso significa matar al agresor; adem\'as, podemos vernos en la necesidad de defendernos por nuestra cuenta sin el auxilio de los poderes p\'ublicos.

Bajo el t\'ermino \emph{leg\'{i}tima defensa} estoy abarcando tres categor\'{i}as relacionadas: a) a nivel individual es la propiamente llamada \emph{leg\'{i}tima defensa}; b) la defensa de los ciudadanos realizada en tiempos de paz por la polic\'{i}a y poderes p\'ublicos equivalentes; y c) la as\'{i} llamada \emph{guerra justa}.\footnote{La \emph{pena de muerte} es considerada en los textos cl\'asicos como un caso particular en que una comunidad (ciudad o estado) se defiende de un agresor, siendo su \'ultimo recurso el privarle la vida. Trat\'andose de un tema muy controversial prefiero dejarlo de lado.}

Los tres casos mencionados tienen en com\'un dos caracter\'{i}sticas: a) se trata de acciones donde el que las ejerce pudiera privar de la vida a un agresor; incluso deliberadamente y b) son acciones consideradas \emph{defensivas} cada una seg\'un su \'ambito y nivel.

?`Son estos casos aceptables? ?`bajo qu\'e condiciones? Parece ser que no toda intervenci\'on armada es justa, incluso si es meramente defensiva; determinar la legitimidad de matar en defensa propia exige condiciones m\'as precisas:%
    \footnote{Cito aqu\'{i} el n\'umero 2309 del Catecismo de la Iglesia Cat\'olica, por ser el que lo expresa de la manera m\'as concisa; las causas enunciadas son las de la \emph{guerra justa} pero no te costar\'a trabajo hacer las adecuaciones para los dem\'as casos de la leg\'{i}tima defensa. Puedes consultarlo en la siguiente liga: http://www.vatican.va/archive/catechism\_sp/p3s2c2a5\_sp.html}

\begin{enumerate}
\item Que el daño causado por el agresor a la nación o a la comunidad de las naciones sea duradero, grave y cierto.
\item Que todos los demás medios para poner fin a la agresión hayan resultado impracticables o ineficaces.
\item Que se reúnan las condiciones serias de éxito.
\item Que el empleo de las armas no entrañe males y desórdenes más graves que el mal que se pretende eliminar. El poder de los medios modernos de destrucción obliga a una prudencia extrema en la apreciación de esta condición.
\end{enumerate}

\noindent
Las condiciones arriba citadas para la validez de la leg\'{i}tima defensa no est\'an \emph{expl\'{i}citamente} enunciadas en la Biblia como veremos a continuaci\'on.

\subsubsection{Argumentaciones b\'{i}blicas insuficientes}

Aparentemente es de sentido com\'un que existen condiciones que hacen l\'{i}cito el uso de la fuerza para defenderse, incluso si como consecuencia el defensor da muerte al agresor; sin embargo, no hay ning\'un vers\'{i}culo que lo diga con claridad y que permita saber sin ambig\"uedad bajo qu\'e condiciones aplica este principio.

Me tom\'e la libertad de buscar fundamentaciones b\'{i}blicas de tipo cristiano-evang\'elicas acerca de la leg\'{i}tima defensa; estos son los sitios que me parecen m\'as pertinentes y completos:%
    \footnote{En espan\~nol \'unicamente est\'a el estudio de Tito Mart\'{i}nez (citado y re-citado en m\'ultiples sitios web), no lo utilizo por dos razones: a) insuficiente, b) a juzgar por otros escritos suyos est\'a muy desviada su doctrina. Este es su estudio: http://www.las21tesisdetito.com/autodefensa.htm}

\begin{enumerate}
\item http://www.biblicalselfdefense.com/
\item http://www.kingjamesbibleonline.org/Bible-Verses-About-Self-Defense/
\item http://armedcitizensnetwork.org/the-bible-and-self-defense
\item http://www.nationalreview.com/corner/338845/biblical-and-natural-right-self-defense-david-french
\end{enumerate}

\noindent
Estas son las citas m\'as relevantes:

\begin{enumerate}
\item Neh 4,8-23
\item Est 8-9
\item Lc 22,35-38
\item Ap 11
\item Ex 22,2
\item Rom 13,1-4
\end{enumerate}

\noindent
Revisemos cada una de estas citas bajo los siguientes criterios:

\begin{description}
\item[Universalidad] se trata de un caso general no limitado a personas, grupos o tiempos particulares.
\item[Explicitez] se trata de una indicaci\'{o}n expl\'{i}cita y no de meras inferencias; est\'a hablando en sentido literal y no en sentido figurado.
\item[Autoridad] es Dios mismo el que lo indica as\'{i} y no los hombres por su cuenta.
\end{description}

\noindent
Revisemos cada uno de los pasajes y apliquemos a cada uno los criterios arriba citados.

\paragraph{Nehem\'{i}as 4,8-23}

El pasaje no deja ver con claridad ninguno de los tres criterios: a) \emph{Universalidad:} No es claro si aplica \'unicamente para ellos o para todo el que se encuentre en situaci\'on similar; b) \emph{Explicitez:} No es una indicaci\'on expl\'{i}cita; M\'as a\'un, de hecho \emph{no combatieron} y el v.20 al decir <<Dios pelear\'a por nosotros>> no permite saber si literalmente los israelitas no tendr\'an necesidad de pelear o bien si en caso de hacerlo, Dios les dar\'{i}a la victoria; y c) \emph{Autoridad:} No es Dios el que les da la indicaci\'on aunque no se los proh\'{i}be.

\paragraph{Esther 8-9}

Este pasaje carece de \emph{universalidad} y de \emph{autoridad} para ser considerado un caso v\'alido de leg\'{i}tima defensa. Fue un caso particular en un momento particular y Dios no se los orden\'o expl\'{i}cita o impl\'{i}citamente. Es, en cambio, manifiesto que los israleitas fueron autorizados por parte del rey pagano Asuero para quitar la vida a aquellos que los amenazaban y de hecho as\'{i} lo hicieron.

No hay ninguna indicaci\'on acerca de la aprobaci\'on o rechazo por parte de Dios a estas acciones simplemente se anota que de hecho sucedieron.

\paragraph{Ap 11}

Este pasaje habla de dos testigos y muestra claramente como se defienden de cualquiera que atente contra su vida (\emph{explicitez}) y que lo hacen con la aprobaci\'on y poder de Dios (\emph{autoridad}).

En el texto es muy claro que \'unicamente se refiere a los dos testigos mientras profetizan durante 1260 d\'{i}as. No puede aplicarse fuera de este caso. Carece de \emph{universalidad}.

\paragraph{Lucas 22,35-38}

Aqu\'{i} existe toda la \emph{autoridad} pues es Jes\'us el que habla.

Sin embargo, este pasaje carece de \emph{univeralidad} para ser aplicado en la leg\'{i}tima defensa, pues no permite ver si su aplicaci\'on es \'unicamente en ese momento o bien en cualquier situaci\'on de peligro.

Por otro lado, no es suficientemente \emph{expl\'{i}cito} si, en el caso de \emph{comprar una espada}, se refiere literalmente a comprar una o est\'a hablando en sentido figurado. Tampoco menciona el tema de privar la vida a un agresor. Y es evidente que Jes\'us NO se estaba refiriendo a resistir a los soldados de los sumos sacerdotes, pues \emph{ten\'{i}a que cumplirse la Escritura}.

Cuando en el v.38 los ap\'ostoles le dicen a Jes\'us que tienen dos espadas, \'El les responde: <<Basta>> (\emph{c.f.} Lc 22,38); pero no queda claro si <<Basta>> quiere decir \emph{con eso es suficiente para defendernos} o bien \emph{no me entendieron as\'{i} que dejemos de hablar de <<espadas>>}. De hecho la expresi\'on griega $\iota\kappa\alpha\nu o \nu$ (suficiente) $\varepsilon\sigma\tau\iota\nu$ (est\'a siendo) tiene ambas acepciones.%
    \footnote{Puedes consultar cualquier Nuevo Testamento griego-espa\~nol interlineal y alg\'un diccionario de l\'exico griego del Nuevo Testamento.}
    
Por mi parte, aunque no estoy cerrado a la interpretaci\'on literal, me inclino m\'as a que Jes\'us estaba hablando en sentido figurado y los ap\'ostoles lo entendieron en sentido literal; pues este sentido es el que va m\'as acorde con todo el pasaje, desde la \'ultima cena hasta que Jes\'us es llevado ante el sumo sacerdote.%
    \footnote{Puedes consultar un interesante estudio en la siguiente liga: http://www.biblestudytools.com/commentaries/utley/lucas/lucas22.html}

\paragraph{Ex 22,2}

En este caso s\'{i} se dan los tres elementos como indicamos anteriormente: \emph{universalidad, explicitez y autoridad}. Sin embargo, \'unicamente constituye un caso que es el del robo y no habla de mayores circunstancias que la de que \emph{es de noche}.

En los estudios consultados interpretan esta circunstancia diciendo que de noche significa que no es posible saber si el invasor va a robar o va a matar. Por esta raz\'on su muerte no es imputable, mientras que el d\'{i}a representa la claridad de intenciones y en tal caso, la muerte es imputable, pues la vida vale m\'as que la propiedad.

Por tanto, a partir de este texto, \'unicamente de manera parcial se puede inferir el principio de leg\'{i}tima defensa.

\paragraph{Rom 13,1-4}

Al igual que en Ex 22,2 se cumplen los tres criterios a cabalidad. Con todo, se trata de un \'unico caso, a saber: el derecho que tiene la autoridad para usar un poder letal (la espada) como instrumento para mantener el orden.

En este caso, la Biblia indica con claridad el \emph{sujeto} autorizado a usar la espada: el poder civil. A\'un as\'{i}, no hay informaci\'on suficiente acerca del modo adecuado de usar la espada y los casos en que es l\'{i}cito privar la vida a otro en el ejercicio de su deber.

Despu\'es de revisar los pasajes m\'as relevantes podemos formular algunas conclusiones y analizar las consecuencias.

\subsubsection{Conclusiones y Consecuencias}

Los pasajes anteriormente citados ponen de manifiensto que \textbf{el \emph{argumento de excepci\'on} carece de validez as\'{i} como lo has formulado}; esto por dos razones:

\begin{enumerate}
\item Las citas b\'{i}blicas analizadas son insuficientes para establecer de manera congruente y \emph{expl\'{i}cita} la existencia el principio de la leg\'{i}tima defensa siguiendo tu l\'ogica sobre las excepciones.
\item A\'un suponiendo que en los pasajes citados pudiera deducirse de manera impl\'{i}cita la leg\'{i}tima defensa, no hay informaci\'on suficiente en la Biblia para establecer con precisi\'on las circunstancias y modos de efectuarla.
\end{enumerate}

\noindent
\emph{De aqu\'{i} que sean los hombres quienes mediante el uso de la raz\'on que Dios les dio, provean estos casos y sus l\'{i}mites} con lo cual se invalida el \emph{argumento de excepci\'on}.

Siendo inv\'alido el argumento de excepci\'on para el mandamiento <<no matar\'as>>, podemos concluir que, adem\'as de las excepciones expl\'{i}citamente se\~naladas por Dios, existan exepciones no definidas o explicitadas por Dios en que es l\'{i}cito a un hombre privar de la vida a otro hombre.

\subsection{El argumento de excepci\'on no es aplicable al caso de la idolatr\'{i}a}

A\'un suponiendo que el argumento de excepci\'on fuera v\'alido para Ex 20,13 <<no matar\'as>> todav\'ia tropieza con un obst\'aculo: no es aplicable al caso de la idolatr\'{i}a.

La siguiente parte del estudio comienza por definir el t\'ermino \emph{idolatr\'{i}a} y posteriormente analiza si Dios hace excepciones para este caso. Inmediatamente despu\'es hacemos un an\'alisis a profundidad de Ex 20,4-6 revisando enunciado por enunciado si Dios hace excepciones a su mandamiento.

\subsubsection{Definicion de idolatr\'{i}a}

Del griego $\varepsilon\iota\delta\omega\lambda o$ y $\lambda\alpha\tau\rho\varepsilon\iota\alpha$; el vocablo $\varepsilon\iota\delta\omega\varsigma$ significa lo aparente, un reflejo sin realidad, un fantasma, una \emph{imagen} en sentido mental; la palabra \emph{idea} se deriva de este vocablo; por otra parte, $\lambda\alpha\tau\rho\varepsilon\iota\alpha$ significa adoraci\'on.

Bas\'andonos \'unica y exclusivamente en la etimolog\'{i}a, la definici\'on de \emph{idolatr\'{i}a} ser\'{i}a: \emph{adoraci\'on de las apariencias}.

Pero no basta con analizar el vocablo y su ra\'{i}z griega, es preciso conocer el uso que se le da en la Biblia; ahora bien, la versi\'on de los LXX emple\'o la palabra $\varepsilon\iota\delta\omega\lambda o$ para designar varios t\'erminos hebreos, la Gran Enciclopedia Rialp los enuncia de la siguiente manera:

\begin{quote}
Idolo (en griego eídólon) es la traducción más común de unos nombres hebreos, diversos entre sí. La palabra eídólon significa propiamente la imagen, el fantasma forjado por la fantasía. En la traducción del A. T. al griego por los Setenta se emplea para designar unas realidades más concretas, expresadas en el original hebreo por voces diversas: Selem: que significa «talla», «escultura» (Num 33,52); 'Asabbim: usado siempre en plural, significa «imagen tallada» (1 Sam 31,9); Semel: nombre de origen fenicio; significa «estatua de piedra» o «de madera» (Ez 8,3.5); Massékáh: «imagen fundida», en molde de arcilla (Ex 32,4.8); 'Eben maskith: «piedra con alguna imagen tallada» (Lev 26,1). La palabra maskith no significa necesariamente una imagen idolátrica; puede designar las imaginaciones de la fantasía. etc.\\
(Gran Enciclopedia Rialp)\footnote{No he podido dar con la versi\'on completa \emph{online}, en la siguiente liga hay algunos extractos: http://www.mercaba.org/Rialp/I/idolatria\_escritura.htm}
\end{quote}

\noindent
Teniendo esto en cuenta, podemos ampliar la definici\'on inicial para que quede como sigue:

\begin{quote}
El idólatra es el que <<\emph{aplica a cualquier cosa, en lugar de a Dios, la indestructible noción de Dios}>>\footnote{La cita es de un autor cristiano del siglo III: Orígenes, \emph{Contra Celsum}, 2, 40} y act\'ua en consecuencia: d\'andole \emph{adoraci\'on}.
\end{quote}

\noindent
Esta definici\'on abarca una serie de actitudes y acciones como las que sintetiza el Catecismo de la Iglesia Cat\'olica en su n\'umero 2113:

\begin{quote}
La idolatría no se refiere sólo a los cultos falsos del paganismo. Es una tentación constante de la fe. \emph{Consiste en divinizar lo que no es Dios. Hay idolatría desde el momento en que el hombre honra y reverencia a una criatura en lugar de Dios.} Trátese de dioses o de demonios (por ejemplo, el satanismo), de poder, de placer, de la raza, de los antepasados, del Estado, del dinero, etc. <<No podéis servir a Dios y al dinero>>, dice Jesús (Mt 6, 24). Numerosos mártires han muerto por no adorar a <<la Bestia>> (cf Ap 13-14), negándose incluso a simular su culto. La idolatría rechaza el único Señorío de Dios; es, por tanto, incompatible con la comunión divina (cf Gál 5, 20; Ef 5, 5).\footnote{http://www.vatican.va/archive/catechism\_sp/p3s2c1a1\_sp.html El subrayado es m\'{i}o.}
\end{quote}

\subsubsection{Dios no ordena la idolatr\'{i}a en ning\'un caso}

No hay NINGUN caso en toda la Escritura donde Dios ordene o permita que se de adoraci\'on a ning\'un otro que no sea \'El mismo.

Las citas sobre el Arca, la decoraci\'on del Templo y la serpiente de bronce no indican adoraci\'on a tales objetos; no pueden ser consideradas \emph{casos en los que Dios permite la idolatr\'{i}a} pues no hay \emph{adorac\'on} a tales objetos.

Puedes revisar 2 Re 18,3-4 para saber lo que le sucedi\'o a la serpiente de bronce cuando el pueblo de Israel le rindi\'o culto de adoraci\'on.

Por lo tanto \textbf{el \emph{argumento de excepci\'on} es \emph{inv\'alido} para el caso de idolatr\'{i}a}.

\subsubsection{Ci\~n\'endonos a lo que dice literalmente Ex 20,4-6}

Cuando dijiste en el comentario que me enviaste que \emph{lo mismo aplica para el caso de la idolatr\'{i}a} lo m\'as probable es que no te estuvieras refiriendo a la \emph{adoraci\'on}; pero ?`qu\'e hay entonces de atenerse a lo que dice literalmente Ex 20,4-6?

\begin{quote}
<<$^4$No te har\'as esculturas ni imagen alguna de lo que hay en lo alto de los cielos, ni de lo que hay abajo sobre la tierra, ni de lo que hay en las aguas debajo de la tierra. $^5$No te postrar\'as ante ellas, y no las servir\'as, porque yo soy Yahveh, tu Dios, un Dios celoso, que castiga en los hijos las iniquidades de los padres hasta la tercera y cuarta generaci\'on de los que me odian, $^6$y hago misericordia hasta mil generaciones de los que me aman y guardan mis mandamientos.>>\\ 
Ex 20,4-6 NC
\end{quote}

\noindent
En el pasaje arriba citado podemos identificar claramente cuatro partes:

\begin{enumerate}
\item La acci\'on material: <<No te har\'as esculturas\ldots>> (v4)
\item La acci\'on corporal: <<No te postrar\'as ante ellas>> (v5a)
\item La acci\'on espiritual: <<y no las servir\'as>> (v5b)
\item El motivo y las consecuencias: <<porque yo soy Yahveh, un Dios celoso\ldots>> (vv5c-6)
\end{enumerate}

\noindent
Revisemos con detenimiento las tres primeras partes (vv4-5b) pues constituyen el mandato de Dios.

\subsubsection{Dios, en algunos casos manda hacer im\'agenes y objetos semejantes}

Son los casos ya mencionados del Arca, la decoraci\'on del Templo y la serpiente de bronce. Y constituyen el centro del argumento de excepci\'on aplicado a Ex 20,4-6.

La \emph{premisa mayor}\footnote{Busca un manual de l\'ogica si tienes dudas sobre este t\'ermino, por ejemplo: http://es.wikipedia.org/wiki/Premisa} parece ser que Dios, en el Ex 20,4-6; prohibe \emph{tres} cosas: a) Hacer im\'agenes (v4), b) inclinarse ante ellas (v5a) y c) servirlas (v5b). Vuelve a leer este p\'arrafo, es esencial para entender el resto de esta secci\'on.

Suponiendo que esta tesis fuera cierta, revisemos en este apartado la primera prohibici\'on, dejando las otras dos para los apartados siguientes.

El argumento de \emph{excepci\'on} aplicado a la primera prohibici\'on quedar\'{i}a enunciado como sigue:

\begin{quote}
\emph{Ex 20,4 NC dice: <<No te har\'as esculturas ni imagen alguna de lo que hay en lo alto de los cielos, ni de lo que hay abajo sobre la tierra, ni de lo que hay en las aguas debajo de la tierra>>. El \'unico caso en que le es l\'{i}cito al hombre hacer im\'agenes es cuando Dios se lo indica, a quienes se lo indica, como se lo indica; tales son los casos del Arca, el Templo y la serpiente de bronce. No le es l\'{i}cito al hombre tomar estos ejemplos para hacer sus propias r\'eplicas o sus propias im\'agenes o dise\~nos por muy buenas y sinceras que sean sus intenciones.}
\end{quote}

\noindent
Est\'a muy claro: el Arca, el Templo y la serpiente de bronce son las \'UNICAS excepciones dispuestas por Dios. NO HAY M\'AS EXCEPCIONES. !`NINGUNA! No es l\'{i}cito a los hombres inventarse sus propios casos, no le est\'a permitido hacer sus propias r\'eplicas\ldots tampoco \emph{interpretar} la Escritura para <<anular el mandamiento de Dios con la tradici\'on>> (\emph{c.f.} Mc 7,9).

Por ejemplo: hacer un becerro de oro (Ex 32), una r\'eplica de Baal o Astart\'e (Num 25,3; Jue 2,13) o una estatuilla de Budha, Vishn\'u o Shiva.

Veamos algunos ejemplos un poco m\'as dif\'{i}ciles:

\begin{enumerate}
\item Tu credencial de elector con fotograf\'{i}a
\item Los animales de juguete de tus hijos
\item Hacer figuras de animales con plastilina
\item Las ilustraciones de la enciclopedia de tu familia, donde se ve la selva, las estrellas, el oc\'eano, \emph{lo que hay arriba en el cielo, abajo sobre la tierra y en las aguas debajo de la tierra} (\emph{c.f.} Ex 20,4).
\item Las fotos de tu madre, esposa e hijos
\end{enumerate}

\noindent
Creo deber\'{i}as deshacerte de esas abominaciones antes de acusarme de tener im\'agenes. Lo digo de verdad, si Ex 20,4-5b prohibe tres cosas entonces son tres cosas las que proh\'{i}be. No pretendo torcer tus argumentos a mi conveniencia, contin\'ua leyendo y ver\'as.

En estos momentos te viene a la mente la aclaraci\'on que me hiciste acerca de que en Ex 20,4 \emph{<<!`el punto malo es hacerlas con motivos religiosos!>>}, pues \emph{<<el hombre por motivos personales condenables en TODA la escritura es que se hace imágenes y las venera. ?`Y por qu\'e es condenable? precisamente porque las adoran y les sirven>>}; respondo con tus mismas palabras: \emph{<<NO FIJATE BIEN Y LEELO OTRA VEZ, ES HACERLAS>>}; no le a\~nadas a la Escritura tu interpretaci\'on diciendo: \emph{con motivos religiosos}. ?`En qu\'e libro de la Biblia, cap\'{i}tulo o vers\'{i}culo dice \emph{<<con motivos religiosos>>}?

Pero hay otra posibilidad: la posibilidad de que lo mencionado en Ex 20,4-6 se trate de \emph{una} sola prohibici\'on en lugar de tres; prohibici\'on en la cual el \emph{hacer im\'agenes} (v4) constituye el signo \emph{material} del acto \emph{interior} de servirlas (v5b).\footnote{M\'as adelante analizamos conjuntamente todo Ex 20,4-5b con m\'as precisi\'on.}

Si esto es as\'{i}, entonces \textbf{el \emph{argumento de excepci\'on} para la cita textual de Ex 20,4 es \emph{inv\'alido}, pues el v4 \emph{no puede separarse} del v5a-b}.

Antes de analizar todo el mandamiento en su conjunto, revisemos si es posible aplicar el \emph{argumento de excepci\'on} al texto: \emph{<<no te postrar\'as ante ellas>> (Ex 20,5a)}.

\subsubsection{Dios en ning\'un caso manda o permite postrarse ante las im\'agenes}

Ni siquiera ante las im\'agenes que \'El mismo manda realizar ordena o da permiso expl\'{i}cito para que los hombres se inclinen o se postren delante de ellas.

Sin embargo, tenemos el ejemplo \emph{expl\'{i}cito} de Josu\'e postr\'andose delante del Arca junto con todos los ancianos de Israel (Jos 7,6); implora el favor de Dios y \'El le responde, le aclara que el motivo de su disgusto es la prevaricaci\'on de los hijos de Israel, pues se apropiaron de objetos dados al anatema (Jos 7,1); sin embargo Yahveh no le reclama a Josu\'e ni a los ancianos el haberse postrado rostro en tierra \emph{delante} del Arca.

Podr\'as decirme que el Arca era especial y \'unica; que \emph{siendo dise\~nada por Dios no nos lleva a la corrupci\'on}\ldots pero el tema es otro, el tema es que Dios no da su consentimiento ni su mandato \emph{expl\'{i}cito} para lo que Josu\'e y los ancianos de Israel hicieron.

Por lo tanto \textbf{el \emph{argumento de excepci\'on} para la cita textual de Ex 20,5a es \emph{inv\'alido}}.

Revisemos ahora si el \emph{argumento de excepci\'on} tiene validez considerando Ex 20,5b antes de considerar todo el mandamiento en su conjunto.

\subsubsection{Dios en ning\'un caso manda servir a una imagen / estatua / pintura / etc}

Servir significa \emph{obedecer}, estar sujeto a la voluntad de otro, estar atento a sus deseos para cumplirlos.

Dios no ordena ni autoriza en ning\'un caso que se le de obediencia a las im\'agenes, ni siquiera a las que \'El mismo mand\'o hacer (el Arca, el Templo, la serpiente de bronce).

Por lo tanto \textbf{el \emph{argumento de excepci\'on} para la cita textual de Ex 20,5b es \emph{inv\'alido}}.

Estamos en condiciones para emitir una conclusi\'on general acerca del \emph{argumento de excepci\'on}. En la secci\'on siguiente analizaremos Ex 20,4-6 conjuntamente y con mayor profundidad pera tratar de aprehender cual es su significado.

\subsection{Conclusi\'on general sobre el \emph{argumento de excepci\'on}}

Mi conclusi\'on acerca del \emph{argumento de excepci\'on} es la siguiente: aunque no carece de l\'ogica; as\'{i} como lo has planteado, es decir, como la \emph{total exclusi\'on} de cualquier excepci\'on que no sea estrictamente \emph{se\~nalada por Dios} y totalmente \emph{particular} \textbf{\emph{NO TIENE VALIDEZ}}.

En los casos analizados Ex 20,13 y Ex 20,4-6 hemos visto que no es aplicable. Si no es aplicable, entonces \emph{est\'a abierto a otras opciones}.

Antes de analizar Ex 20,4-6 con mayor profundidad hagamos aqu\'{i} una anotaci\'on:

\begin{quote}
El \emph{argumento de excepci\'on} no es \emph{b\'{i}blico}. No hay \emph{ninguna} parte de la escritura donde est\'e enunciado. No se puede \emph{inferir} que lo haya usado nadie en la Biblia. Y siendo para ti, la Biblia, la \emph{autoridad final}, entonces no puedes utilizar ese argumento \emph{con autoridad}. ?`o has dejado de creer en la \emph{sola scriptura}?
\end{quote}

\noindent
Todav\'{i}a quedan muchas cuestiones por resolver para determinar si la pr\'actica cat\'olica de la veneraci\'on de las im\'agenes es conforme con las Santas Escrituras o por el contrario \emph{anula el mandato de Dios con la tradici\'on} (\emph{c.f.} Mc 7,6-13).

En la siguientes dos secciones abordaremos el tema, retomando el hilo de nuestro an\'alisis sobre Ex 20,4-6 donde se qued\'o.

\section{El pecado de idolatr\'{i}a es contra el primer mandamiento}

Una vez visto que no es posible separar Ex 20,4 de Ex 20,5a-b debemos analizar el texto conjuntamente para tener alguna inteligencia de lo que Dios est\'a mandando.

Comenzaremos analizando Ex 20,4-5b consider\'andolo como una unidad; del an\'alisis brotar\'a que se trata de una ampliaci\'on de Ex 20,3, lo cual queda confirmado por la unidad arm\'onica que abre en el Ex 20,3 y cierra en Ex 20,6 pues de otra forma se rompe la sem\'antica del texto.

Despu\'es de esto, entraremos de lleno a la cuesti\'on de si la Iglesia Cat\'olica \emph{escondi\'o} el segundo mandamiento o si los protestantes \emph{dividieron} el primero.

En la secci\'on siguiente analizaremos la legitimidad de la pr\'actica cat\'olica acerca de las im\'agenes.

\subsection{Unidad arm\'onica de Ex 20,3-6/Dt 5,5-10}

Dec\'{i}amos m\'as arriba que en Ex 20,4-6 pod\'{i}amos identificar cuatro partes, esta divisi\'on se centr\'o en distinguir aquello que Dios manda de otras informaciones contenidas en el mandamiento. Ahora nos conviene profundizar m\'as y hacer la distinci\'on completa:\footnote{Los escol\'asticos dec\'{i}an: \emph{<<unir sin confundir y distinguir sin separar>>} para hacer este tipo de an\'alisis en el que se identifican diferentes partes que forman un todo.}

\begin{enumerate}
\item La acci\'on material: <<No te har\'as esculturas ni imagen alguna de lo que hay en lo alto de los cielos, ni de lo que hay abajo sobre la tierra, ni de lo que hay en las aguas debajo de la tierra>> (v4 NC)
\item La acci\'on corporal: <<No te postrar\'as ante ellas>> (v5a NC)
\item La acci\'on espiritual: <<y no las servir\'as>> (v5b NC)
\item El motivo: <<porque yo soy Yahveh, un Dios celoso>> (v5c NC)
\item Las consecuencias de la desobediencia: <<que castiga en los hijos las iniquidades de los padres hasta la tercera y cuarta generaci\'on de los que me odian>> (v5d NC)
\item Las consecuencias de la obediencia: <<y hago misericordia hasta mil generaciones de los que me aman y guardan mis mandamientos>> (v6 NC)
\end{enumerate}

\noindent
Los primeros tres puntos (vv4-5b) constituyen la parte \emph{imperativa} del mandamiento. Revis\'emosla tomada como un \emph{todo} para comprender \emph{qu\'e} es lo que Dios est\'a mandando.

\subsubsection{Parte \emph{imperativa}}

Para analizar la parte \emph{imperativa}, nos ser\'a de utilidad el siguiente ejemplo tomado de la vida cotidiana: \emph{?`es malo caminar?} Respuesta: tomado en \emph{s\'i mismo} no es malo, es indiferente e incluso puede ser bueno, ya que proporciona salud al ejercitarnos. \emph{?`es malo visitar a una anciana y a su hermana?} Respuesta: visitar a los ancianos es una cosa muy biena, pues alivias su soledad. \emph{?`es malo asesinar a sangre fr\'{i}a a una anciana y a su hermana indefensas?}\footnote{El ejemplo lo tomo de la novela de Fiódor Dostoyevski, \emph{Crimen y Castigo}; en el contexto de la novela el homicida \emph{no} se est\'a defendiendo, asesina a sangre fr\'{i}a a dos ancianas inocentes, por lo que queda exclu\'{i}da toda posibilidad de \emph{leg\'{i}tima defensa}.} Respuesta es muy malo y la sangre de esas dos ancianas \emph{clama al cielo} (\emph{c.f.} Gen 4,10).

Y la pregunta es: \emph{?`cu\'al es el juicio moral de las acciones <<caminar>> y <<visitar a una aiciana y a su hermana>> cuando estas acciones van encaminadas a <<asesinar a la anciana y a su hermana>>?} Creo que estar\'as de acuerdo conmigo en que en tal caso, estas acciones son tan malas como el fin que se proponen pues son sus instrumentos.

El ejemplo que te acabo de dar, aunque no es id\'entico, tiene algunas similitudes con Ex 20,4-5b si consideramos los vv4-5b como una unidad, es decir: cuando el acto \emph{material} (hacer im\'agenes) y el acto \emph{corporal} (postrarse ante ellas) van encaminados al acto \emph{espiritual}: (servir a las im\'agenes) es cuando se da el \emph{pecado}. Pues para este caso, el acto \emph{espiritual} (v5b) es el que \emph{califica moralmente a los otros actos}.

Vuelve a leer el p\'arrafo anterior porque tiene consecuencias muy importantes.

\'Esta es la raz\'on por la que puedes con toda tranquilidad tener fotos de tu familia o una enciclopedia ilustrada. En estos casos, las im\'agenes no son utilizadas con el prop\'osito de servirlas.

Por otro lado, \'esta es la raz\'on por la que Josu\'e y los ancianos de Israel no cometieron pecado alguno al postrarse \emph{delante} del Arca (Jos 7,6).

\'Esta es la raz\'on por la cual Ezequ\'{i}as hizo lo que es recto a los ojos de Yahveh al destruir la serpiente de bronce que por indicaci\'on de Dios hab\'{i}a hecho Mois\'es (2 Re 18,3-4).

\'Esta es la raz\'on que t\'u intu\'{i}as cuando dec\'{i}as que el mandamiento s\'olo aplica cuando se hacen im\'agenes \emph{con prop\'osito religioso}. Sin embargo te falta precisi\'on y es importante, entonces, profundizar m\'as para identificar cu\'ales prop\'ositos religiosos est\'an prohibidos por el mandamiento y cu\'ales no.

Hasta aqu\'{i} las similitudes con el ejemplo de la vida cotidiana. En el caso del v4 es claro que se trata de una acci\'on cuya calificaci\'on moral proviene del objeto que se propone. Pero es importante revisar con m\'as detenimiento el v5a-b para aprehender su significado y extraer las consecuencias.

\paragraph{Significado de \emph{postrarse}}

En Ex 20,5a NC dice: <<no te postrar\'as ante ellas>>. ?`qu\'e es postrarse?

Postrarse tiene muchos significados y existen muchos vocablos tanto en hebreo como en griego, pero en los casos en que se habla de adoracion podemos entenderlo gr\'aficamente de la siguiente manera: el que se postra asume una postura corporal que significa abajamiento, reconocimiento de la superioridad del otro, sumisi\'on.

Sumisi\'on que puede ser \emph{absoluta} (adoraci\'on) como en: Ex 34,8 o Is 2,20; 44,15.17 o \emph{relativa} (sin adoraci\'on), como en: Gen 18,2; 37,7-10; 1 Sam 24,8 o Rut 2,10.

% \begin{enumerate}
% \item Gen 19,1
% \item 1 Cro 21,16
% \item Gen 17,17
% \item Gen 37,10; cf. 35;16-19
% \item Gen 42,6
% \item 1 Sam 2,36
% \item 2 Sam 14,4
% %shajah - postracion sin adoracion
% \item Gen 18,2
% \item 1 Sam 24,8 RVR (9 NC)
% \item Rut 2,10
% \item Gen 37,7-10
% %shahah - con adoracion
% \item 1 Sam 15,25 y Jer 7,2
% \item Ex 34,8
% \item Is 2,20; 44,15.17
% %gonupeteo
% \item Mt 17,14; Mc 1,40
% \item Mc 10,17
% \item Mt 27,29
% %proskuneo
% \item Mc 5,6 <-- no se si adorando o no
% \item Mt 8,2; 9,18; 15,25; 20,20; Ap 3,9 ?? <-- adorar
% %prospipto
% \item Mc 3,11; 5,33; 7,25; Lc 8,28.47; Hch 16,29
% %titemigonata
% \item Lc 22,41; Hch 7,60; 9,40; 20,36; 21,5
% %balo
% \item Mt 8,6
% %katastronnumi
% \item 1 Cor 10,5
% %katafero
% \item Hch 20,9
% \end{enumerate}

%http://archive.org/stream/DiccionarioBiblicoVine/DiccionarioBiblicoVine_djvu.txt

\paragraph{Servir a las im\'agenes}

Con el \emph{servicio} pasa algo similar a lo que pasa con la \emph{postraci\'on} tenemos dos casos: de forma \emph{absoluta}, como obediencia y sumisi\'on total (adoraci\'on) como en  Dt 6,13; Mt 4,10; o bien, de forma \emph{relativa}, como en: Mc 9,35; Lc 22,27 o como lo que hizo Jes\'us con sus acciones en el lavatorio de los pies (Jn 13).

\paragraph{Reuniendo las piezas de la \emph{parte imperativa} (Ex 20,4-5b)}

Reuniendo todo lo analizado hasta aqu\'{i} tenemos que la \emph{parte imperativa} significa lo siguiente:

\begin{quote}
Dios pide que s\'olo a \'El se de la obediencia \emph{absoluta} y que s\'olo a \'El se le otorgue el reconocimiento de la superioridad \emph{absoluta}; atribu\'{i}rselo a cualquier otro ser es abominaci\'on, por ejemplo, a las figuras que se fabrican con las manos. 
\end{quote}

\noindent
Como puedes ver, la composici\'on interna del texto nos dice que lo que Dios proh\'{i}be es la \emph{adoraci\'on} a otros dioses, es decir, \textbf{Ex 20,4-5b prohibe tener otros dioses fuera de Yahveh, igual que Ex 20,3}.

Por eso el que se fabrica im\'agenes para inclinarse ante ellas (atribuy\'endoles absoluta superioridad) y las sirve (con sumisi\'on absoluta) comete \emph{idolatr\'{i}a}, es decir, \emph{<<aplica a cualquier cosa en lugar de a Dios, la indestructible noci\'on de Dios>>}.

Esta afirmaci\'on se ve corroborada por la \emph{parte no imperativa} del texto (Ex 20,5c-6) y por Ex 20,3. Examinemos ambas partes por separado.

\subsubsection{Incorporando Ex 20,5c-6 al an\'alisis}

La parte \emph{no imperativa} del texto viene a confirmar lo que venimos analizando: que lo mandado en Ex 20,4-5b es \emph{la prohibici\'on de adorar a otros dioses}.

En el apartado anterior lo dedujimos de la \emph{estructura interna} de Ex 20,4-5b; ahora lo deduciremos del \emph{contexto inmediato} (externo), es decir, de Ex 20,5c-6 y Ex 20,3.

\paragraph{Los celos de Dios}

El Ex 20,5c NC dice: <<porque yo soy Yahveh, un Dios \emph{celoso}>>, la parte \emph{no imperativa} comienza explicando la raz\'on de la \emph{parte imperativa}. Y el motivo es que Yahveh es un Dios \emph{celoso}.

Los \emph{celos} en la Escritura tienen un significado muy profundo. Dios se \emph{ha desposado} con su pueblo (Is 54; 62,1-5; Ez 16,1-14; Ef 5,25-27) y tener otros dioses es como ser infiel al esposo (Jer 3; Ez 16,15-58). El tema central del libro de Oseas es este drama esponsal.

As\'{i} que ir en pos de otros dioses es para el creyente, lo mismo que para la esposa ir en pos de otros hombres.%
% Es como si Dios dijera: \emph{no tengas fotograf\'{i}as de otros pretendientes}. Esta analog\'{i}a de las fotograf\'{i}as nos servir\'a m\'as adelante, no la pierdas de vista.

Por tanto, Ex 20,5c \emph{refuerza} la afirmaci\'on de que \textbf{Ex 20,4-5b forma parte de un \'unico mandamiento con Ex 20,3}.

Antes de analizar Ex 20,3 digamos alguna palabra sobre Ex 20,5d-6.

\paragraph{Las consecuencias de la desobediencia y la obediencia}

En Ex 20,5d-6 NC leemos: \emph{<<que castiga en los hijos las iniquidades de los padres hasta la tercera y cuarta generaci\'on de los que me odian y hago misericordia hasta mil generaciones de los que me aman y guardan mis mandamientos>>}.

Rescato de aqu\'{i} dos palabras, que reflejan dos actitudes contrapuestas: \emph{odiar} y \emph{amar}. Abandonar a Dios e ir en pos de otros dioses equivale a \emph{odiar} a Dios. Serle fiel como una esposa, \emph{amarlo}, se ve reflejado en \emph{guardar sus mandamientos}.

Este texto tiene la estructura de un \emph{pacto} entre Dios y su pueblo: Dios pide a Israel la \emph{fidelidad} que un esposo pide a la esposa. Si el hombre es fiel a los mandamientos de Dios, recibir\'a \emph{misericordia} por mil generaciones. Si el hombre prostituye su fe y se va en pos de otros dioses, ser\'a \emph{castigado} hasta la tercera y cuarta generaci\'on.

Esta idea del \emph{pacto} ser\'a importante para lo que viene a continuaci\'on.

\subsubsection{Incorporando Ex 20,3 al an\'alisis}

No \'unicamente Ex 20,5c-6 refuerza la idea de que Ex 20,4-5b se refiere a la adoraci\'on de otros dioses. Tambi\'en Ex 20,3 lo sugiere. El texto sigue una estructura circular en la que:

\begin{enumerate}
\item Dios recuerda al pueblo qui\'en los a rescatado de la esclavitud (Ex 20,2)
\item Pide que no tengan otros dioses (Ex 20,3)
\item lo ilustra de manera gr\'afica (Ex 20,4-5b)
\item indica el \emph{motivo} (Ex 20,5c), el cual confirma lo dicho en el v3
\item cierra con las consecuencias (Ex 20,5d-6)
\end{enumerate}

\noindent
Si t\'u lo separas, rompes su estructura interna: la de un \emph{pacto}\footnote{Una estructura similar aunque m\'as sencilla se da en Ex 15,23-26 y en Ex 19,1-6.} en la que Dios escoge a Israel como propiedad suya y le pide a cambio fidelidad esponsal advirti\'endole las consecuencias de apartarse de su camino. Dentro del mismo pacto aparece el resto de los mandamientos. Se sigue una estructura similar en el v7, otra en los vv8-11 y otra en el v12. Aunque \'estas no re\'unen todos los elementos. El v2 sirve simult\'aneamente como apertura para Ex 20,2-6 y para Ex 20,2-17.

Por tanto, la estructura gramatical y sem\'antica de Ex 20,2-17 tambi\'en refuerza la idea de que \textbf{el primer mandamiento va desde Ex 20,2 hasta Ex 20,6} y que lo que proh\'{i}be es tener \emph{otros dioses}; aqu\'{i} el papel de Ex 20,4-5b es \emph{mostrar el tipo de pr\'acticas que hacen aquellos que adoran otros dioses} m\'as que prohibir hacer im\'agenes, incluso con \emph{prop\'osito religoso}.

\subsection{?`La Iglesia Cat\'olica \emph{escondi\'o} el segundo mandamiento?}

Llegado es el tiempo de resolver la cuesti\'on de si la Iglesia Cat\'olica, en su af\'an de promover la idolatr\'{i}a \emph{escondi\'o} el segundo mandamiento que proh\'{i}be las im\'agenes o bien si los protestantes \emph{dividieron} el primer mandamiento para fundamentar su separaci\'on de Roma.

El Catecismo de la Iglesia Cat\'olica lo resume en el n\'umero 2066 de la siguiente manera:

\begin{quote}
\emph{La división y numeración de los mandamientos ha variado en el curso de la historia}. El presente catecismo sigue la división de los mandamientos establecida por san Agustín y que ha llegado a ser tradicional en la Iglesia católica. Es también la de las confesiones luteranas. Los Padres griegos hicieron una división algo distinta que se usa en las Iglesias ortodoxas y las comunidades reformadas.\footnote{http://www.vatican.va/archive/catechism\_sp/p3s2\_sp.html}
\end{quote}

\noindent
El an\'alisis con la pregunta \emph{<<Seg\'un la Biblia ?`Cu\'antos son los mandamientos del dec\'alogo?>>}; posteriormente revisaremos tres formas de numeraci\'on que han existido y sus principales caracter\'{i}sticas; entonces podremos responder a la pregunta acerca de si la Iglesia Cat\'olica \emph{escondi\'o} el segundo mandamiento o bien los protestantes \emph{dividieron} el primero.

\subsubsection{?`Cu\'antos son los mandamientos?}

La palabra \emph{dec\'alogo} proviene del griego \emph{$o \iota$ $ \delta\varepsilon\kappa\alpha$ $\lambda o \gamma o \iota$} el cual es utilizado en la versi\'on de los LXX para traducir \emph{{\lq}aseret haddebarim} en Dt 10,4 y significa <<diez palabras>>.\footnote{http://www.mercaba.org/DicTB/D/decalogo.htm}

Seg\'un Dt 5,22 fueron escritos en dos tablas de piedra. Pero hay dos cosas que la Biblia no precisa: a) de qu\'e forma se \emph{distribuyeron} las palabras en las tablas y b) la numeraci\'on exacta de las palabras.

Algunos discuten si las diez palabras se distribuyeron entre las dos tablas o bien si cada una de las tablas conten\'{i}a todo el dec\'alogo repetido como si fuera un contrato con dos copias: una para Dios y otra para el pueblo.\footnote{Estos temas son bastante extensos y muchos aspectos son controvertibles; as\'{i} que me centrar\'e en lo esencial para nuestra discusi\'on.}

\subsubsection{Distintas maneras de hacer la numeraci\'on de los mandamientos}

Sin adentrarme en esas cuestiones, voy a se\~nalar cuatro tradiciones interpretativas antiguas que corresponden con tres maneras de hacer la numeraci\'on:\footnote{En esta liga puedes encontrar un cuadro comparativo: http://es.wikipedia.org/wiki/Diez\_Mandamientos; sin embargo, el cuadro requiere algunas precisiones de las tradiciones que lista: a) Donde dice \emph{septuaginta} deber\'{i}a decir \emph{Or\'{i}genes de Alejandr\'{i}a}; y b) Donde dice \emph{Catecismo de la Iglesia Cat\'olica} es redundante, pues el mismo catecismo indica que est\'a siguiendo a San Agust\'{i}n.}

\begin{enumerate}
\item El Talmud\footnote{http://www.tora.org.ar/contenido.asp?idcontenido=856}
\item Fil\'on de Alejandr\'{i}a\footnote{http://dadun.unav.edu/bitstream/10171/13235/1/ST\_XXIX-2\_03.pdf}
\item Or\'{i}genes de Alejandr\'{i}a\footnote{http://dadun.unav.edu/bitstream/10171/13256/1/ST\_XXX-1\_03.pdf}
\item San Agust\'{i}n de Hipona\footnote{http://www.augustinus.it/spagnolo/ y http://dialnet.unirioja.es/descarga/articulo/233589.pdf}
\end{enumerate}

\noindent
Te recomiendo ampliamente que revises las fuentes que estoy dejando en los pies de p\'agina, en especial las que se refieren a Fil\'on, Or\'{i}genes y Agust\'{i}n. Valen mucho la pena m\'as all\'a del debate que estamos sosteniendo nosotros.

\paragraph{El Talmud}

afirma que el primer mandamiento es Ex 20,3 y el segundo mandamiento es Ex 20,4-6; mientras que el d\'ecimo mandamiento es Ex 20,17 colocando a la mujer entre las \emph{posesiones} del pr\'ojimo. Esta tradici\'on es importante, porque muy probablemente fue la que estaba vigente en tiempo de Jes\'us en Jerusal\'en.

\paragraph{Fil\'on de Alejandr\'{i}a  (13 a.C.-50 d.C.)}

se\~nala que el primer mandamiento es Ex 20,3; el segundo es Ex 20,4-6 y el d\'ecimo es Ex 20,17. La obra de Fil\'on intenta armonizar su fe jud\'{i}a con el pensamiento griego. Armonizar la revelaci\'on de Dios con el conocimiento filos\'ofico. Sin embargo se mantiene dentro de los horizontes del Antiguo Testamento.

Fil\'on de Alejandr\'{i}a es importante por dos razones: a) conform\'o la tradici\'on de los jud\'{i}os de habla griega, distanci\'andose del juda\'{i}smo tradicional y b) Tendr\'a una influencia \emph{decisiva} en Or\'{i}genes.

\paragraph{Or\'{i}genes de Alejandr\'{i}a (185-254 d.C.)}

recibi\'o una influencia considerable de Fil\'{o}n de Alejandr\'{i}a tanto en el esfuerzo de armonizar el pensamiento filos\'{o}fico con el religioso; como en la \emph{numeraci\'on} de los mandamientos.

Sin embargo, entre Fil\'on y Or\'{i}genes hay \emph{dos siglos} de cristianismo. El \emph{contenido} de los comentarios de Or\'{i}genes al dec\'alogo tiene toda la impronta cristiana.

Or\'{i}genes sigue al pie de la letra a Fil\'on en la manera de dividir los mandamientos; sin embargo, como dato importante destaca la \emph{justificaci\'on} que da Or\'{i}genes a dividir en dos mandamientos Ex 20,3 y Ex 20,4-6:

\begin{quote}
<<La razón que da Orígenes a este desdoblamiento es que aunque
algunos piensan que todas estas cosas forman un sólo mandamiento, pero
entonces no se llegaría al número diez y entonces ¿dónde estaría la verdad
del término decálogo? Por tanto, hay que desglosarlo en dos mandamientos
diferentes y explicó cada uno de ellos>>.\\
(LLUCH Baixauli, Miguel; \emph{La Interpretación de Orígenes al Decálogo})\footnote{http://dadun.unav.edu/bitstream/10171/13256/1/ST\_XXX-1\_03.pdf}
\end{quote}

\noindent
Llama la atenci\'on esta \emph{justificaci\'on} porque \emph{no alude ning\'un motivo teol\'ogico} sino m\'as bien el que sean \emph{diez} los mandamientos. Este dato nos revela dos cosas: a) que Or\'{i}genes no ve objeci\'on \emph{teol\'ogica} a que Ex 20,3 forme \emph{un} mandamiento con Ex 20,4-6 y b) que en el tema de Ex 20,17 se mantiene en la misma l\'{i}nea que el juda\'{i}smo.

La influencia de Or\'{i}genes en el pensamiento cristiano es enorme. Pero para el tema que nos ocupa diremos que es importante porque \emph{las iglesias de Oriente} siguen a Or\'{i}genes en la numeraci\'on de los mandamientos.

\emph{A\'un as\'{i}, las iglesias de Oriente no interpretan Ex 20,4-6 en el sentido que lo hacen los protestantes; oriente es la cuna de los \'{i}conos cristianos y ellos no tuvieron ning\'un conflicto con eso}. 

\emph{Siglos m\'as tarde, fue en la ciudad griega (oriental) de Nicea donde en el a\~no 787 en concilio ecum\'enico se proclam\'o la legitimidad de las im\'agenes religiosas as\'{i} como la manera correcta de usarlas; todo esto frente a la herej\'{i}a iconoclasta, la cual surgi\'o, entre otras razones, por influencia del Islam}.

\subsubsection{San Agust\'{i}n de Hipona (354-430 d.C.)}

Es el m\'as grande pensador cristiano de la antig\"uedad. Para el tema que nos ocupa cito a Miguel Lluch que dice:

\begin{quote}
Será San Agustín el que unirá los dos primeros origenianos
y distinguirá los dos últimos. Para San Agustín es distinto desear
la mujer del prójimo que el resto de los bienes del prójimo y así, entenderá
dos prohibiciones distintas en el último mandamiento y reunirá en uno
sólo el primero. Esta enumeración agustiniana se convertirá en la propia
de la tradición católica. En el siglo XVI Lutero la mantuvo también, \emph{pero
Calvino se separó de ella y volvió a la tradición flloniana}.\footnote{\emph{\'{I}dem.}, el subrayado es m\'{i}o.}
\end{quote}

\noindent
Antes de responder a la pregunta \emph{<<?`es leg\'{i}tima la divisi\'on que hace San Agust\'{i}n de Ex 20,17?>>} quiero resaltar un hecho: los \emph{motivos} que da el santo de Hipona para hacerlo, \emph{no tienen nada que ver} con \emph{esconder} un mandamiento para practicar la \emph{idolatr\'{i}a}.

\paragraph{?`Es leg\'{i}tima la divisi\'on que hace San Agust\'{i}n de Ex 20,17?}

Para comenzar, San Agust\'{i}n est\'a siguiendo a Dt 5,21 donde \emph{s\'{i}} est\'an separados, la \emph{mujer} y los \emph{bienes} del pr\'ojimo.

Los motivos que da el santo de Hipona para hacer as\'{i} la divisi\'on de Dt 5,21 en dos (y unir Ex 20,3-6 en uno solo) son de \'{i}ndole teol\'ogica, entre ellos, podemos se\~nalar dos:\footnote{Revisa http://www.augustinus.it/spagnolo/discorsi/discorso\_009\_testo.htm y \\ http://www.augustinus.it/spagnolo/discorsi/discorso\_010\_testo.htm}

\begin{description}
\item[Para unir Ex 20,3-6] \emph{La confesi\'on de la fe trinitaria} manifestada en tres mandamientos: Ex 20,3-6; Ex 20,7 y Ex 20,8-11. Pues cada uno de \'estos mandamientos nos revela algo sobre cada una de las tres divinas personas.
\item[Para dividir Dt 5,21] El deseo de la mujer es de naturaleza \emph{enteramente distinta} al deseo de los bienes; y as\'i como al mandamiento que proh\'{i}be el robo (Dt 5,19), le corresponde un mandamiento que proh\'{i}be desear lo ageno (Dt 5, 21b), de la misma forma al mandamiento que proh\'{i}be el adulterio (Dt 5,18) le corresonde aquel que proh\'{i}be desear la mujer del pr\'ojimo (Dt 5,18).
\end{description}

\noindent
Como puedes ver, la Iglesia Cat\'olica \emph{no escondi\'o} ning\'un mandamiento para justificar la idolatr\'{i}a; las cosas siguieron otro curso. Preg\'untate ahora si no fue Juan Calvino quien, para justificar sus doctrinas, busc\'o apoyo en tradiciones no cristianas, como la de Fil\'on de Alejandr\'{i}a.

En cualquier caso, m\'as que seguir una numeraci\'on u otra (la Iglesia Cat\'olica no lo considera un dogma), lo importante es atender al \emph{contenido y significado}\ldots despu\'es de un breve par\'entesis profundizaremos un poco m\'as al respecto.

\subsubsection{La Iglesia Cat\'olica NO \emph{escondi\'o} el segundo mandamiento}
La controversia sobre la numeraci\'on de los mandamientos nos lleva a la conclusi\'on de que lo importante no es saber con certeza \emph{a qu\'e n\'umero de mandamiento} pertenece lo mandado en Ex 20,4-6; sino al hecho de que lo prohibido por el mandamiento es la adoraci\'on de \emph{otros dioses}.

Ex 20,4-6 es, en realidad, una \emph{ampliaci\'on} de Ex 20,3 y no se le puede entender separadamente, aunque se le numere por separado.

Por otra parte, puede ser que a estas alturas todav\'{i}a sientas demasiado forzado el argumento, pues claramente Dt 5,18 es \emph{un} solo vers\'{i}culo.

Sin embargo, como buen conocedor de la Biblia que eres, no te ser\'a desconocido, que \emph{la divisi\'on en cap\'{i}tulos y vers\'{i}culos no forma parte de la revelaci\'on}.

Esta inserci\'on fue hecha durante la Edad Media por los \emph{te\'ologos cat\'olicos} para facilitar el estudio de la Sagrada Escritura. As\'{i} que apoyarse en la numeraci\'on de cap\'{i}tulos y vers\'{i}culos para hacer fundamentaciones teol\'ogicas es err\'oneo; es algo que ya sabes, pero hay que tener cuidado.

\section{Legitimidad de la pr\'actica cat\'olica de tener im\'agenes}

La prohibici\'on de las im\'agenes en Ex 20,4-6 no es absoluta, esta ligada al \emph{uso} de dichas im\'agenes el cual est\'a prohibido cuando se trata de adorar a otros dioses, como se indica en Ex 20,3.

Que \'esta sea la interpretaci\'on correcta se deduce de lo que hemos discutido anteriormente. As\'{i} lo han cre\'{i}do los cristianos de todos los tiempos. \'Unicamente los herejes iconoclastas de los siglos VII-VIII y los protestantes en el siglo XVI en adelante, basados en interpretaciones privadas de la escritura (2 Pe 1, 20) han considerado las im\'agenes de Jes\'us y los bienaventurados como idol\'atricas.

Dado que a) el \emph{argumento de excepci\'on} carece de validez b\'{i}blica y b) el mandamiento de Ex 20,3-6 hace una prohibici\'on \emph{relativa} a la adoraci\'on de otros dioses y no absoluta como pretende el protestantismo iconoclasta, podemos hacer la siguiente afirmaci\'on:

\begin{quote}
\emph{Existen usos leg\'itimos de im\'agenes con prop\'osito religioso. Los ejemplos del Arca, el Templo y la serpiente de bronce constituyen gu\'{i}as de c\'omo hacerlo.}
\end{quote}

\noindent
Ya has dicho \emph{los \'unicos casos v\'alidos que encontramos son cuando Dios expl\'{i}citamente ordena la elaboraci\'on de las im\'agenes}, sin embargo tal afirmaci\'on es falsa, pues encontramos en la escritura acciones no ordenadas por Dios y no reprobadas y por otro lado encontramos acciones reprobadas sobre objetos que \'El mand\'o construir.

\begin{itemize}
\item \emph{Acciones no solicitadas sobre objetos solicitados y que son aprobadas} as\'{i} por ejemplo cuando se postran delante del Arca (Jos 7,6)

\item \emph{Acciones no solicitadas sobre objetos no solicitados y que son aprobadas} por ejemplo cuando Eliseo recibi\'o el manto de El\'{i}as y con \'el separ\'o las aguas del r\'{i}o (2 Re 2,9-14) o cuando Naam\'an pide llevarse tierra de Israel en se\~nal de que s\'olo adorar\'a a Yahveh (2 Re 5,17); pero tambi\'en en el N.T. encontramos algunos ejemplos: una mujer es curada al tocar el manto de Jes\'us (Mc 5,27-29); pon\'{i}an a los enfermos a que la sombra de Pedro pasara sobre ellos y as\'{i} se curaran (Hch 5,15-16)\footnote{Respecto a lo que dec\'{i}as de que la Biblia dice que los colocaban mas no que se curaran, estas considerando que la segunda parte del v16 unicamente aplica a la primera parte y no al v15, es una asociaci\'on arbitraria pues rompes el sentido del pasaje; por otra parte, como ocurre en otras ocasiones, Pedro pudo muy bien aclararles que lo que hac\'{i}an era incorrecto pero no lo hizo; revisa: Hch 14,8-18.}; o los pa\~nos y delantales de Pablo que curaban enfermos (Hch 19,11-12).\footnote{Estos ejemplos son figuras y anticipos de la pr\'actica cat\'olica de la veneraci\'on de las reliquias, que en algunos casos Dios obra milagros por medio de ellas y en otros no, revisa por ejemplo: 2 Re 13,21.}

\item \emph{Acciones no solicitadas sobre objetos s\'{i} solicitados pero que son condenadas} Como cuando el pueblo ador\'o la serpiente de bronce de Mois\'es y por eso, Ezequ\'{i}as (sin mandato expl\'{i}cito de Dios) la destruy\'o\footnote{T\'u puedes decirme: \emph{ah\'{i} no dice que la adoraran sino que le <<quemaron incienso>>}. A lo cual yo te respondo: 1) el pasaje muestra que la serpiente de bronce era el objeto destinatario del incienso, y el incienso en el A.T. es ofrecido \'unicamente a Dios como signo de adoraci\'on (revisa Lev 1-7); y 2) tu literalismo material y estrecho te impide entender las Escrituras y el poder de Dios (Mt 22,29), y si a ese literalismo te apegas pues debes saber que \emph{ni} en Ex 20,3-6 \emph{ni} en Dt 5,5-10 se menciona la palabra \emph{incienso} as\'{i} que seg\'un tu criterio, <<quemar incienso>> a una imagen no est\'a prohibido, entonces ?`con qu\'e autoridad Ezequ\'{i}as destruyo un objeto mandado hacer por Dios?}
\end{itemize}

El criterio no es que \emph{Dios haya ordenado hacer esos objetos} (argumento de excepci\'on) sino el \emph{uso} que se les da: si son instrumentos para la adoraci\'on a Dios o por el contrario son adorados en lugar del Creador.

Pudi\'eramos pensar, a partir de lo visto, que hipot\'eticamente existiera alguna forma de tener im\'agenes con prop\'osito religioso y que sea conforme al mandato de Dios. Pero que \emph{definitivamente no es la pr\'actica cat\'olica, la cual se asemeja m\'as al paganismo que ninguna otra y a\'un es peor, al pretender justificarse torciendo la Escritura}; por ejemplo: el hecho de que se le ofrezca incienso a las im\'agenes cat\'olicas, lo cual es una se\~nal inequ\'ivoca de adoraci\'on como yo mismo acabo de decir.

\subsection{?`Es leg\'{i}timo el uso que hacen los cat\'olicos de las im\'agenes religiosas?}

?`La Biblia permite hacer representaciones de Yahveh en el Antiguo Testamento? ?`Ha cambiado algo con la Nueva Alianza que permita o autorice la elaboraci\'on de im\'agenes? ?`En qu\'e consiste la \emph{veneraci\'on} cat\'olica de las im\'agenes?

Despu\'es de responder estas preguntas podremos determinar si la pr\'actica cat\'olica sobre las im\'agenes es conforme con el mandato de Dios o se le opone.

\subsection{Imposibilidad en el Antiguo Testamento de representar a Yahveh}

?`La prohibici\'on de hacer im\'agenes abarca \'unicamente a los \'{i}dolos o tambi\'en proh\'{i}be \emph{hacer representaciones de Yahveh}?

% El libro de los jueces (Jue 17,1-13;18,18-31) menciona la efigie de Mik\'a y la menciona como una representaci\'on de Yahveh; pero m\'as adelante al mencionar c\'omo los danitas se llevan la imagen, Mik\'a les reclama que <<se llevaron a su \emph{dios}>> (Jue 18,24) con lo cual pone de manifiesto que estaba haciendo de esa imagen algo muy cercano a un \'{i}dolo.

% Tomo este ejemplo para abrir el tema sobre representar a Yahveh en el Antiguo Testamento. La respuesta es negativa.

% As\'{i} que en Ex 20,3-6 tenemos dos prohibiciones superpuestas: a) adorar otros dioes (con todo lo que implica hacerles im\'agenes y reverenciarlas, etc) y b) hacer im\'agenes de Yahveh. Sin embargo los motivos son enteramente diferentes, como enteramente diferentes son el Dios verdadero y los falsos dioses.

Cito a continuaci\'on a G. Barbaglio que en el Nuevo Diccionario de Teolog\'{i}a moral (Ed. Paulinas, 1992) dice lo siguiente al comentar sobre el dec\'alogo en el libro del Deuteronomio:\footnote{http://www.mercaba.org/DicTM/TM\_decalogo.htm el \emph{subrayado} es m\'{i}o.}

\begin{quote}
1) El segundo mandamiento: <<No te harás ídolos ni imagen alguna>> (cf ZIÑihÍER1.1, Das Zweite Gebot; G. vorr RAD, 246-254), prohibe representar en estatuas a la divinidad. \emph{Se refería originariamente a representaciones de Yhwh}. El precepto contrastaba con la costumbre de los pueblos vecinos, que consideraban la estatua como el medio del encuentro con Dios y con su revelación. El sentido del mandamiento no es el de salvaguardar la espiritualidad de Yhwh, preocupación ésta ausente en Israel y ajena a la linea del significado que revestía la estatua en los ambientes circundantes. \emph{Se quería, con ello, proteger la libertad de Yhwh, que no es un Dios que el hombre pueda aferrar ni [est\'a]\footnote{Esta palabra no est\'a en el original pero me parece que es la m\'as adecuada para que el enunciado tenga sentido.} sometido a la limitación de sus fieles. Por medio de la estatua, en la que se consideraba presente a la divinidad, se pretendía dominarla para someterla a los propios deseos}. Medio exclusivo de revelación de Yhwh al pueblo y ámbito único de encuentro es su palabra y su acción en la historia.

2) La posterior ampliación deuteronomista del mandamiento: <<No te postrarás ante ellos y no les servirás>>, \emph{constituye claramente una repetición del tema del primer mandamiento}, en cuanto que en él ya estaban prohibidos la adoración y el culto dedicados a los ídolos. Lo que significa que para la interpretación deuteronomista posterior, \emph{la prohibición de hacer estatuas, originariamente aplicada a las representaciones de Yhwh, se refiere a los ídolos de los dioses extranjeros y a su culto}. Y por lo tanto, según esta interpretación, \emph{la prohibición de las imágenes no constituye ya un segundo mandamiento distinto del primero, sino más bien la continuación y el desarrollo de este último}. Tendríamos así una numeración distinta de los diez mandamientos según la actualización deuteronomista, que, como veremos más adelante, para restablecer el número de diez desdoblará el último mandamiento.\footnote{Revisa la url para ver c\'omo resuelven ellos la \emph{divisi\'on} del \'ultimo mandamiento en dos.}

Sigue después la motivación del exclusivo reconocimiento de Yhwh: <<Porque yo, Yhwh, soy tu Dios, un Dios celoso, que castiga la culpa de los padres en los hijos hasta la tercera y cuarta generación para los que me odian pero que demuestra su favor en mil generaciones con quienes me aman y observan mis mandamientos>>. Yhwh es un Dios celoso, \emph{atento con todas sus fuerzas y su energía a afirmar su derecho frente al pueblo y a no tolerar a ningún otro como Dios de aquéllos a quienes él ha liberado de Egipto}, dispuesto a castigar la culpa de la infidelidad, pero infinitamente benévolo con quienes lo aman y le son fieles.
\end{quote}

\noindent
Voy a enumerar las ideas principales:

\begin{enumerate}
\item Hay una prohibici\'on de hacer representaciones de Yahveh
\item \textbf{El \emph{motivo} es para salvaguardar la \emph{libertad} de Dios, que no est\'a sometido (por medio de im\'agenes) al hombre. M\'as exactamente, que al pueblo le quede muy claro que a Dios no se le puede controlar.}
\item Sin excluir esta idea, el mismo mandamiento es un desarrollo de Dt 5,7
\item Lo cual queda reforzado por los \emph{celos} de Dios
\end{enumerate}

\noindent
He resaltado la idea que considero m\'as importante: \emph{el motivo para prohibir la representaci\'on de Dios es quitarle la tentaci\'on al hombre de querer <<controlar>> a Dios}.
%Esto queda confirmado por lo que pas\'o con la efigie de Mik\'a (Jue 17,1-13;18,18-31), pues terminaron \emph{reduciendo} al Dios alt\'{i}simo a un <<dios>> atrapado en una imagen.

% Esta idea se ve reforzada por el cap\'{i}tulo 4 del Deuteronomio, donde se le recuerda al pueblo que cuando Dios habl\'o en el Horeb enmedio del fuego, el pueblo de Israel no vio figura alguna (Dt 4,15) y si intentan hacer alguna representaci\'on de Dios (a quien no vieron) entonces terminar\'an adorando a lo que no es Dios (Dt 4,16-19).

% De aqu\'{i} que sea \emph{impensable} en el A.T. tener representaciones de Dios. No se puede representar lo que no se ve (\'El es \emph{trascendente}) y no se le puede controlar mediante una imagen (\'El es \emph{libre}); y aquellos que intentan representarlo y/o controlarlo terminan, en su imaginaci\'on, reduciendo al Dios alt\'{i}simo a un \'{i}dolo.

\subsubsection{Salvaguardar la libertad de Yahveh}

\emph{Someter} a Dios y ponerlo a nuestro servicio es una tentaci\'on permanente. Con o sin im\'agenes. Puede ser que \emph{exteriormente} se est\'e reverenciando, al Dios alt\'{i}simo, pero \emph{interiormente} se quiera \emph{forzar} a Dios a cumplir los propios deseos.

Pongo tres casos a tu consideraci\'on:

\begin{enumerate}
\item La efigie de Mik\'a (Jue 17,1-13;18,18-31)\footnote{En la versi\'on Reina-Valera 1960 menciona que eran muchos dioses y no uno solo, mientras que en la N\'acar Colunga 1944 menciona que es uno s\'olo; no he podido validar las traducciones contra el original en hebreo -- entre otras cosas porque no se hebreo -- pero en caso de que la forma correcta sea la de RVR entonces la argumentaci\'on mantiene su curso pero sin el ejemplo de Jue 17-18.}
\item La segunda tentaci\'on de Jes\'us (Mt 4,5-7)
\item El fariseo y el publicano (Lc 18,9-14)
\end{enumerate}

\paragraph{La efigie de Mik\'a (Jue 17,1-13;18,18-31)}

Mik\'a tiene una idea \emph{reducida} de Dios, cree que lo puede \emph{encerrar} en una imagen y que lo favorezca (Jue 17,13); cuando le quitan la imagen lo que reclama es que le quitan a su dios, donde <<su>> significa de su propiedad: <<mi dios, \emph{el que yo he hecho}>> (\emph{c.f.} Jue 18, 24).

El problema no es tan simple como parece a primera vista: no se trata, simplemente, de que haga una \emph{representaci\'on} f\'{i}sica de Dios y por este acto su coraz\'on quede corrompido y su mente quede confundida; se trata de que al querer representar a Yahveh, \emph{?`Con qu\'e figura lo va a hacer si nunca lo ha visto?} (Dt 4,15). Cualquier forma que haya elegido manifiesta que ha <<moldeado>> a Dios seg\'un su propia idea.

\paragraph{La segunda tentaci\'on de Jes\'us (Mt 4,5-7)}

M\'as all\'a de la representaci\'on f\'{i}sica, el problema va en el sentido de si el Dios \emph{real} se acomoda a \emph{mis} deseos y conceptos o si yo busco obedecerle \emph{tal cual es}.

Satan\'as reta a Jes\'us a que se lance desde lo alto del tempo y ponga a prueba la fidelidad de Dios, para esto cita el salmo 91 (90); si Dios realmente es quien dice ser que lo cumpla y lo pongo a prueba \emph{a ver si me convence}.

Por eso, la respuesta de Jes\'us es completamente exacta (Mt 4,7 \emph{c.f.} Dt 6,16).

\paragraph{El fariseo y el publicano (Lc 18,9-14)}

El problema va m\'as all\'a. Tambi\'en significa pretender \emph{obligar} a Dios a cumplir \emph{su parte del trato}; los fariseos se sienten \emph{merecedores} del favor divino por \emph{cumplir la ley}; los profetas, especialmente Isa\'{i}as y Jerem\'{i}as criticaron esa actitud del que cumple con los preceptos de pureza ritual pero descuidan la justicia.

Finalmente el fariseo que sube al templo a orar, seg\'un nos cuenta Lucas, le agradece a Dios no ser \emph{como los dem\'as hombres}, los decribe incluyendo a <<\emph{ese publicano}>> y despu\'es enumera sus \emph{m\'eritos} ante Dios. No lo menciona pero el silencio da muy bien a entender que interiormente lo que le est\'a diciendo a Dios es algo como lo siguiente: <<\emph{Yo si cumplo con la ley, y t\'u ?`qu\'e me vas a dar a cambio?}>>.

\paragraph{Resumiendo}

Es \emph{impensable} en el A.T. tener representaciones de Dios. No se puede representar lo que no se ve (\'El es \emph{trascendente}) y no se le puede controlar mediante una imagen (\'El es \emph{libre}); y aquellos que intentan representarlo y/o controlarlo terminan, en su imaginaci\'on, reduciendo al Dios alt\'{i}simo a un \'{i}dolo.

En Ex 20,3-6 y Dt 5,7-10 hay dos prohibiciones superpuestas: a) adorar otros dioses y b) pretender \emph{controlar} a Yahveh. Ambas prohibiciones tienen sus \emph{manifestaciones externas}: hacerse im\'agenes, poner a Dios a prueba, imponerle a Dios las propias condiciones\ldots Pero el \emph{acto interior} es el que da la pauta moral y no la \emph{manifestaci\'on externa}.

\subsection{Una comparaci\'on \'util}

Voy a utilizar una comparaci\'on que nos va a ayudar a entender lo que estamos analizando:

\begin{quote}
\emph{Una mujer casada con un hombre bueno tuvo un accidente mientras su esposo estaba de viaje, como resultado de este accidente ella perdi\'o parcialmente la memoria y era incapaz de recordar el aspecto f\'{i}sico de su esposo y por el mismo accidente ella perdi\'o todas las fotograf\'{i}as que ten\'{i}a de \'el.}

\emph{Lo amaba pero no recordaba su rostro\ldots con el tiempo el esposo le enviaba cartas de amor para consolarla y le promet\'{i}a que volver\'{i}a pronto\ldots}
\end{quote}

\noindent
Dado que ella es incapaz de recordar a su esposo\ldots ?`qu\'e suceder\'{i}a si ella busca una foto de alg\'un hombre para recordar a su esposo? pues lo mas probable es que a) no se le parezca y b) que ella se enamore del hombre de la foto y no de su esposo.

Por esa raz\'on el marido en una de sus cartas le advierte que no trate de tener fotograf\'{i}as de \'el no vaya a ser que se enamore de otro hombre\ldots

\subsection{?`Qu\'e ha pasado de la Antigua a la Nueva Alianza?}
?`Qu\'e ha cambiado entre el monte Sina\'{i} y el monte Calvario?

Lo m\'as hermoso que jam\'as haya escuchado ser humano alguno: el Verbo eterno de Dios que tambi\'en es Dios se hizo carne y puso su morada entre nosotros (Jn 1,14)\ldots El esposo vino a visitar a la esposa, sigamos con el ejemplo que tra\'{i}amos:

\begin{quote}
\emph{La mujer sabe que por su accidente ella es incapaz de recordar el rostro de su marido, as\'{i} que ella decide no tener fotograf\'{i}as de \'el para poderlo reconocer cuando llegue.}

\emph{Tiempo despu\'es el marido regresa de viaje y ella es capaz de reconocerlo precisamente por las cartas y por no formarse una imagen f\'{i}sica de un rostro ausente en su mente; pero ahora el esposo es visible y ella sabe que es el esposo y ella lo puede reconocer.}

\emph{A\~nos m\'as tarde muere el marido y la mujer tiene que continuar algunos a\~nos recordando a su marido antes de ir con \'el a la morada celestial\ldots}
\end{quote}

\noindent
El marido est\'a ausente, pero ella puede recordar su rostro ?`Tiene sentido mantener la prevenci\'on de no tener fotograf\'{i}as siendo que ahora tendr\'{i}a \'unicamente aquellas que son de su esposo? ?`No le ayudar\'{i}an ahora las fotograf\'{i}as \emph{de su marido} a recordarlo y serle fiel?

Ahora lo ha visto y es perfectamente capaz de reconocer la foto del hombre al que ama y no confundirlo con la foto de cualquier otro hombre. Y esta foto le permite recordarlo, junto con sus cartas y muchas otras cosas y tener \'animo para ir con \'el.

Pero tu pudieras objetarme que a) \emph{dado que la enfermedad persiste, cuando se fue el marido ella lo volvi\'o a olvidar y persistir\'{i}a la prevenci\'on para serle fiel} y b) por tal caso \emph{el esposo le sigue recomendando por carta que no tenga fotograf\'{i}as de aquel rostro que ha olvidado de nuevo}, revisemos este escenario\ldots

\subsubsection{?`El esposo sigue prohibiendo a la esposa tener fotograf\'{i}as?}

Sabes que en el Nuevo Testamento no hay ning\'un refrendo expl\'{i}cito de Ex 20,4-6 y que los casos que hay siempre estan asociados a \emph{otros dioses}.

T\'u podr\'as decirme: \emph{pero no hay NINGUN texto que indique que se haya revocado Ex 20,4-6 (que revise Mt 5,18)}\ldots

Yo te respondo que \emph{s\'{i} lo hay pero no lo puedes ver}: en Dt 4,12.15 viene expl\'{i}citamente el \emph{motivo} de la prohibici\'on de las im\'agenes. En Dt 4,16-19 habla de qu\'e sucede cuando hacen una imagen si no han visto a Dios ?`Por qu\'e? porque la imagen que sea \emph{corresponder\'a con una creatura y no con Dios} y suceder\'a lo que ven\'{i}amos platicando en el ejemplo que traemos; por lo tanto esas advertencias estan \emph{subordinadas} a que el pueblo escogido haya visto o no haya visto una figura en el Horeb.

\textbf{La encarnaci\'on del Verbo invalida Dt 4,12.15 y hace que no haya peligro con Dt 4,16-19.}

\subsubsection{?`La esposa sigue perdiendo la memoria?}

No te es extra\~no que la analog\'{i}a que estoy utilizando identifica a la mujer que tuvo el accidente con el pueblo de Dios, tanto del Antiguo como del Nuevo Testamento, el esposo es Dios y m\'as espec\'{i}ficamente Jesucristo y la muerte del esposo en el relato representa la ascenci\'on de Jes\'us con su Padre\ldots

Ahora bien, \emph{es absurdo que los disc\'{i}pulos hayan olvidado el rostro de Jes\'us}; tan absurdo como decir que la redenci\'on tuvo un efecto moment\'aneo y hemos vuelto a la situaci\'on de antes. Revisa Ef 5,25-27; Col 1,18; Mt 28,20; Jn 16,13 y 1 Tim 3,15.

Si afirmas que la esposa ha olvidado al esposo est\'as negando los pasajes de la Escritura arriba mencionados.

Podr\'as decirme que \emph{no hay fotograf\'{i}as de Jes\'us} y que por tanto \emph{cualquier imagen suya es una representaci\'on que no corresponde con Jes\'us y por tanto sigue siendo vigente Dt 4,15-19}\ldots 

Yo te respondo que: a) Jes\'us mismo es la imagen visible de Dios invisible (Col 1,15) que quien lo ve, ve al Padre (Jn 14,9) y b) aunque la imagen no corresponda al aspecto f\'{i}sico de Jes\'us, s\'{i} corresponde con la verdad que encierra: \emph{que el Hijo de Dios se hizo Hijo del Hombre} y la imagen claramente hace referencia a ese misterio y a la persona del Cristo.

\textbf{Realizar im\'agenes de Jes\'us es confesar que su Encarnaci\'on y la obra redentora que realiz\'o es permanente, negar las im\'agenes en la Nueva Alianza es negar la Encarnaci\'on y la Redenci\'on.}

Una vez que qued\'o claro que los \emph{\'{i}dolos no son nada}\ldots las comunidades cristianas no tuvieron problema en aprovechar elementos visuales para fomentar la verdadera adoraci\'on de Dios

\subsection{La intenci\'on remite al original}

\emph{La Iglesia Cat\'olica quema incienso a las im\'agenes} y como he dicho m\'as arriba \emph{es una sen\~nal inequ\'ivoca de adoraci\'on idol\'attrica pues la imagen no es Dios y el incienso es s\'olo para Dios}.

Continuando con el relato de la esposa y su accidente:

\begin{quote}
\emph{?`Qu\'e har\'a la mujer con la foto de su difunto esposo?}

\emph{Esa mujer besar\'a la foto, pero no por la foto misma sino con la intenci\'on de besar al esposo, tendr\'a la foto frente a ella y le hablar\'a pero no por la foto sino por el esposo a quien ama.}
\end{quote}

Es evidente que la mujer no esta enamor\'andose de la foto, sino que la usa como instrumento para recordar al esposo\ldots tambi\'en es evidente que cuando le habla a la foto o la besa su intenci\'on es enviarle el beso y las palabras al esposo f\'{i}sicamente ausente.

Para ponerlo mas claro (perdona el ejemplo pero es para darme a entender con claridad): \emph{imag\'{i}nate por un momento que le muestras a un amigo la foto de tu amada esposa ?`si la amas verdad? y \'el toma esa fotograf\'{i}a, la rompe, la tira al piso, la escupe y la pisotea}. ?`C\'omo te sentir\'{i}as? seguramente muy ofendido, ?`pero por qu\'e si la foto claramente \emph{no es} tu esposa? porque la \emph{representa} y adem\'as \emph{es evidente que esas acciones denigrantes van dirigidas a tu esposa representada en la foto.}

Esto es precisamente lo que dice el II Concilio de Nicea (a\~no 787) cuando dice que \emph{la intenci\'on del honor de la imagen se dirige al original}.

Por lo tanto \emph{quemarle incienso a un crucifijo} es dirigirle incienso a Jesucristo que muri\'o por nosotros.

%%%%%%%%%%%%%%%%%%%%%%%%%%%%%%%%%%%%%%%%%%%%%%%%%%%%%%%%%%%%%%%%%%%%%%%
% \section{Insostenibilidad del principio de \emph{Sola Scriptura}}

% \begin{flushright}
% \emph{<<{\ldots}pero si tardo, para que sepas cómo hay que portarse\\
% en la casa de Dios, que es la Iglesia del Dios vivo,\\
% columna y fundamento de la verdad>>}.\\
% 1 Tim 3,15 LBJ
% \end{flushright}

% \noindent
% El esfuerzo de descalificar el culto cat\'olico a las im\'agenes parte del presupuesto de que el principio llamado \emph{Sola Scriptura} es v\'alido, m\'as a\'un, que forma parte de la misma revelaci\'on.

% Sin embargo esto no es verdad; tal argumento es una \emph{tradici\'on de hombres} (\emph{cf.} Mt 15,1-9; Mc 7,6-13) muy posterior, aproximadamente 1500 a\~nos despu\'es de que Jesucristo envi\'o a sus ap\'ostoles a predicar.

% Voy a enunciar una serie de razones por las que considero que el principio de \emph{Sola Scriptura} es inv\'alido.

% En primer lugar, est\'a el hecho de que t\'u mismo usaste argumentos que no est\'an en la Biblia para sostener tu posici\'on. En segundo lugar, el principio de \emph{Sola Scriptura} no es b\'{i}blico. En tercer lugar, este mismo principio deja sin explicaci\'on a la misma Biblia.

% En torno a estas tres ideas giran las razones que enuncio a continuaci\'on para mostrar la invalidez de considerar la Biblia como \emph{\'unica regla de fe}.

% \subsection{En qu\'e consiste el principio de \emph{Sola Scriptura}}


% \subsection{Tus argumentos se alejan del principio de \emph{Sola Scriptura}}


% \subsection{Invalidez del argumento: <<Si no hay ejemplos en la Biblia entonces es falso/malo/inv\'alido>>}


% \subsection{La <<autoridad final>> seg\'un la Biblia}


% \subsection{El texto b\'{i}blico requiere interpretaci\'on}


% \subsection{Insuficiencia de informaci\'on en los textos: un ejemplo: la fiesta del Yom Kippur}


% \subsection{Un par\'entesis: Suficiencia Material y Formal en la Biblia (tomado de Dave Armstrong)}

% \begin{verbatim}
%     #VI. Material and Formal Sufficiency of Scripture
%     <http://socrates58.blogspot.mx/2006/11/bible-church-tradition-canon-index.html>
%     <http://socrates58.blogspot.mx/2004/04/material-vs-formal-sufficiency-of.html>
% \end{verbatim}

% \subsection{Si el principio de \emph{Sola Scriptura} fuera v\'alido la Biblia no existir\'{i}a}


% \subsection{?`En qu\'e libro de la Biblia viene qu\'e libros son inspirados? La formaci\'on del canon}


% \subsection{Trasfondo hist\'orico del principio \emph{Sola Scriptura}}


% \subsection{El principio de \emph{Sola Scriptura} termina haciendo del sujeto que lee, <<la autoridad final>>}

%%%%%%%%%%%%%%%%%%%%%%%%%%%%%%%%%%%%%%%%%%%%%%%%%%%%%%%%%
\section{Posibles objeciones y sus respuestas}

% \begin{enumerate}
% \item Sobre la leg\'{i}tima defensa
% \item Si el argumento de excepci\'on fuera inv\'alido cualquiera podr\'{i}a inventar excusas para violar los preceptos divinos
% \item Los cat\'olicos se inclinan, reverencian y sirven a las im\'agenes, ponen su confianza en ellas
% \item Objeci\'on contra la expresi\'on: <<la intenci\'on remite al original>> usada en el concilio de Efeso
% \item Sobre <<interpretar>> las Escrituras
% \item Sobre el culto en la fiesta del \emph{Yom Kippur}
% \item <<Ustedes han anulado el mandato de Dios con vuestra tradici\'on>> (\emph{c.f.} Mc 7,9)
% \item La Iglesia Cat\'olica se desacredita por sus obras, no puede ser de Dios, <<por sus frutos los conocer\'as>> (\emph{c.f.} Mt 7,16.20; Lc 6,44)
% \end{enumerate}

% \subsection{Sobre la leg\'{i}tima defensa}

% <<No existe tal cosa llamada leg\'{i}tima defensa>>

% El mandamiento en Ex 20,13 si consideramos su original en hebreo es: <<no asesinar\'as>>, es decir, no privar a alguien de la vida injustamente.

% \subsection{Si el argumento de excepci\'on fuera inv\'alido cualquiera podr\'{i}a inventar excusas para violar los preceptos divinos}

% \subsection{El argumento de excepci\'on aplicado al segundo mandamiento se refiere a hacer im\'agenes, no a la idolatr\'{i}a}

% \subsection{El Arca, el Templo, la serpiente de bronce, siendo dise\~nados por Dios no nos llevan a la corrupci\'on}

% Revisa 2 Re 18,-35

% \subsection{Los cat\'olicos se inclinan, reverencian y sirven a las im\'agenes, ponen su confianza en ellas}


% \subsection{Objeci\'on contra la expresi\'on: <<la intenci\'on remite al original>> usada en el concilio de Efeso}


% \subsection{Sobre <<interpretar>> las Escrituras}


% \subsection{Sobre el culto en la fiesta del \emph{Yom Kippur}}


% \subsection{<<Ustedes han anulado el mandato de Dios con vuestra tradici\'on>> (\emph{c.f.} Mc 7,9)}


% \subsection{La Iglesia Cat\'olica se desacredita por sus obras, no puede ser de Dios, <<por sus frutos los conocer\'as>> (\emph{c.f.} Mt 7,16.20; Lc 6,44)}


\section{Conclusi\'on}

\nocite{*}
\bibliographystyle{plain}
\bibliography{references}
\end{document}
