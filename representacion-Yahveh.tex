?`La Biblia permite hacer representaciones de Yahveh en el Antiguo Testamento? ?`Ha cambiado algo con la Nueva Alianza que permita o autorice la elaboraci\'on de im\'agenes? ?`En qu\'e consiste la \emph{veneraci\'on} cat\'olica de las im\'agenes?

Despu\'es de responder estas preguntas podremos determinar si la pr\'actica cat\'olica sobre las im\'agenes es conforme con el mandato de Dios o se le opone.

\subsection{Imposibilidad en el Antiguo Testamento de representar a Yahveh}

?`La prohibici\'on de hacer im\'agenes abarca \'unicamente a los \'{i}dolos o tambi\'en proh\'{i}be \emph{hacer representaciones de Yahveh}?

El libro de los jueces (Jue 17,1-13;18,18-31) menciona la efigie de Mik\'a y la menciona como una representaci\'on de Yahveh; pero m\'as adelante al mencionar c\'omo los danitas se llevan la imagen, Mik\'a les reclama que <<se llevaron a su \emph{dios}>> (Jue 18,24) con lo cual pone de manifiesto que estaba haciendo de esa imagen algo muy cercano a un \'{i}dolo.

Tomo este ejemplo para abrir el tema sobre representar a Yahveh en el Antiguo Testamento. La respuesta es negativa.

As\'{i} que en Ex 20,3-6 tenemos dos prohibiciones superpuestas: a) adorar otros dioes (con todo lo que implica hacerles im\'agenes y reverenciarlas, etc) y b) hacer im\'agenes de Yahveh. Sin embargo los motivos son enteramente diferentes, como enteramente diferentes son el Dios verdadero y los falsos dioses.

Cito a continuaci\'on a G. Barbaglio que en el Nuevo Diccionario de Teolog\'{i}a moral (Ed. Paulinas, 1992) dice lo siguiente al comentar sobre el dec\'alogo en el libro del Deuteronomio:\footnote{http://www.mercaba.org/DicTM/TM\_decalogo.htm el \emph{subrayado} es m\'{i}o.}

\begin{quote}
1) El segundo mandamiento: <<No te harás ídolos ni imagen alguna>> (cf ZIÑihÍER1.1, Das Zweite Gebot; G. vorr RAD, 246-254), prohibe representar en estatuas a la divinidad. \emph{Se refería originariamente a representaciones de Yhwh}. El precepto contrastaba con la costumbre de los pueblos vecinos, que consideraban la estatua como el medio del encuentro con Dios y con su revelación. El sentido del mandamiento no es el de salvaguardar la espiritualidad de Yhwh, preocupación ésta ausente en Israel y ajena a la linea del significado que revestía la estatua en los ambientes circundantes. \emph{Se quería, con ello, proteger la libertad de Yhwh, que no es un Dios que el hombre pueda aferrar ni [est\'a]\footnote{Esta palabra no est\'a en el original pero me parece que es la m\'as adecuada para que el enunciado tenga sentido.} sometido a la limitación de sus fieles. Por medio de la estatua, en la que se consideraba presente a la divinidad, se pretendía dominarla para someterla a los propios deseos}. Medio exclusivo de revelación de Yhwh al pueblo y ámbito único de encuentro es su palabra y su acción en la historia.

2) La posterior ampliación deuteronomista del mandamiento: <<No te postrarás ante ellos y no les servirás>>, \emph{constituye claramente una repetición del tema del primer mandamiento}, en cuanto que en él ya estaban prohibidos la adoración y el culto dedicados a los ídolos. Lo que significa que para la interpretación deuteronomista posterior, \emph{la prohibición de hacer estatuas, originariamente aplicada a las representaciones de Yhwh, se refiere a los ídolos de los dioses extranjeros y a su culto}. Y por lo tanto, según esta interpretación, \emph{la prohibición de las imágenes no constituye ya un segundo mandamiento distinto del primero, sino más bien la continuación y el desarrollo de este último}. Tendríamos así una numeración distinta de los diez mandamientos según la actualización deuteronomista, que, como veremos más adelante, para restablecer el número de diez desdoblará el último mandamiento.\footnote{Revisa la url para ver c\'omo resuelven ellos la \emph{divisi\'on} del \'ultimo mandamiento en dos.}

Sigue después la motivación del exclusivo reconocimiento de Yhwh: <<Porque yo, Yhwh, soy tu Dios, un Dios celoso, que castiga la culpa de los padres en los hijos hasta la tercera y cuarta generación para los que me odian pero que demuestra su favor en mil generaciones con quienes me aman y observan mis mandamientos>>. Yhwh es un Dios celoso, \emph{atento con todas sus fuerzas y su energía a afirmar su derecho frente al pueblo y a no tolerar a ningún otro como Dios de aquéllos a quienes él ha liberado de Egipto}, dispuesto a castigar la culpa de la infidelidad, pero infinitamente benévolo con quienes lo aman y le son fieles.
\end{quote}

\noindent
Voy a enumerar las ideas principales:

\begin{enumerate}
\item Hay una prohibici\'on de hacer representaciones de Yahveh
\item \textbf{El \emph{motivo} es para salvaguardar la \emph{libertad} de Dios, que no est\'a sometido (por medio de im\'agenes) al hombre. M\'as exactamente, que al pueblo le quede muy claro que a Dios no se le puede controlar.}
\item Sin excluir esta idea, el mismo mandamiento es un desarrollo de Dt 5,7
\item Lo cual queda reforzado por los \emph{celos} de Dios
\end{enumerate}

\noindent
He resaltado la idea que considero m\'as importante: \emph{el motivo para prohibir la representaci\'on de Dios es quitarle la tentaci\'on al hombre de querer <<controlar>> a Dios}.
%Esto queda confirmado por lo que pas\'o con la efigie de Mik\'a (Jue 17,1-13;18,18-31), pues terminaron \emph{reduciendo} al Dios alt\'{i}simo a un <<dios>> atrapado en una imagen.

% Esta idea se ve reforzada por el cap\'{i}tulo 4 del Deuteronomio, donde se le recuerda al pueblo que cuando Dios habl\'o en el Horeb enmedio del fuego, el pueblo de Israel no vio figura alguna (Dt 4,15) y si intentan hacer alguna representaci\'on de Dios (a quien no vieron) entonces terminar\'an adorando a lo que no es Dios (Dt 4,16-19).

% De aqu\'{i} que sea \emph{impensable} en el A.T. tener representaciones de Dios. No se puede representar lo que no se ve (\'El es \emph{trascendente}) y no se le puede controlar mediante una imagen (\'El es \emph{libre}); y aquellos que intentan representarlo y/o controlarlo terminan, en su imaginaci\'on, reduciendo al Dios alt\'{i}simo a un \'{i}dolo.

\subsubsection{Salvaguardar la libertad de Yahveh}

\emph{Someter} a Dios y ponerlo a nuestro servicio es una tentaci\'on permanente. Con o sin im\'agenes. Puede ser que \emph{exteriormente} se est\'e reverenciando, al Dios alt\'{i}simo, pero \emph{interiormente} se quiera \emph{forzar} a Dios a cumplir los propios deseos.

Pongo tres casos a tu consideraci\'on:

\begin{enumerate}
\item La efigie de Mik\'a (Jue 17,1-13;18,18-31)\footnote{En la versi\'on Reina-Valera 1960 menciona que eran muchos dioses y no uno solo, mientras que en la N\'acar Colunga 1944 menciona que es uno s\'olo; no he podido validar las traducciones contra el original en hebreo -- entre otras cosas porque no se hebreo -- pero en caso de que la forma correcta sea la de RVR entonces la argumentaci\'on mantiene su curso pero sin el ejemplo de Jue 17-18.}
\item La segunda tentaci\'on de Jes\'us (Mt 4,5-7)
\item El fariseo y el publicano (Lc 18,9-14)
\end{enumerate}

\paragraph{La efigie de Mik\'a (Jue 17,1-13;18,18-31)}

Mik\'a tiene una idea \emph{reducida} de Dios, cree que lo puede \emph{encerrar} en una imagen y que lo favorezca (Jue 17,13); cuando le quitan la imagen lo que reclama es que le quitan a su dios, donde <<su>> significa de su propiedad: <<mi dios, \emph{el que yo he hecho}>> (\emph{c.f.} Jue 18, 24).

El problema no es tan simple como parece a primera vista: no se trata, simplemente, de que haga una \emph{representaci\'on} f\'{i}sica de Dios y por este acto su coraz\'on quede corrompido y su mente quede confundida; se trata de que al querer representar a Yahveh, \emph{?`Con qu\'e figura lo va a hacer si nunca lo ha visto?} (Dt 4,15). Cualquier forma que haya elegido manifiesta que ha <<moldeado>> a Dios seg\'un su propia idea.

\paragraph{La segunda tentaci\'on de Jes\'us (Mt 4,5-7)}

M\'as all\'a de la representaci\'on f\'{i}sica, el problema va en el sentido de si el Dios \emph{real} se acomoda a \emph{mis} deseos y conceptos o si yo busco obedecerle \emph{tal cual es}.

Satan\'as reta a Jes\'us a que se lance desde lo alto del tempo y ponga a prueba la fidelidad de Dios, para esto cita el salmo 91 (90); si Dios realmente es quien dice ser que lo cumpla y lo pongo a prueba \emph{a ver si me convence}.

Por eso, la respuesta de Jes\'us es completamente exacta (Mt 4,7 \emph{c.f.} Dt 6,16).

\paragraph{El fariseo y el publicano (Lc 18,9-14)}

El problema va m\'as all\'a. Tambi\'en significa pretender \emph{obligar} a Dios a cumplir \emph{su parte del trato}; los fariseos se sienten \emph{merecedores} del favor divino por \emph{cumplir la ley}; los profetas, especialmente Isa\'{i}as y Jerem\'{i}as criticaron esa actitud del que cumple con los preceptos de pureza ritual pero descuidan la justicia.

Finalmente el fariseo que sube al templo a orar, seg\'un nos cuenta Lucas, le agradece a Dios no ser \emph{como los dem\'as hombres}, los decribe incluyendo a <<\emph{ese publicano}>> y despu\'es enumera sus \emph{m\'eritos} ante Dios. No lo menciona pero el silencio da muy bien a entender que interiormente lo que le est\'a diciendo a Dios es algo como lo siguiente: <<\emph{Yo si cumplo con la ley, y t\'u ?`qu\'e me vas a dar a cambio?}>>.

\paragraph{Resumiendo}

Es \emph{impensable} en el A.T. tener representaciones de Dios. No se puede representar lo que no se ve (\'El es \emph{trascendente}) y no se le puede controlar mediante una imagen (\'El es \emph{libre}); y aquellos que intentan representarlo y/o controlarlo terminan, en su imaginaci\'on, reduciendo al Dios alt\'{i}simo a un \'{i}dolo.

En Ex 20,3-6 y Dt 5,7-10 hay dos prohibiciones superpuestas: a) adorar otros dioses y b) pretender \emph{controlar} a Yahveh. Ambas prohibiciones tienen sus \emph{manifestaciones externas}: hacerse im\'agenes, poner a Dios a prueba, imponerle a Dios las propias condiciones\ldots Pero el \emph{acto interior} es el que da la pauta moral y no la \emph{manifestaci\'on externa}.
